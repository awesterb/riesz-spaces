\documentclass[main.tex]{subfiles}
\begin{document}
% 1
\section{Preliminaries}
%
%                  1.1
%
\begin{psec}[Definitions]
\label{1.1}
\begin{enumerate}
% 1
\item 
A \keyword{lattice} is a (partially) ordered set in which
every nonempty finite subset has an infimum and a supremum. 
If $a,b$ are elements of a lattice~$\Lambda$ 
we write~$a\wedge b = \inf\{a,b\}$
and~$a\vee b = \sup\{a,b\}$.
% 2
\item 
A lattice $\Lambda$ is \keyword{distributive}
if for all $a,b,c$ in $\Lambda$
\begin{align*}
(a\vee b)\wedge c &= (a\wedge c)\vee (b\wedge c)\htam{, } &
(a\wedge b)\vee c &= (a\vee c) \wedge (b\vee c) 
\end{align*}
% 3
\item 
A \keyword{Boolean algebra}~$\Lambda$ is a distributive lattice
with a smallest element~$0$
and a largest element~$1$
such that for every~$x$ in~$\Lambda$
there is a \keyword{complement},
i.e., 
an element~$y$ of~$\Lambda$
with~$x\wedge y=0$, $x\vee y=1$.
In a Boolean algebra the complement is unique.
The complement of~$x$ is denoted:~$x'$.
% 4
\item 
Let~$X$ be a set.
A collection~$\mathcal{R}$ 
of subsets of~$X$
is called a \keyword{ring} if
\begin{align*}
&A,B\in\mathcal{R} \quad \implies \quad 
  A\cap B \in \mathcal{R}\htam{, }\ 
  A\cup B \in \mathcal{R}\htam{, }\ 
  A\backslash B \in \mathcal{R}\htam{; } \\
&\varnothing \in \mathcal{R} 
\htam{.} \\
\intertext{%
A ring~$\mathcal{R}$ is an \keyword{algebra} if%
}
&X \in \mathcal{R}
\htam{.} \\
\intertext{%
(An algebra of sets is a Boolean algebra.) 
A $\sigma$-algebra in~$X$
is an algebra~$\mathcal{R}$
of subsets of~$X$
with%
}
& A_1, A_2, \ldots \in \mathcal{R}\quad \implies\quad 
  \bigcup_n A_n \in \mathcal{R}
\htam{.}
\end{align*}
\end{enumerate}
\end{psec}
%
%                  1.2
%
\begin{psec}[Definition]
\label{1.2}
An \keyword{ordered vector space} 
is a (real) vector~space~$E$
endowed with an ordering such that
\begin{equation*}
x,y\in E\htam{, }\ x\leq y
 \quad \implies \quad
\begin{cases}
x+a \leq y+a  
  & (a\in E)\htam{, } \\
\lambda x \leq \lambda y 
  & (\lambda \in [0,\infty)\,)\htam{. }
\end{cases}
\end{equation*}
Observe that then for all $x,y\in E$
\begin{equation*}
x\leq y \quad \implies \quad -x \geq -y\htam{.}
\end{equation*}
(Take $a:=-x-y$.)
\end{psec}
%
%                  1.3
%
\begin{psec}[Definition]
\label{1.3}
A \keyword{Riesz space} ($=$ \keyword{vector lattice})
is an ordered vector space that is a lattice.

If~$E$ is an ordered vector space 
in which for every element~$x$
the set~$\{x,-x\}$ has a supremum,
then~$E$ is a Riesz space;
indeed,
for $x,y\in E$
we then have
\begin{align*}
\frac{x+y}{2} + \sup\left\{ \frac{x-y}{2}, -\frac{x-y}{2} \right\} 
  & = x\vee y\htam{,} \\ 
\frac{x+y}{2} - \sup\left\{ \frac{x-y}{2}, -\frac{x-y}{2} \right\}
  & = x\wedge y\htam{. }
\end{align*}
\end{psec}
%
%                  1.4
%
\begin{psec}{Elementary properties}
\label{1.4}
Let~$E$ be a Riesz space. 
We collect a number of basic identities and inequalities.
For the sake of readability
we occasionally omit clauses such as ``for all~$x$ in~$E$''.
\begin{enumerate}
\item % 
\label{1.4-1}
Let~$X\subseteq E$, and let~$A\colon E\ra E$ 
be an order isomorphism.
\snote{``order isomorphism'' has not been defined.}
If~$X$ has a supremum,
then so does $A(X)$,
and
\begin{align*}
\sup A(X) &= A(\sup X)
\htam{.} \\
\intertext{%
Similarly, %
}
\inf A(X) &= A(\inf X)
\htam{.}
\end{align*}
\item %
\label{1.4-2}
If~$x,y\in E$, then
\begin{equation*}
\frmd{ (x\vee y) = (-x)\wedge(-y) }
\htam{.}
\end{equation*}
\item %
\label{1.4-3}
It follows from \ref{1.4-1} that
\snote{So: $\lambda\cdot -$ is an order isomorphism.} 
\begin{alignat*}
\lambda \sup X &= \sup \lambda X \qquad &&(\lambda \in [0,\infty)\,) \\
\intertext{%
whenever~$X\subseteq E$ and~$\sup X$ exists.
In particular, }
\lambda (x \vee y) &=  (\lambda x) \vee (\lambda y) 
  \qquad &&(x,y\in E\htam{; }\lambda\in[0,\infty)\,)\htam{.}
\end{alignat*}
\item %
\label{1.4-4}
For~$X\subseteq E$ and $a\in E$,
set~$X+a:=\{x+a\colon x\in X\}$.
By \ref{1.4-1} we have
\begin{align*}
\sup(X+a) &= (\sup X) + a \\
\intertext{%
if $X\subseteq A$ and $\sup X$ exists.
Similarly, }
\inf (X+a) &= (\inf X)+a
\end{align*}
if~$X$ has an infimum.
\item %
\label{1.4-5}
In particular,
\snote{So: $-+a$ is an order isomorphism.}
\begin{equation*}
\label{i1.4-5}
\frmd{%
(x+a)\vee   (y+a)=(x\vee y)+a
\htam{, } \qquad 
(x+a)\wedge (y+a)=(x\wedge y)+a }
\htam{.}
\end{equation*}
\item %
\label{1.4-6}
\begin{equation*}
x\vee y + x\wedge y = x+y
\end{equation*}
because~$(x\vee y)-x-y 
\isref{1.4-5}
(x-x-y)\vee(y-x-y)=(-y)\vee(-x)=x\wedge y$.
\item %
\label{1.4-7}
Take~$a\in E$ and let $M_a\colon E\ra E$ 
be either the map~$x\mapsto x\vee a$
or~$x\mapsto x\wedge a$. 
Let~$A\subseteq E$.
Then
\begin{align*}
M_a(\sup X) &= \sup M_a(X) \\
\intertext{%
if $X$ has a supremum and}
M_a(\inf X) &= \inf M_a(X)
\end{align*}
if~$X$ has an infimum.
Proof for $M_a(x)=x\vee a$:
\begin{itemize}
\item %
If~$X$ has a supremum:  
For~$z\in E$ we
\begin{align*}
z\ge (\sup X)\vee a  
  & \iff z\ge \sup X\ \htam{ and }\  z\ge a \\
  & \iff z\ge x\quad (x\in X)\ \htam{ and }\  z\ge a\\
  & \iff z\ge x\vee a\quad (x\in X) \\
  & \iff z\htam{ is upper bound for }M_a(X)\htam{.}
\end{align*}
\item %
If~$X$ has an infimum: 
The fact that $M_a$ is increasing 
implies that $M_a(\inf X)$ is a lower bound
for $M_a(X)$.
Now let~$z$
be any lower bound for~$M_a(X)$;
we prove~$z\leq M_a(\inf X)$.
For all~$x\in X$
we have
$z\leq x\vee a 
\isref{1.4-6}
x+a - x\wedge a
\leq x+a-(\inf X)\wedge a$,
whence
$x\ge z-a+(\inf X)\wedge a$.
Then $\inf X\geq z-a+(\inf X)\wedge a$
so that 
$z\leq \inf X +a - (\inf X)\wedge a
\isref{1.4-6}
(\inf X)\vee a = M_a(\inf X)$.
\end{itemize}
\item %
\label{1.4-8}
Hence, $E$ is distributive:
\begin{equation*}
\frmd{(x\wedge y)\vee a = (x\vee a)\wedge(y\vee a)
\htam{, }\qquad
(x\vee y)\wedge a = (x\wedge a)\vee (y\wedge a)}
\htam{.}
\end{equation*}
\item %
\label{1.4-9}
Define
\begin{equation*}
\frmd{%
x^+ := x\vee 0\htam{, }\qquad
x^- := (-x)\vee 0\htam{, }\qquad
|x| := x\vee(-x)}
\htam{.}
\end{equation*}
Then 
\begin{equation*}
\frmd{ x = x^+ - x^- }
\end{equation*}
since $x^+ - x^-
\isref{1.4-2}(x\vee 0)+(x\wedge 0) 
\isref{1.4-6}x+0$.

Furthermore:
$x^+ + x^- 
= x^+ + (x^+ - x) 
= 2(x\vee 0) -x
= (2x)\vee 0 -x
\isref{1.4-5} (2x-x)\vee(0-x) = x\vee(-x) = |x|$:
\begin{equation*}
\frmd{|x| = x^+ + x^-}
\end{equation*}
which implies
\begin{equation*}
\frmd{|x|\ge 0}
\end{equation*}
and 
thereby~$|x|=|x|\vee 0
= x\vee(-x)\vee 0 \isref{1.4-7} (x\vee 0) \vee (-x\vee 0)$,
i.e., 
\begin{equation*}
\frmd{|x| = x^+ \vee x^-}
\htam{.}
\end{equation*}
Hence, $x^+ + x^- = x^+ \vee x^-$,
so that, 
by \ref{1.4-6},
\begin{equation*}
\frmd{x^+ \wedge x^- = 0} 
\htam{.}
\end{equation*}

From the definition of $|-|$
it is apparent that~$|-x|=|x|$.
Hence, 
\begin{equation*}
\frmd{|\lambda x|=|\lambda| |x|} 
  \qquad (\lambda\in\R\htam{, }\ x\in E)\htam{.}
\end{equation*}
Of course,
$x=x^+ - x^-\leq x^+ + x^- = |x|$
and $-x\leq|-x|=|x|$,
so $-|x|\leq x \leq |x|$.
Consequently,
\begin{equation*}
\frmd{|x|=0 \quad \implies \quad x=0}
\htam{.}
\end{equation*}
%
\item%
\label{1.4-10}
If $x,y\in E$,
then $x^+\geq x$, 
$y^+\geq y$,
so $x^+ + y^+ \geq x+y$.
Also,
$x^+ + y^+\geq 0$.
Thus
\begin{align*}
&\frmd{(x+y)^+ \leq x^+ + y^+}\\
\intertext{ and, 
similarly, 
$(x+y)^- \leq x^- + y^-$.
It follows that}
&\frmd{|x+y|\leq|x|+|y|}
\htam{.}
\end{align*}
From this,
it follows that
\begin{equation*}
||x|-|y|| \leq |x-y| 
\htam{.}
\end{equation*}

In \ref{1.3} 
we have already observed:
\[ \frmd{
\begin{aligned}
x \vee y 
  &= \frac{x+y}{2} + \left | \frac{x-y}{2} \right | \\
x \wedge y 
  &= \frac{x+y}{2} - \left | \frac{x-y}{2} \right | 
\htam{.}
\end{aligned} } \]
%
\item
\label{1.4-11}
Define the relation $\perp$ on~$E$ by
\begin{equation*}
\frmd{x \perp y \quad \iff \quad |x|\wedge|y|=0}
\htam{.}
\end{equation*}

For all $x$ we have seen that $x^+\wedge x^- = 0$,
so
\begin{equation*}
\frmd{x^+ \perp x^-}
\htam{.}
\end{equation*}
For any $x$ and $y$, 
$|x-x\wedge y|\wedge|y-x\wedge y|
=(x-x\wedge y) \wedge (y-x\wedge y)
\isref{1.4-5} x\wedge y - x\wedge y$:
\begin{equation*}
\frmd{x-x\wedge y \quad \perp \quad y- x\wedge y}
\htam{.}
\end{equation*}

It follows from the triangle inequality
 that~$x\perp y_1 + y_2$
as soon as~$x\perp y_1$ and $x\perp y_2$.
From this,
one infers easily that for every~$x$
the set~$\{ y: x\perp y\}$
is a linear subspace of~$E$.
%
\item \label{1.14-12}
The \keyword{positive cone} of~$E$ is the set
\begin{equation*}
E^+ := \{x\in E\colon x\ge 0\}
\htam{.}
\end{equation*}
\end{enumerate}
\end{psec}
%
%                  1.5
%
\begin{psec}[Examples]
\label{1.5}
\begin{enumerate}
\item
\label{1.5-1}
For any set~$X$
the functions~$X\ra \R$
form an ordered vector space~$\R^X$;
this~$\R^X$ is a Riesz space:
\begin{equation*}
(f\vee g)(x) = f(x)\vee g(x)\htam{, } \quad 
(f\wedge g)(x) = f(x)\wedge g(x)\qquad
(x\in X\htam{; }f,g\in\R^X)
\htam{.}
\end{equation*}
%
\item
\label{1.5-2}
For any set~$X$ the bounded functions on~$X$ form a 
Riesz~space~$\ell^\infty(X)$.
%
\item
\label{1.5-3}
(See below)
\snote{Add continuation.}
\end{enumerate}
\end{psec}
%
%                  1.6
%
\begin{psec}[Intermezzo]
If $E$ is an ordered vector space, 
then every linear subspace of~$E$ is an ordered vector space
under the ordering it inherits from~$E$.

Now let $E$ be a Riesz space,
$D$ a linear subspace of~$E$.
Then~$D$ is an ordered vector space.
$D$ is called a \keyword{Riesz subspace} of~$E$
if
\begin{equation*}
x,y\in D \quad \implies \quad x\vee y\in D\htam{, }\quad x\wedge y\in D
\htam{.}
\end{equation*}
In this case,
$x\vee y$ and $x\wedge y$ are the least upper bound
and the greatest lower bound of~$\{x,y\}$ in the ordered set~$D$,
so that~$D$ is a Riesz space and the lattice operations on~$D$
are the restrictions of those on~$E$.
(See, however, Exercise~\ref{1.7}.)

With the formulas given in~\ref{1.3} one sees
that a linear subspace~$D$ of~$E$ is a Riesz subspace if
\begin{align*}
x\in D\quad &\implies \quad |x|\in D
\htam{,} \\ 
\intertext{%
or, 
equivalently
(since $|x|=x^+ + (-x)^+$), if}
x\in D \quad &\implies \quad x^+ \in D
\htam{.}
\end{align*}
\end{psec}
%
%                  1.5 --- part 2
%
\begin{psec}[``1.5 continued'']
\label{1.5_2}
\snote{Fix.}
Many Riesz spaces are obtained as subspaces of~$\R^X$:
\snote{This enumeration should start with (3).}
\begin{enumerate}
\item
\label{1.5-3}
For a topological space~$X$ the space $\Cont{X}$
of all continuous function~$X\ra\R$ is a Riesz subspace of~$\R^X$.
The space~$\BCont{X}$ of all bounded continuous functions 
is a Riesz subspace of~$\Cont{X}$,
hence of~$\R^X$.
%
\item
\label{1.5-4}
The differentiable functions on~$[0,1]$ 
do not form a Riesz space
under pointwise ordering:
there is no smallest differentiable function
that majorizes both~$x\mapsto x$ and~$x\mapsto 1-x$.
(But see~\ref{1.7}.)
%
\item
\label{1.5-5}
For $X=\N$: 
The space $\cseq$ of all convergent sequences 
is a Riesz subspace of~$\R^\N$,
as are the space~$\cseq_0$
of all sequences that converge to~$0$,
and the space~$\ell^1$
of all sequences~$(x_n)_n$
for which~$\sum|x_n|$ is finite.
%
\item
\label{1.5-6}
The sequences~$(x_n)_n$ for which~$\sum x_n$ exists (and is finite)
do not form a Riesz space:
with~$u=(1,\,-\sfrac{1}{2},\, \sfrac{1}{3},\, -\sfrac{1}{4},\, \ldots)$
the set~$\{u,-u\}$
does not even have an upper bound.
%
\item
(See below)
\end{enumerate}
\end{psec}
%
%                  1.7
%
\begin{psec}[Exercise]
\label{1.7}
A function~$f$ on~$[0,1]$ is called \keyword{affine}
if there exist numbers~$a$ and~$b$ 
such that~$f(x)=ax+b\quad (x\in[0,1])$.
Show that the vector space of all affine functions on~$[0,1]$
is a Riesz space under the natural ordering,
but not a Riesz subspace of~$\R^{[0,1]}$ or~$\Cont{[0,1]}$.
\end{psec}
%
%                  1.5 --- part 3
%
\begin{psec}[``1.5 continued'']
\label{1.5_3}
\snote{Psec should have number 1.5 and enumeration should start with (7)}
\begin{enumerate}
\item
\label{1.5-7}
If $\mathcal{R}$ is a ring of subsets of a set~$X$,
the $\mathcal{R}$-step functions form a 
Riesz subspace~$\lbb \mathcal{R} \rbb$ of $\R^X$.
%
\item
\label{1.5-8}
If $\mathcal{A}$ is a $\sigma$-algebra 
in\snote{Should it be ``on''?} set~$X$,
the $\mathcal A$-measurable functions on~$X$
form a Riesz subspace~$\mathcal F_{\mathcal A}$ of~$\R^X$.
%
\item
\label{1.5-9}
If $(X,\mathcal A,\mu)$ is a measure space,
the space~$\mathcal L_{\mathcal A}[\mu]$
of all $\mathcal A$-measurable 
$\mu$-integrable functions is a Riesz subspace
of~$\mathcal F_{\mathcal A}$.
\end{enumerate}
\end{psec}
%
%                  1.8
%

The construction of the next example is more involved. 
(See~\ref{1.10}.)
\snote{We do not want an indent here. 
Do we want more space above then below since this remark relates 
to the text below?}
\begin{psec}
\label{1.8}
Consider a real-valued function~$\varphi$ on a lattice~$\Lambda$.
\begin{enumerate}
\item
\label{1.8-1}
For $a,b\in\Lambda$ with $a\leq b$
we define the variation of $\varphi$ on $[a,b]$
\begin{multline*}
\Var{[a,b]}{\varphi}
  \ :=\  \sup\bigl\{ 
         \textstyle{\sum_{n=1}^{N}} |\varphi(s_n) - \varphi(s_{n-1})|\colon
         N\in\N\htam{; } \\
         a=s_0 \leq s_1 \leq \ldots \leq s_N = b \bigr\}
\htam{.}
\end{multline*}
(Possibly, the variation is $\infty$.)
If $\varphi$ is increasing, 
$\Var{[a,b]}{\varphi} = \varphi(b)-\varphi(a)$.
%
\item
\label{1.8-2}
$\varphi$ is said to be \keyword{modular} if
\begin{equation*}
\varphi(s\vee t) + \varphi(s \wedge t) = \varphi (s) + \varphi (t)
\qquad (s,t\in \Lambda)
\htam{.}
\end{equation*}
E.g., if $\Lambda$ is a Riesz space, 
every linear function is modular (\ref{1.4}\ref{1.4-6}).
If $\Lambda$ is a ring of sets,
additive functions are modular.
On~$[0,1]$, all functions are modular.
On $\Cont{[0,1]}$, 
the function~$\smash{f\mapsto\int_0^1 f(x)^{37} dx}$ is modular.
\end{enumerate}
\end{psec}
%
%                  1.9
%
\begin{psec}[Lemma]
\snote{Text should have emphasis.}
\label{1.9}
Let $\varphi$ be a modular real-valued function 
on a lattice~$\Lambda$
and let~$a,b,c\in\Lambda$, $a\leq b\leq c$.
Then
\begin{equation*}
\Var{[a,c]}{\varphi} 
  = \Var{[a,b]}{\varphi} + \Var{[b,c]}{\varphi}
\htam{.}
\end{equation*}
\end{psec}
%\begin{proof}
\textbf{Proof} 
\snote{Add proof environment. }
The proof of the inequality ``$\geq$'' is straightforward.
For the reverse,
assume~$a=s_0\leq s_1 \leq \ldots \leq s_N = c$.
Then
\begin{equation*}
a \leq s_0\wedge b \leq s_1 \wedge b \leq \ldots s_N\wedge b
  = b = s_0 \vee b \leq s_1 \vee b \leq \ldots s_N\vee b = c
\htam{.}
\end{equation*}
For each $n$,
\snote{Flush the even lines to the right.}
\begin{align*}
\bigl|\varphi(s_n) - \varphi(s_{n-1})\bigr| 
  & \ \leq\  \left|\bigl(\varphi (s_n) - \varphi(s_n\vee b)\bigr) 
      - \bigl(\varphi(s_{n-1})  - \varphi(s_{n-1} \vee b)\bigr)\right| \\
 & \qquad  + \bigl|\varphi(s_n\vee b) - \varphi(s_{n-1}\vee b)\bigr| \\ 
 & \ = \ \bigl|\bigl(\varphi(s_n \wedge b) - \varphi(b)\bigr)-
       \bigl(\varphi(s_{n-1}\wedge b) - \varphi(b)\bigr)\bigr| \\
 & \qquad + \bigl|\varphi(s_n \vee b) - \varphi(s_{n+1} \vee b)\bigr| \\
 & \ = \ \bigl| \varphi(s_n\wedge b) -\varphi(s_{n-1}\wedge b) \bigr|
       + \bigl| \varphi(s_n \vee b) - \varphi(s_{n-1}\vee b) \bigr|
\htam{.}
\end{align*}
Therefore,
$\smash{\sum_n}|\varphi(s_n)-\varphi(s_{n-1})|
  \leq \Var{[a,b]}{\varphi} + \Var{[b,c]}{\varphi}$.
The lemma follows.
%\end{proof}
%
%                  1.10
%
\begin{psec}[Theorem]
\snote{Text should be with emphasis.}
\label{1.10}
Let $\Lambda$ be a distributive lattice with a smallest
element, $0$. 
Form the set~$\Mod(\Lambda)$
of all functions~$\varphi\colon\Lambda\ra\R$ satisfying
\begin{align*}
&\varphi\htam{ is modular,} \\
&\varphi(0) = 0\htam{, } \\
&\Var{[0,a]} \varphi < \infty\htam{ for all }a\htam{ in }\Lambda
\htam{.}
\end{align*}
Endow $\Mod(\Lambda)$ with the ordering $\leq$ defined by
\begin{equation*}
\varphi \leq \psi \quad \iff \quad \psi - \varphi\htam{ is increasing.}
\end{equation*}
Under this ordering, $\Mod(\Lambda)$ is a Riesz space.
\end{psec}
%\begin{proof} 
\textbf{Proof}\snote{Fix}
It will be clear that $\Mod(\Lambda)$ is an ordered vector space.

Let $\varphi\in\Mod(\Lambda)$. 
It suffices to prove that the set $\{\varphi,-\varphi\}$
has a supremum in~$\Mod(\Lambda)$. (See~\ref{1.3}.)

Define $\omega\colon\Lambda \ra \R$ by
\begin{equation*}
\omega(s) := \Var{[0,s]}\varphi \qquad (s\in \Lambda)\htam{.}
\end{equation*}
It will turn out that this~$\omega$ 
is the desired supremum of~$\{\varphi,-\varphi\}$.
\begin{itemize}
\item 
We first prove $\omega$ to be modular.
Let $a,b\in \Lambda$; we show that
$\omega(a\vee b) - \omega(a) = \omega(b) - \omega(a\wedge b)$,
or equivalently (by the lemma)
\begin{equation}
\label{eq1.10_1}  \tag{$*$}
\Var{[a,a\vee b]}\varphi = \Var{[a\wedge b,b]}\varphi 
\htam{.}
\end{equation}

With $As := s\vee a \quad (a\wedge b\leq s\leq b)$
and $Bt := t\wedge b \quad (a\leq t \leq a\vee b)$
we have increasing maps~$A\colon [a\wedge b,b]\ra [a,a\vee b]$
and~$B\colon [a,a\vee b]\ra[a\wedge b,b]$.
They are each other's inverses: 
if $s\in[a\wedge b, b]$ and $t\in [a,a\vee b]$
then~$BAs = (s\vee a)\wedge b = (s\wedge b) \vee (a\wedge b) = s$ 
\snote{Distributivity of~$\Lambda$ is not needed here; modularity suffices.}
and~$ABt = (t\wedge b)\vee a = (t\vee a)\wedge(b\vee a) = t$.

Furthermore, if $s\in [a\wedge b, b]$, then
\begin{align*}
\varphi(As) - \varphi(s) 
  & = \varphi(As) - \varphi ((As)\wedge b) \\
  & = \varphi((As)\vee b) - \varphi(b) \\
  & = \varphi(s\vee a\vee b)-\varphi(b) = \varphi(a\vee b)-\varphi(b)
\htam{,}
\end{align*}
independent of $s$.
Hence, $\varphi(As)-\varphi(As')=\varphi(s)-\varphi(s')$
for all $s,s'\in[a\wedge b,b]$.

From these considerations, \eqref{eq1.10_1} is an easy consequence.
%
\item
As $\omega$ is increasing (by the preceding lemma),
for all~$a$ in~$\Lambda$ 
we have~$\Var{[0,a]}\omega = \omega(a)-\omega(0)<\infty$.
Thus, $\omega\in\Mod(\Lambda)$.
%
\item
For $a,b\in\Lambda$ with $a\leq b$ we have
\begin{align*}
(\omega - \varphi)(b) - (\omega-\varphi)(a)
  & = (\omega(b) - \omega(a)) - (\varphi(b) - \varphi(a)) \\
  & = \Var{[a,b]}\varphi - (\varphi(b) - \varphi(a)) \ge 0
\htam{.}
\end{align*}
Thus, $\omega-\varphi$ is increasing, 
i.e., $\omega\ge \varphi$ in the ordering of~$\Mod(\Lambda)$.
Similarly, $\omega\geq -\varphi$,
so $\omega$ is an upper bound of~$\{ \varphi, -\varphi \}$.
%
\item
Finally, 
let $\psi$ be any upper bound of~$\{\varphi,-\varphi\}$;
we prove $\psi\ge \omega$,
i.e. $\psi-\omega$ is increasing.
Let $a\leq b$ in $\Lambda$.
Then 
\begin{equation*}
(\psi-\omega)(b) - (\psi-\omega)(a)
  = \psi(b) - \psi(a) - \Var{[a,b]}\varphi
\htam{; }
\end{equation*}
Thus, 
we wish to show that
\begin{equation}
\label{eq1.10_2}  \tag{$**$}
\Var{[a,b]}\varphi \ \leq\  \psi(b)-\psi(a) \htam{.}
\end{equation}
Let $a=s_0\leq s_1 \leq \ldots \leq s_N = b$.
As $\psi-\varphi$ and $\psi+\varphi$ are increasing we have
\begin{align*}
\varphi(s_n) - \varphi(s_{n-1})
  &\  \leq\  \psi(s_n) - \psi(s_{n-1}) 
\quad\htam{ and }\quad \\
-\varphi(s_n) + \varphi(s_{n-1})
  &\  \leq\  \psi(s_n) - \psi(s_{n-1})
\htam{,}
\end{align*}
i.e., 
$|\varphi(s_n) - \varphi(s_{n-1})| 
  \leq \psi(s_n) - \psi(s_{n-1})$ 
for each $n$.
Hence~$\smash{\sum} |\varphi(s_n) - \varphi(s_{n-1})|
  \leq \psi (b) - \psi(a)$.
This establishes~\eqref{eq1.10_2}.
\end{itemize}
%\end{proof}
\end{document}
