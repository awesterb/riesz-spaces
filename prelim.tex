\documentclass[main.tex]{subfiles}
\begin{document}
% 1
\chapter{Preliminaries}
%
% 1.1
\section{Definitions}
\begin{enumerate}
% 1
\item 
A \keyword{lattice} is a (partially) ordered set in which
every nonempty finite subset has an infimum and a supremum. 
If $a,b$ are elements of a lattice~$\Lambda$ 
we write~$a\wedge b = \inf\{a,b\}$
and~$a\vee b = \sup\{a,b\}$.
% 2
\item 
A lattice $\Lambda$ is \keyword{distributive}
if for all $a,b,c$ in $\Lambda$
\begin{align*}
(a\vee b)\wedge c &= (a\wedge c)\vee (b\wedge c)\htam{, } &
(a\wedge b)\vee c &= (a\vee c) \wedge (b\vee c) 
\end{align*}
% 3
\item 
A \keyword{Boolean algebra}~$\Lambda$ is a distributive lattice
with a smallest element~$0$
and a largest element~$1$
such that for every~$x$ in~$\Lambda$
there is a \keyword{complement},
i.e., 
an element~$y$ of~$\Lambda$
with~$x\wedge y=0$, $x\vee y=1$.
In a Boolean algebra the complement is unique.
The complement of~$x$ is denoted:~$x'$.
% 4
\item 
Let~$X$ be a set.
A collection~$\mathcal{R}$ 
of subsets of~$X$
is called a \keyword{ring} if
\begin{align*}
&A,B\in\mathcal{R} \quad \implies \quad 
  A\cap B \in \mathcal{R}\htam{, }\ 
  A\cup B \in \mathcal{R}\htam{, }\ 
  A\backslash B \in \mathcal{R}\htam{; } \\
&\varnothing \in \mathcal{R} 
\htam{.} \\
\intertext{%
A ring~$\mathcal{R}$ is an \keyword{algebra} if%
}
&X \in \mathcal{R}
\htam{.} \\
\intertext{%
(An algebra of sets is a Boolean algebra.) 
A $\sigma$-algebra in~$X$
is an algebra~$\mathcal{R}$
of subsets of~$X$
with%
}
& A_1, A_2, \ldots \in \mathcal{R}\quad \implies\quad 
  \bigcup_n A_n \in \mathcal{R}
\htam{.}
\end{align*}
\end{enumerate}
%
%
%
% 1.2
%
\section{Definition}
An \keyword{ordered vector space} 
is a (real) vector~space~$E$
endowed with an ordering such that
\begin{equation*}
x,y\in E\htam{, }\ x\leq y
 \quad \implies \quad
\begin{cases}
x+a \leq y+a  
  & (a\in E)\htam{, } \\
\lambda x \leq \lambda y 
  & (\lambda \in [0,\infty)\,)\htam{. }
\end{cases}
\end{equation*}
Observe that then for all $x,y\in E$
\begin{equation*}
x\leq y \quad \implies \quad -x \geq -y\htam{.}
\end{equation*}
(Take $a:=-x-y$.)
%
%
% 1.3
%
\section{Definition}
A \keyword{Riesz space} ($=$ \keyword{vector lattice})
is an ordered vector space that is a lattice.

If~$E$ is an ordered vector space 
in which for every element~$x$
the set~$\{x,-x\}$ has a supremum,
then~$E$ is a Riesz space;
indeed,
for $x,y\in E$
we then have
\begin{align*}
\frac{x+y}{2} + \sup\left\{ \frac{x-y}{2}, -\frac{x-y}{2} \right\} 
  & = x\vee y\htam{,} \\ 
\frac{x+y}{2} - \sup\left\{ \frac{x-y}{2}, -\frac{x-y}{2} \right\}
  & = x\wedge y\htam{. }
\end{align*}
%
%
%
% 1.4
%
\section{Elementary properties}
Let~$E$ be a Riesz space. 
We collect a number of basic identities and inequalities.
For the sake of readability
we occasionally omit clauses such as ``for all~$x$ in~$E$''.
\begin{enumerate}
\item % 
\label{i:1.4-1}
Let~$X\subseteq E$, and let~$A\colon E\rightarrow E$ 
be an order isomorphism.
\snote{``order isomorphism'' has not been defined.}
If~$X$ has a supremum,
then so does $A(X)$,
and
\begin{align*}
\sup A(X) &= A(\sup X)
\htam{.} \\
\intertext{%
Similarly, %
}
\inf A(X) &= A(\inf X)
\htam{.}
\end{align*}
\item %
If~$x,y\in E$, then
\begin{equation*}
\frmd{ (x\vee y) = (-x)\wedge(-y) }
\htam{.}
\end{equation*}
\item %
It follows from \ref{i:1.4-1} that
\snote{So: $\lambda\cdot -$ is an order isomorphism.} 
\begin{alignat*}
\lambda \sup X &= \sup \lambda X \qquad &&(\lambda \in [0,\infty)\,) \\
\intertext{%
whenever~$X\subseteq E$ and~$\sup X$ exists.
In particular, }
\lambda (x \vee y) &=  (\lambda x) \vee (\lambda y) 
  \qquad &&(x,y\in E\htam{; }\lambda\in[0,\infty)\,)\htam{.}
\end{alignat*}
\item %
For~$X\subseteq E$ and $a\in E$,
set~$X+a:=\{x+a\colon x\in X\}$.
By \ref{i:1.4-1} we have
\begin{align*}
\sup(X+a) &= (\sup X) + a \\
\intertext{%
if $X\subseteq A$ and $\sup X$ exists.
Similarly, }
\inf (X+a) &= (\inf X)+a
\end{align*}
if~$X$ has an infimum.
\item %
In particular,
\snote{So: $-+a$ is an order isomorphism.}
\begin{equation*}
\label{i1.4-5}
\frmd{%
(x+a)\vee   (y+a)=(x\vee y)+a
\htam{, } \qquad 
(x+a)\wedge (y+a)=(x\wedge y)+a }
\htam{.}
\end{equation*}
\item %
\label{i1.4-6}
\begin{equation*}
x\vee y + x\wedge y = x+y
\end{equation*}
because~$(x\vee y)-x-y 
\smash{\stackrel{\htam{\ref{i1.4-5}}}{=}}
(x-x-y)\vee(y-x-y)=(-y)\vee(-x)=x\wedge y$.
\item %
Take~$a\in E$ and let $M_a\colon E\rightarrow E$ 
be either the map~$x\mapsto x\vee a$
or~$x\mapsto x\wedge a$. 
Let~$A\subseteq E$.
Then
\begin{align*}
M_a(\sup X) &= \sup M_a(X) \\
\intertext{%
if $X$ has a supremum and}
M_a(\inf X) &= \inf M_a(X)
\end{align*}
if~$X$ has an infimum.
Proof for $M_a(x)=x\vee a$:
\begin{itemize}
\item %
If~$X$ has a supremum:  
For~$z\in E$ we
\begin{align*}
z\ge (\sup X)\vee a  
  & \iff z\ge \sup X\ \htam{ and }\  z\ge a \\
  & \iff z\ge x\quad (x\in X)\ \htam{ and }\  z\ge a\\
  & \iff z\ge x\vee a\quad (x\in X) \\
  & \iff z\htam{ is upper bound for }M_a(X)\htam{.}
\end{align*}
\item %
If~$X$ has an infimum: 
The fact that $M_a$ is increasing 
implies that $M_a(\inf X)$ is a lower bound
for $M_a(X)$.
Now let~$z$
be any lower bound for~$M_a(X)$;
we prove~$z\leq M_a(\inf X)$.
For all~$x\in X$
we have
$z\leq x\vee a 
\smash{\stackrel{\htam{\ref{i1.4-6}}}{=}}
x+a - x\wedge a
\leq x+a-(\inf X)\wedge a$,
whence
$x\ge z-a+(\inf X)\wedge a$.
Then $\inf X\geq z-a+(\inf X)\wedge a$
so that 
$z\leq \inf X +a - (\inf X)\wedge a
\smash{\stackrel{\htam{\ref{i1.4-6}}}{=}}
(\inf X)\vee a = M_a(\inf X)$.
\end{itemize}
\end{enumerate}
\end{document}
