\documentclass[main.tex]{subfiles}
\begin{document}
% 1
\section{Preliminaries}
%
%                  1.1
%
\begin{psec}[Definitions]
\label{1.1}
\begin{enumerate}
% 1
\item 
A \keyword{lattice} is a (partially) ordered set in which
every nonempty finite subset has an infimum and a supremum. 
If $a,b$ are elements of a lattice~$\Lambda$ 
we write~$a\wedge b = \inf\{a,b\}$
and~$a\vee b = \sup\{a,b\}$.
% 2
\item 
A lattice $\Lambda$ is \keyword{distributive}
if for all $a,b,c$ in $\Lambda$
\begin{align*}
(a\vee b)\wedge c &= (a\wedge c)\vee (b\wedge c)\htam{, } &
(a\wedge b)\vee c &= (a\vee c) \wedge (b\vee c) 
\end{align*}
% 3
\item 
A \keyword{Boolean algebra}~$\Lambda$ is a distributive lattice
with a smallest element~$0$
and a largest element~$1$
such that for every~$x$ in~$\Lambda$
there is a \keyword{complement},
i.e., 
an element~$y$ of~$\Lambda$
with~$x\wedge y=0$, $x\vee y=1$.
In a Boolean algebra the complement is unique.
The complement of~$x$ is denoted:~$x'$.
% 4
\item 
Let~$X$ be a set.
A collection~$\mathcal{R}$ 
of subsets of~$X$
is called a \keyword{ring} if
\begin{align*}
&A,B\in\mathcal{R} \quad \implies \quad 
  A\cap B \in \mathcal{R}\htam{, }\ 
  A\cup B \in \mathcal{R}\htam{, }\ 
  A\backslash B \in \mathcal{R}\htam{; } \\
&\varnothing \in \mathcal{R} 
\htam{.} \\
\intertext{%
A ring~$\mathcal{R}$ is an \keyword{algebra} if%
}
&X \in \mathcal{R}
\htam{.} \\
\intertext{%
(An algebra of sets is a Boolean algebra.) 
A $\sigma$-algebra in~$X$
is an algebra~$\mathcal{R}$
of subsets of~$X$
with%
}
& A_1, A_2, \ldots \in \mathcal{R}\quad \implies\quad 
  \bigcup_n A_n \in \mathcal{R}
\htam{.}
\end{align*}
\end{enumerate}
\end{psec}
%
%                  1.2
%
\begin{psec}[Definition]
\label{1.2}
An \keyword{ordered vector space} 
is a (real) vector~space~$E$
endowed with an ordering such that
\begin{equation*}
x,y\in E\htam{, }\ x\leq y
 \quad \implies \quad
\begin{cases}
x+a \leq y+a  
  & (a\in E)\htam{, } \\
\lambda x \leq \lambda y 
  & (\lambda \in [0,\infty)\,)\htam{. }
\end{cases}
\end{equation*}
Observe that then for all $x,y\in E$
\begin{equation*}
x\leq y \quad \implies \quad -x \geq -y\htam{.}
\end{equation*}
(Take $a:=-x-y$.)
\end{psec}
%
%                  1.3
%
\begin{psec}[Definition]
\label{1.3}
A \keyword{Riesz space} ($=$ \keyword{vector lattice})
is an ordered vector space that is a lattice.

If~$E$ is an ordered vector space 
in which for every element~$x$
the set~$\{x,-x\}$ has a supremum,
then~$E$ is a Riesz space;
indeed,
for $x,y\in E$
we then have
\begin{align*}
\frac{x+y}{2} + \sup\left\{ \frac{x-y}{2}, -\frac{x-y}{2} \right\} 
  & = x\vee y\htam{,} \\ 
\frac{x+y}{2} - \sup\left\{ \frac{x-y}{2}, -\frac{x-y}{2} \right\}
  & = x\wedge y\htam{. }
\end{align*}
\end{psec}
%
%                  1.4
%
\begin{psec}{Elementary properties}
\label{1.4}
Let~$E$ be a Riesz space. 
We collect a number of basic identities and inequalities.
For the sake of readability
we occasionally omit clauses such as ``for all~$x$ in~$E$''.
\begin{enumerate}
\item % 
\label{1.4-1}
Let~$X\subseteq E$, and let~$A\colon E\rightarrow E$ 
be an order isomorphism.
\snote{``order isomorphism'' has not been defined.}
If~$X$ has a supremum,
then so does $A(X)$,
and
\begin{align*}
\sup A(X) &= A(\sup X)
\htam{.} \\
\intertext{%
Similarly, %
}
\inf A(X) &= A(\inf X)
\htam{.}
\end{align*}
\item %
\label{1.4-2}
If~$x,y\in E$, then
\begin{equation*}
\frmd{ (x\vee y) = (-x)\wedge(-y) }
\htam{.}
\end{equation*}
\item %
\label{1.4-3}
It follows from \ref{1.4-1} that
\snote{So: $\lambda\cdot -$ is an order isomorphism.} 
\begin{alignat*}
\lambda \sup X &= \sup \lambda X \qquad &&(\lambda \in [0,\infty)\,) \\
\intertext{%
whenever~$X\subseteq E$ and~$\sup X$ exists.
In particular, }
\lambda (x \vee y) &=  (\lambda x) \vee (\lambda y) 
  \qquad &&(x,y\in E\htam{; }\lambda\in[0,\infty)\,)\htam{.}
\end{alignat*}
\item %
\label{1.4-4}
For~$X\subseteq E$ and $a\in E$,
set~$X+a:=\{x+a\colon x\in X\}$.
By \ref{1.4-1} we have
\begin{align*}
\sup(X+a) &= (\sup X) + a \\
\intertext{%
if $X\subseteq A$ and $\sup X$ exists.
Similarly, }
\inf (X+a) &= (\inf X)+a
\end{align*}
if~$X$ has an infimum.
\item %
\label{1.4-5}
In particular,
\snote{So: $-+a$ is an order isomorphism.}
\begin{equation*}
\label{i1.4-5}
\frmd{%
(x+a)\vee   (y+a)=(x\vee y)+a
\htam{, } \qquad 
(x+a)\wedge (y+a)=(x\wedge y)+a }
\htam{.}
\end{equation*}
\item %
\label{1.4-6}
\begin{equation*}
x\vee y + x\wedge y = x+y
\end{equation*}
because~$(x\vee y)-x-y 
\isref{1.4-5}
(x-x-y)\vee(y-x-y)=(-y)\vee(-x)=x\wedge y$.
\item %
\label{1.4-7}
Take~$a\in E$ and let $M_a\colon E\rightarrow E$ 
be either the map~$x\mapsto x\vee a$
or~$x\mapsto x\wedge a$. 
Let~$A\subseteq E$.
Then
\begin{align*}
M_a(\sup X) &= \sup M_a(X) \\
\intertext{%
if $X$ has a supremum and}
M_a(\inf X) &= \inf M_a(X)
\end{align*}
if~$X$ has an infimum.
Proof for $M_a(x)=x\vee a$:
\begin{itemize}
\item %
If~$X$ has a supremum:  
For~$z\in E$ we
\begin{align*}
z\ge (\sup X)\vee a  
  & \iff z\ge \sup X\ \htam{ and }\  z\ge a \\
  & \iff z\ge x\quad (x\in X)\ \htam{ and }\  z\ge a\\
  & \iff z\ge x\vee a\quad (x\in X) \\
  & \iff z\htam{ is upper bound for }M_a(X)\htam{.}
\end{align*}
\item %
If~$X$ has an infimum: 
The fact that $M_a$ is increasing 
implies that $M_a(\inf X)$ is a lower bound
for $M_a(X)$.
Now let~$z$
be any lower bound for~$M_a(X)$;
we prove~$z\leq M_a(\inf X)$.
For all~$x\in X$
we have
$z\leq x\vee a 
\isref{1.4-6}
x+a - x\wedge a
\leq x+a-(\inf X)\wedge a$,
whence
$x\ge z-a+(\inf X)\wedge a$.
Then $\inf X\geq z-a+(\inf X)\wedge a$
so that 
$z\leq \inf X +a - (\inf X)\wedge a
\isref{1.4-6}
(\inf X)\vee a = M_a(\inf X)$.
\end{itemize}
\item %
\label{1.4-8}
Hence, $E$ is distributive:
\begin{equation*}
\frmd{(x\wedge y)\vee a = (x\vee a)\wedge(y\vee a)
\htam{, }\qquad
(x\vee y)\wedge a = (x\wedge a)\vee (y\wedge a)}
\htam{.}
\end{equation*}
\item %
\label{1.4-9}
Define
\begin{equation*}
\frmd{%
x^+ := x\vee 0\htam{, }\qquad
x^- := (-x)\vee 0\htam{, }\qquad
|x| := x\vee(-x)}
\htam{.}
\end{equation*}
Then 
\begin{equation*}
\frmd{ x = x^+ - x^- }
\end{equation*}
since $x^+ - x^-
\isref{1.4-2}(x\vee 0)+(x\wedge 0) 
\isref{1.4-6}x+0$.

Furthermore:
$x^+ + x^- 
= x^+ + (x^+ - x) 
= 2(x\vee 0) -x
= (2x)\vee 0 -x
\isref{1.4-5} (2x-x)\vee(0-x) = x\vee(-x) = |x|$:
\begin{equation*}
\frmd{|x| = x^+ + x^-}
\end{equation*}
which implies
\begin{equation*}
\frmd{|x|\ge 0}
\end{equation*}
and 
thereby~$|x|=|x|\vee 0
= x\vee(-x)\vee 0 \isref{1.4-7} (x\vee 0) \vee (-x\vee 0)$,
i.e., 
\begin{equation*}
\frmd{|x| = x^+ \vee x^-}
\htam{.}
\end{equation*}
Hence, $x^+ + x^- = x^+ \vee x^-$,
so that, 
by \ref{1.4-6},
\begin{equation*}
\frmd{x^+ \wedge x^- = 0} 
\htam{.}
\end{equation*}

From the definition of $|-|$
it is apparent that~$|-x|=|x|$.
Hence, 
\begin{equation*}
\frmd{|\lambda x|=|\lambda| |x|} 
  \qquad (\lambda\in\mathbb{R}\htam{, }\ x\in E)\htam{.}
\end{equation*}
Of course,
$x=x^+ - x^-\leq x^+ + x^- = |x|$
and $-x\leq|-x|=|x|$,
so $-|x|\leq x \leq |x|$.
Consequently,
\begin{equation*}
\frmd{|x|=0 \quad \implies \quad x=0}
\htam{.}
\end{equation*}
%
\item%
\label{1.4-10}
If $x,y\in E$,
then $x^+\geq x$, 
$y^+\geq y$,
so $x^+ + y^+ \geq x+y$.
Also,
$x^+ + y^+\geq 0$.
Thus
\begin{align*}
&\frmd{(x+y)^+ \leq x^+ + y^+}\\
\intertext{ and, 
similarly, 
$(x+y)^- \leq x^- + y^-$.
It follows that}
&\frmd{|x+y|\leq|x|+|y|}
\htam{.}
\end{align*}
From this,
it follows that
\begin{equation*}
||x|-|y|| \leq |x-y| 
\htam{.}
\end{equation*}

In \ref{1.3} 
\snote{Should point to 1.3---will be fixed.}
we have already observed:
\snote{put frame around this}
\begin{align*}
x \vee y 
  &= \frac{x+y}{2} + \left | \frac{x-y}{2} \right | \\
x \wedge y 
  &= \frac{x+y}{2} - \left | \frac{x-y}{2} \right | 
\htam{.}
\end{align*}
%
\item
\label{1.4-11}
Define the relation $\perp$ on~$E$ by
\begin{equation*}
\frmd{x \perp y \quad \iff \quad |x|\wedge|y|=0}
\htam{.}
\end{equation*}

For all $x$ we have seen that $x^+\wedge x^- = 0$,
so
\begin{equation*}
\frmd{x^+ \perp x^-}
\htam{.}
\end{equation*}
For any $x$ and $y$, 
$|x-x\wedge y|\wedge|y-x\wedge y|
=(x-x\wedge y) \wedge (y-x\wedge y)
\isref{1.4-5} x\wedge y - x\wedge y$:
\begin{equation*}
\frmd{x-x\wedge y \quad \perp \quad y- x\wedge y}
\htam{.}
\end{equation*}

It follows from the triangle inequality
 that~$x\perp y_1 + y_2$
as soon as~$x\perp y_1$ and $x\perp y_2$.
From this,
one infers easily that for every~$x$
the set~$\{ y: x\perp y\}$
is a linear subspace of~$E$.
%
\item \label{1.14-12}
The \keyword{positive cone} of~$E$ is the set
\begin{equation*}
E^+ := \{x\in E\colon x\ge 0\}
\htam{.}
\end{equation*}
\end{enumerate}
\end{psec}
%
%                  1.5
%
\begin{psec}[Examples]
\label{1.5}
\begin{enumerate}
\item
\label{1.5-1}
For any set~$X$
the functions~$X\ra \R$
form an ordered vector space~$\R^X$;
this~$\R^X$ is a Riesz space:
\begin{equation*}
(f\vee g)(x) = f(x)\vee g(x)\htam{, } \quad 
(f\wedge g)(x) = f(x)\wedge g(x)\qquad
(x\in X\htam{; }f,g\in\R^X)
\htam{.}
\end{equation*}
%
\item
\label{1.5-2}
For any set~$X$ the bounded functions on~$X$ form a 
Riesz~space~$\ell^\infty(X)$.
%
\item
\label{1.5-3}
(See below)
\snote{Add continuation.}
\end{enumerate}
\end{psec}
\end{document}
