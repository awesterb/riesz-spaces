\documentclass[main.tex]{subfiles}
\begin{document}
% 1
\section{Preliminaries}
%
% 1.1
\begin{psec}[Definitions]
\begin{enumerate}
% 1
\item 
A \keyword{lattice} is a (partially) ordered set in which
every nonempty finite subset has an infimum and a supremum. 
If $a,b$ are elements of a lattice~$\Lambda$ 
we write~$a\wedge b = \inf\{a,b\}$
and~$a\vee b = \sup\{a,b\}$.
% 2
\item 
A lattice $\Lambda$ is \keyword{distributive}
if for all $a,b,c$ in $\Lambda$
\begin{align*}
(a\vee b)\wedge c &= (a\wedge c)\vee (b\wedge c)\htam{, } &
(a\wedge b)\vee c &= (a\vee c) \wedge (b\vee c) 
\end{align*}
% 3
\item 
A \keyword{Boolean algebra}~$\Lambda$ is a distributive lattice
with a smallest element~$0$
and a largest element~$1$
such that for every~$x$ in~$\Lambda$
there is a \keyword{complement},
i.e., 
an element~$y$ of~$\Lambda$
with~$x\wedge y=0$, $x\vee y=1$.
In a Boolean algebra the complement is unique.
The complement of~$x$ is denoted:~$x'$.
% 4
\item 
Let~$X$ be a set.
A collection~$\mathcal{R}$ 
of subsets of~$X$
is called a \keyword{ring} if
\begin{align*}
&A,B\in\mathcal{R} \quad \implies \quad 
  A\cap B \in \mathcal{R}\htam{, }\ 
  A\cup B \in \mathcal{R}\htam{, }\ 
  A\backslash B \in \mathcal{R}\htam{; } \\
&\varnothing \in \mathcal{R} 
\htam{.} \\
\intertext{%
A ring~$\mathcal{R}$ is an \keyword{algebra} if%
}
&X \in \mathcal{R}
\htam{.} \\
\intertext{%
(An algebra of sets is a Boolean algebra.) 
A $\sigma$-algebra in~$X$
is an algebra~$\mathcal{R}$
of subsets of~$X$
with%
}
& A_1, A_2, \ldots \in \mathcal{R}\quad \implies\quad 
  \bigcup_n A_n \in \mathcal{R}
\htam{.}
\end{align*}
\end{enumerate}
\end{psec}
%
%
%
% 1.2
%
\begin{psec}[Definition]
An \keyword{ordered vector space} 
is a (real) vector~space~$E$
endowed with an ordering such that
\begin{equation*}
x,y\in E\htam{, }\ x\leq y
 \quad \implies \quad
\begin{cases}
x+a \leq y+a  
  & (a\in E)\htam{, } \\
\lambda x \leq \lambda y 
  & (\lambda \in [0,\infty)\,)\htam{. }
\end{cases}
\end{equation*}
Observe that then for all $x,y\in E$
\begin{equation*}
x\leq y \quad \implies \quad -x \geq -y\htam{.}
\end{equation*}
(Take $a:=-x-y$.)
\end{psec}
%
%
% 1.3
%
\begin{psec}[Definition]
A \keyword{Riesz space} ($=$ \keyword{vector lattice})
is an ordered vector space that is a lattice.

If~$E$ is an ordered vector space 
in which for every element~$x$
the set~$\{x,-x\}$ has a supremum,
then~$E$ is a Riesz space;
indeed,
for $x,y\in E$
we then have
\begin{align*}
\frac{x+y}{2} + \sup\left\{ \frac{x-y}{2}, -\frac{x-y}{2} \right\} 
  & = x\vee y\htam{,} \\ 
\frac{x+y}{2} - \sup\left\{ \frac{x-y}{2}, -\frac{x-y}{2} \right\}
  & = x\wedge y\htam{. }
\end{align*}
\end{psec}
\stub
\end{document}
