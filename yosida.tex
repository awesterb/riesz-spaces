\documentclass[main.tex]{subfiles}
\begin{document}
% 3
\section{The Yosida Theorem}
%
%                  3.1
%
\begin{psec}{3.1}{Definitions}
We call an element $x$ 
of a Riesz space $E$
\keyword{infinitesimal}
if the set
\begin{equation*}
\{ n x\colon n\in \mathbb Z \}
\end{equation*}
has an upper bound in $E$.
A Riesz space~$E$ is said to be
\keyword{Archimedean}
if~$0$ is the only infinitesimal element of~$E$.
\end{psec}
%
%                  3.2
%
\begin{psec}{3.2}{Exercise}
Let $E$ be a Riesz space.
Prove the equivalence of:
\begin{enumerate}
\item[$(\alpha)$] \label{3.2-alpha}
$E$ is Archimedean
%
\item[$(\beta)$] \label{3.2-beta}
If $a,b\in E^+$ and $na\leq b$ for all $n\in \N$, then $a=0$.
%
\item[$(\gamma)$] \label{3.2-gamma}
For all $a$ in $E^+$, 
$\inf \{ \frac{1}{n} a\colon n\in\N \} = 0$.
%
\item[$(\delta)$] \label{3.2-delta}
For all $a,b\in E^+$ with $a\neq 0$ 
the set $\{ \lambda\in \R\colon \lambda a \leq b \}$
has a largest element.
\end{enumerate}
\end{psec}
\clearpage
\end{document}
