\documentclass[main.tex]{subfiles}
\begin{document}
% 3
\section{The Yosida Theorem}
%
%                  3.1
%
\begin{psec}{3.1}{Definitions}
We call an element $x$ 
of a Riesz space $E$
\keyword{infinitesimal}
if the set
\begin{equation*}
\{ n x\colon n\in \mathbb Z \}
\end{equation*}
has an upper bound in $E$.
A Riesz space~$E$ is said to be
\keyword{Archimedean}
if~$0$ is the only infinitesimal element of~$E$.
\end{psec}
%
%                  3.2
%
\begin{psec}{3.2}{Exercise}
Let $E$ be a Riesz space.
Prove the equivalence of:
\begin{enumerate}
\item[$(\alpha)$] \label{3.2-alpha}
$E$ is Archimedean
%
\item[$(\beta)$] \label{3.2-beta}
If $a,b\in E^+$ and $na\leq b$ for all $n\in \N$, then $a=0$.
%
\item[$(\gamma)$] \label{3.2-gamma}
For all $a$ in $E^+$, 
$\inf \{ \frac{1}{n} a\colon n\in\N \} = 0$.
%
\item[$(\delta)$] \label{3.2-delta}
For all $a,b\in E^+$ with $a\neq 0$ 
the set $\{ \lambda\in \R\colon \lambda a \leq b \}$
has a largest element.
\end{enumerate}
\end{psec}
%
%                  3.3
%
\begin{psec}{3.3}{Examples}
\begin{enumerate}
\item \label{3.3-1}
$\R^X$ is Archimedean for every set $X$.
Riesz subspaces of Archimedean spaces are Archimedean,
so~$\Cont{X}$, $\ell^\infty$, $\cseq_0$, \ldots{} 
are all Archimedean.
%
\item \label{3.3-2}%
\newcommand{\leqlex}{\leq_{\mathrm{lex}}}
Defining a binary relation $\leqlex$ on $\R^2$ by
\begin{align*}
(x_1,x_2) \leqlex (y_1,y_2)   \quad \iff \quad
& \htam{ either }x_1<y_2 \\
& \htam{ or }x_1 = y_1\htam{, }\ x_2 \leq y_2\htam{, }
\end{align*}
we have a total ordering,
turning $\R^2$ into a Riesz space.
The element~$(0,1)$ is infinitesimal
(\,$(0,n)\leqlex(1,0)$ for all $n$ in $\N$).
The element $(1,0)$ is a strong unit.
$(0,1)$ is a weak but not a strong unit.

$\R^2$ with the ordering $\leqlex$ is called
the \keyword{lexicographic plane}.
%
\item \label{3.3-3}%
(Quotients of Archimedean spaces may fail to be Archimedean.)
Let~$E$ be the quotient Riesz space $\R^X / \ell^\infty$
and let~$Q$ be the quotient map $\R^\N\ra E$.
For $a=(a_1, a_2, \dotsc)\in \R^\N$,
let $a^* := (a_1,2a_2,3a_3,\dotsc)$.
If $a\in \smash{{(\R^\N)}^+}$, then for all~$n$ in~$\N$
\begin{equation*}
na \leq a^* + (na_1,\, na_2,\, \dotsc, na_n,\, 0,\, 0,\, \dotsc)
\end{equation*}
whence $n\, Qa\leq Qa^*$.

It follows that all elements of $E$ are infinitesimal.
\end{enumerate}
\end{psec}
%
%                  3.4
%
\begin{psec}{3.4}{Exercise}
Let $E$ be a Riesz space;
let $J$ be the set of its infinitesimal elements.
Prove the following:

$J$ is a Riesz ideal of $E$, 
and $E/J$ is Archimedean.
If~$T$ is a Riesz homomorphism of~$E$
into an Archimedean Riesz  space~$F$,
then there is a unique
$T_*\colon E/J\ra F$ with
\begin{equation*}
T_*(x+J) = Tx \qquad (x\in E)\htam{; }
\end{equation*}
this $T_*$ is a Riesz homomorphism.
\begin{equation*}
\xymatrix{
 E \ar[d]_{\htam{Quotient map}}\ar[r]^T & F \\
 E/J\ar@{.>}[ru]_{T_*} &
}
\end{equation*}
\end{psec}
%
%                  3.5
%
\begin{psec}{3.5}{Lemma}
\statement{
Let $E$ be a Riesz space, $E\neq \{0\}$
and $E$ are the only Riesz ideals of~$E$.
Then $E$ is Riesz isomorphic to $\R$.
}
\end{psec}
\begin{proof}
\begin{enumerate}[label=(\Roman*)]
\item \label{3.5-I} 
($E$ is totally ordered.) 
Let $a\in E$; 
we prove that $a\geq 0$ or $a\leq 0$.
The set $\{ x\colon x\perp a^+ \}$
is a Riesz ideal (see~\textbf{?!?}),
hence is either~$\{0\}$ or~$E$.
In the first case,
$a^-=0$, 
which implies $a=a^+\geq 0$.
In the second case
we have $a^+ \perp a^+$, so $a^+=0$ 
and $a=-a^-\leq 0$.
%
\item \label{3.5-II}
($E$ is Archimedean.)
Let $a,b\in E$, $n|a|\leq b$ for all $n\in \N$;
we prove $a=0$. 
That is easy. 
Suppose~$a\neq 0$.
The set $J:=\{ x\colon n|x|\leq b\htam{ for all }n\}$
is a Riesz ideal containing~$a$,
hence unequal to~$\{0\}$.
Then $J=E$,
so $|a|+b\in J$,
$|a|+b\leq b$,
$|a|\leq 0$,
and $a=0$.
%
\item \label{3.5-III} 
(A totally ordered Archimedean space is at most $1$-dimensional.)
Choose $a\in E^+$, $a\neq 0$.
The map $\lambda \mapsto \lambda a$ 
is a linear order isomorphism
of~\R{} onto a $1$-dimensional linear subspace of~$E$.
We are done if this subspace is~$E$ itself.
Thus, take~$b$ in~$E^+\backslash\{0\}$;
it suffices to show that~$b=\lambda a$
for some $\lambda \in \R$.

According to Exercise~\ref{3.2} there exist
$\sigma_0,\tau_0\in[0,\infty)$ such that
\begin{alignat*}{2}
&\htam{ if }\sigma\in\R\htam{, then }
  & \sigma a\leq b 
  &\iff \sigma\leq\sigma_0\htam{, } \\
&\htam{ if }\tau\in\R\htam{, then }
  & \tau b \leq a 
  &\iff \tau \leq \tau_0\htam{.}
\end{alignat*}
If $\sigma\in\R$ with $\sigma>\sigma_0$
we have $\sigma a\nleq b$,
so $b\leq \sigma a$,
$\sigma^{-1} b\leq a$
(note that $\sigma > \sigma_0 \geq 0$),
and therefore $\sigma^{-1}\leq \tau_0$
and $1\leq \sigma\tau_0$.
Hence, $1\leq \inf_{\sigma > \sigma_0}\sigma \tau_0 = \sigma_0 \tau_0$.
This yields $\sigma_0 a \leq b \leq \sigma_0 \tau_0 b \leq \sigma_0 a$,
so that $b = \sigma_0 a$. \xqed
\end{enumerate}
\end{proof}
%
%                  3.6
%
\begin{psec}{3.6}%
Let $E$ be a Riesz space.
A Riesz ideal $D$ of $E$ is called \keyword{proper}
if $D\neq E$.
Under inclusion,
the proper Riesz ideals of~$E$
form an ordered set.
The maximal elements of this set are,
by definition,
the \keyword{maximal Riesz ideals} of~$E$.

Let $M$ be such a maximal ideal and 
let $Q\colon E\ra E/M$ be the quotient map.
$E/M$ is a Riesz space.
If $D$ is a Riesz ideal in $E/M$,
then one easily sees 
that $Q^{-1}(D)$ is a Riesz ideal in~$E$
that contains~$M$.
The maximality of~$M$ 
implies that~$Q^{-1}(D)$ is either $M$ or $E$,
so that~$D$ is either~$\{ 0\}$ or~$E/M$.

Thus, $E/M$ is Riesz isomorphic to~$\R$,
and~$M$ is the kernel of a Riesz homomorphism
$E\ra \R$.
\end{psec}
%
%                  3.7
%
\begin{psec}{3.7}%
Let $E$ be a unitary Riesz space.
\begin{enumerate}
\item \label{3.7-1}
Choose a unit $e$ in $E$. 
The only Riesz ideal that contains~$e$ is~$E$ itself.
Hence, if~$D$ is any Riesz ideal, then
\begin{equation*}
D\htam{ is proper }\quad \iff \quad e\notin D\htam{.}
\end{equation*}
%
\item \label{3.7-2}
Let $Z$ be a subset of $E$.
Zorn's Lemma implies:
If $D_0$ is a Riesz ideal
and~$D_0\subseteq Z$,
then among the Riesz ideals~$D$ 
with $D_0\subseteq D\subseteq Z$
there is a maximal one.
%
\item \label{3.7-3}
Consequently:
Every proper Riesz ideal of~$E$
is contained in a maximal Riesz ideal.
%
\item \label{3.7-4}
Therefore,
with the last part of~\ref{3.6}:
If $D$ is any proper Riesz ideal of~$E$,
there exists a nonzero Riesz homomorphism
$\varphi\colon E\ra\R$ whose kernel contains~$D$.
\end{enumerate}
\end{psec}
%
%                  3.8
%
\begin{psec}{3.8}{Example}
Consider the (unitary) space $\ell^\infty$
of~\ref{2.8}\ref{2.8-2}.
The evaluations $x\mapsto x_n$ 
are Riesz homomorphisms $\ell^\infty\ra \R$,
as are their positive scalar multiples
(and the zero function, of course).

It follows from~\ref{3.7}\ref{3.7-4}
that there are other Riesz homomorphisms
$\ell^\infty \ra \R$.
Indeed,
$\cseq_0$ (see \ref{1.5}\ref{1.5-5}) 
is a proper Riesz ideal in~$\ell^\infty$,
so there must be nonzero Riesz homomorphisms
$\ell^\infty\ra \R$ that vanish on~$\cseq_0$.
\end{psec}
%
%                  3.9
%
\begin{psec}{3.9}{Lemma}
\statement{
Let $E$ be an Archimedean unitary Riesz space
and let $a\in E$, $a>0$.
Then there is a Riesz homomorphism
$\varphi\colon E \ra \R$ with $\varphi(a)>0$.
}
\end{psec}
\begin{proof}
Choose a unit $e$ in $E$.
As $E$ is Archimedean
there is an $n$ in $\N$
with $na\nleq e$,
i.e. $na-e\nleq 0$.
But $na-e\leq(na-e)^+$,
so $(na-e)^+\neq 0$.

By exercise~\ref{2.14},
$J:=\{x\colon x\perp(na-e)^+\}$
is a Riesz ideal of~$E$;
it is proper because it does not contain $(na-e)^+$.
By~\ref{3.7}\ref{3.7-4}
there exists a nonzero Riesz homomorphism 
$f\colon E\ra \R$
whose kernel contains $J$.

Now $(na-e)^-\in J$,
so $\smash{0=f\bigl((na-e)^-\bigr)=\bigl(nf(a)-f(e)\bigr)^-}$,
which implies that
$nf(a)-f(e)=(nf(a)-f(e))^+\geq 0$.
Thus,
$nf(a)\geq f(e)$.
The kernel of~$f$
is a Riesz ideal,
proper because~$f$ is nonzero.
Hence, $f(e)>0$ and $f(a)>0$.
\end{proof}
%
%                  3.10
%
\begin{psec}{3.10}%
For the main theorem of this chapter
we need some results from functional analysis.
\begin{enumerate}
\item \label{3.10-1}
Let $X$ be a topological space.

For $f\in \BCont{X}$ (\ref{1.5}\ref{1.5-3}) we set
\begin{equation*}
\| f \|_\infty := \sup \bigl\{ | f(x) | \colon x \in X\bigr\}\htam{.}
\end{equation*}
The function $\|\cdot \|_\infty$ is a norm on $\BCont{X}$.
The convergence it induces is uniform convergence:
$\|f_n-f\|\ra 0$ if and only if for every~$\varepsilon >0$
there is an $N\in \N$ with
\begin{equation*}
\htam{ if }n\geq N
\htam{, then }|f_n(x)-f(x)|\leq \varepsilon
\htam{ for all }x\htam{.}
\end{equation*}
%
\item \label{3.10-2}
\textbf{Stone--Weierstrass Theorem}\  \statement{%
Let $X$ be a compact Hausdorff space,
$D$ a Riesz subspace of $\Cont{X}$($=\BCont{X}$)
with the properties
\begin{align*}
& \mathbb{1}\in D\htam{, } \\
& \htam{if }x,y\in X\htam{ and }x\neq y
\htam{, there is an }f\htam{ in }D
\htam{ with }f(x)\neq f(y)\htam{.}
\end{align*}
Then $D$ is dense in $\Cont{X}$.
}
\end{enumerate}
\end{psec}
%
%                  3.11
%
\begin{psec}{3.11}{Definition}
Let $E$ be an Archimedean Riesz space
with a (strong) unit~$e$.
For every $x\in E$
there exist positive numbers~$\lambda$
with $|x|\leq \lambda e$;
define
\begin{equation*}
\| x \|_e := \inf\bigl\{ \lambda\in[0,\infty): |x|\leq \lambda e\bigr\}\htam{.}
\end{equation*}
Thus, we obtain a function $\|\cdot\|\colon E \ra [0,\infty)$.
\end{psec}
%
%                  3.12
%
\begin{psec}{3.12}{Exercise}
(Situation of~\ref{3.11}) Prove the following.
\begin{enumerate}
\item \label{3.12-1}
If $x\in E$, then 
\begin{equation*}
|x|\leq \|x\|_e e\htam{.}
\end{equation*}
(See exercise~\ref{3.2}.)
%
\item \label{3.12-2}
$\|\cdot\|_e$ is a norm.
%
\item \label{3.12-3}
If $a,x_1,x_2,\dotsc\in E$,
the sequence $(x_n)_n$
converges to~$a$ in the sense
of the norm~$\|\cdot\|_e$
if and only if 
for every $\varepsilon>0$ there exists 
an~$N$ with
\begin{equation*}
|x_n-a|\leq \varepsilon e \quad\htam{ if }\quad n\geq N\htam{.} 
\end{equation*}
%
\item \label{3.12-4}
If $x,y\in E$ and $0\leq x\leq y$,
then $\|x\|_e \leq \|y\|_e$.
%
\item \label{3.12-5}
$\| \, | x|\, \|_e=\| x \|_e$ for all $x\in E$.
%
\item \label{3.12-6}
If $X$ is a set and $E=\ell^\infty(X)$
(\ref{1.5}\ref{1.5-2}),
or if~$X$ is a topological space 
and $E=\BCont{X}$,
then $\|\cdot\|_\mathbb{1}$ 
is the supremum norm $\|\cdot\|_\infty$.
\end{enumerate}
\end{psec}
%
%                  3.13
%
\begin{psec}{3.13}{Exercise}
(Situation of~\ref{3.11} and~\ref{3.12})
Show that for all $x,y\in E$
\begin{align*}
\|\,|x|-|y|\,\|_e & \leq \|x-y\|_e\htam{,}\\
\| x^+ - y^+ \|_e & \leq \|x-y\|_e\htam{.}
\end{align*}
Thus, the lattice operations $x\mapsto \| x\|$
and $x\mapsto x^+$
are (uniformly) continuous.
\end{psec}
%
%                  3.14
%
\begin{psec}{3.14}{Yosida Representation Theorem}
\statement{
Let $E$ be an Archimedean Riesz space
with a unit, $e$.
Let $\Phi$ be the set of all Riesz homomorphisms
$\varphi\colon E \ra \R$ with $\varphi(e)=1$,
topologized as a subset of $\R^E$ (product topology).
For $x\in E$ define a function~$\hat{x}$ on~$\Phi$ by
\begin{equation*}
\hat{x}(\varphi):=\varphi(x)\qquad (\varphi\in \Phi)\htam{,}
\end{equation*}
(so that $\hat{e}=\mathbb{1}$),
and put $\hat{E}:=\{\hat{x}\colon x\in E\}.$
Then:}

\statement{
$\Phi$ is a compact Hausdorff space,
$\hat{E}$ is a Riesz subspace of $\Cont{\Phi}$,
dense in the sense of~$\|\cdot\|_\infty$.
The map $x\mapsto \hat{x}$
is a Riesz isomorphism $E\ra \hat{E}$,
and $\|x\|_e = \|\hat{x}\|_\infty$
for all~$x$ in~$E$.
}
\end{psec}
\begin{proof}
\begin{enumerate}[label=(\Roman*)]
\item \label{3.14-I}
If $x\in E$, then $|x|\leq \| x\|_e\, e$\ 
 (\ref{3.12}\ref{3.12-1}),
so that for all~$\varphi$ in~$\Phi$
\begin{equation*}
|\varphi(x)|=\varphi(|x|)\leq \|x\|_e\, \varphi(e)=\|x\|_e\htam{.}
\end{equation*}
It follows that (also topologically)
\begin{equation*}
\Phi 
\ \subseteq  \ 
\prod_{x\in E} \bigl[-\| x \|_e , \|x \|_e \bigr] 
\ \subseteq \ \R^E\htam{.}
\end{equation*}
Hence,
by Tychonoff's Theorem,
$\Phi$ will be compact if it is closed in~$\R^E$.
Thus, let $(\varphi_i)_{i\in I}$
be a net in~$\Phi$,
converging to some $\varphi\in\R^E$;
we need $\varphi\in\Phi$.
From the definition of the product topology we obtain
\begin{equation}
\label{eq3.14-I} \tag{$*$}
\varphi_i(x) \longrightarrow \varphi(x) \qquad (x\in E)\htam{.}
\end{equation}
Hence, 
if $x,y\in E$,
then $\varphi_i(x)\rightarrow \varphi(x)$,
$\varphi_i(y)\rightarrow \varphi(y)$,
$\varphi_i(x+y)\rightarrow \varphi(x+y)$,
whereas $\varphi_i(x)+\varphi_i(y)=\varphi_i(x+y)$ for all $i$;
it follows that $\varphi(x)+\varphi(y) = \varphi(x+y)$.
Similarly, $\varphi(\lambda x)=\lambda \varphi(x)
\quad (x\in E,\ \lambda\in \R)$,
$\varphi(|x|)=|\varphi(x)|\quad(x\in E)$,
and $\varphi(e)=1$.
Thus~$\varphi\in\Phi$.
%
\item \label{3.14-II}
Formula~\eqref{eq3.14-I} 
read as ``$\hat{x}(\varphi_i)\rightarrow\hat{x}(\varphi)\quad (x\in E)$''
shows that $\hat{E}\subseteq\Cont{\Phi}$.
From the definitions it is apprarent that $x\mapsto\hat{x}$
is a Riesz homomorphism $E\ra\Cont{\Phi}$.
Then its range, $\hat{E}$, 
is a Riesz subspace of~$\Cont{\Phi}$.
%
\item \label{3.14-III}
If $\varphi,\psi\in\Phi$ and $\varphi\neq\psi$,
then trivially there is an~$x$ in~$E$ with $\varphi(x)\neq\psi(x)$,
i.e. $\hat{x}(\varphi) \neq\hat{x}(\psi)$.
Hence, $\hat{E}$ is dense in~$\Cont{\Phi}$ by the 
Stone--Weierstrass Theorem.
%
\item \label{3.14-IV}
It remains to prove that the map $x\mapsto \hat{x}$
is injective.
Let $a\in E$, $a\neq 0$;
it suffices to show that $\hat{a}\neq 0$.
By lemma~\ref{3.8},
there is a Riesz homomorphism $\omega\colon E\ra \R$
with $\omega(|a|)>0$.
As $|a|\leq ne$ for some $n\in \N$
we have $\omega(e)>0$.
Setting $\varphi:=\omega(e)^{-1} \omega$
we find $\varphi\in \Phi$ 
and $|\hat{a}(\varphi)| = |\varphi(a)| = \varphi(|a|)>0$.
%
\item \label{3.14-V}
Finally,
$x\mapsto \hat{x}$
being an isomorphism $E\ra \hat{E}$:
\begin{equation*}
\|x\|_e = \|\hat{x}\|_{\hat{e}} = \|\hat{x}\|_\mathbb{1} 
= \|\hat{x}\|_\infty\qquad (x\in E)\mathrlap{\htam{. \qquad \xqed}}
\end{equation*}
\end{enumerate}
\end{proof}
%
%                  3.15
%
\begin{psec}{3.15}{Exercise}
Let $X$ be a compact Hausdorff space.
An application of Yosida's Theorem with $E=\Cont{X}$,
$e=\mathbb{1}$ yields
a compact Hausdorff space~$\Phi$
and a map $f\mapsto\hat{f}$ of~$\Cont{X}$ into~$\Cont{\Phi}$.
Show that~$\Phi$ is homeomorphic to~$X$
and that, in fact, there exists a homeomorphism
$\tau\colon \Phi\ra X$ with $\hat{f}=f\circ \tau\quad(f\in \Cont{X}\, )$.
Infer that $\hat{E}$ is all of $\Cont{\Phi}$.
\end{psec}
\clearpage
\end{document}
