\documentclass[main.tex]{subfiles}
\begin{document}
% 3
\section{The Yosida Theorem}
%
%                  3.1
%
\begin{psec}{3.1}{Definitions}
We call an element $x$ 
of a Riesz space $E$
\keyword{infinitesimal}
if the set
\begin{equation*}
\{ n x\colon n\in \mathbb Z \}
\end{equation*}
has an upper bound in $E$.
A Riesz space~$E$ is said to be
\keyword{Archimedean}
if~$0$ is the only infinitesimal element of~$E$.
\end{psec}
%
%                  3.2
%
\begin{psec}{3.2}{Exercise}
Let $E$ be a Riesz space.
Prove the equivalence of:
\begin{enumerate}
\item[$(\alpha)$] \label{3.2-alpha}
$E$ is Archimedean
%
\item[$(\beta)$] \label{3.2-beta}
If $a,b\in E^+$ and $na\leq b$ for all $n\in \N$, then $a=0$.
%
\item[$(\gamma)$] \label{3.2-gamma}
For all $a$ in $E^+$, 
$\inf \{ \frac{1}{n} a\colon n\in\N \} = 0$.
%
\item[$(\delta)$] \label{3.2-delta}
For all $a,b\in E^+$ with $a\neq 0$ 
the set $\{ \lambda\in \R\colon \lambda a \leq b \}$
has a largest element.
\end{enumerate}
\end{psec}
%
%                  3.3
%
\begin{psec}{3.3}{Examples}
\begin{enumerate}
\item \label{3.3-1}
$\R^X$ is Archimedean for every set $X$.
Riesz subspaces of Archimedean spaces are Archimedean,
so~$\Cont{X}$, $\ell^\infty$, $\cseq_0$, \ldots{} 
are all Archimedean.
%
\item \label{3.3-2}%
\newcommand{\leqlex}{\leq_{\mathrm{lex}}}
Defining a binary relation $\leqlex$ on $\R^2$ by
\begin{align*}
(x_1,x_2) \leqlex (y_1,y_2)   \quad \iff \quad
& \htam{ either }x_1<y_2 \\
& \htam{ or }x_1 = y_1\htam{, }\ x_2 \leq y_2\htam{, }
\end{align*}
we have a total ordering,
turning $\R^2$ into a Riesz space.
The element~$(0,1)$ is infinitesimal
(\,$(0,n)\leqlex(1,0)$ for all $n$ in $\N$).
The element $(1,0)$ is a strong unit.
$(0,1)$ is a weak but not a strong unit.

$\R^2$ with the ordering $\leqlex$ is called
the \keyword{lexicographic plane}.
%
\item \label{3.3-3}%
(Quotients of Archimedean spaces may fail to be Archimedean.)
Let~$E$ be the quotient Riesz space $\R^X / \ell^\infty$
and let~$Q$ be the quotient map $\R^\N\ra E$.
For $a=(a_1, a_2, \dotsc)\in \R^\N$,
let $a^* := (a_1,2a_2,3a_3,\dotsc)$.
If $a\in \smash{{(\R^\N)}^+}$, then for all~$n$ in~$\N$
\begin{equation*}
na \leq a^* + (na_1,\, na_2,\, \dotsc, na_n,\, 0,\, 0,\, \dotsc)
\end{equation*}
whence $n\, Qa\leq Qa^*$.

It follows that all elements of $E$ are infinitesimal.
\end{enumerate}
\end{psec}
\clearpage
\end{document}
