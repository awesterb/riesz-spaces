\documentclass[main.tex]{subfiles}
\begin{document}
% 3
\section{The Yosida Theorem}
%
%                  3.1
%
\begin{psec}{3.1}{Definitions}
We call an element $x$ 
of a Riesz space $E$
\keyword{infinitesimal}
if the set
\begin{equation*}
\{ n x\colon n\in \mathbb Z \}
\end{equation*}
has an upper bound in $E$.
A Riesz space~$E$ is said to be
\keyword{Archimedean}
if~$0$ is the only infinitesimal element of~$E$.
\end{psec}
%
%                  3.2
%
\begin{psec}{3.2}{Exercise}
Let $E$ be a Riesz space.
Prove the equivalence of:
\begin{enumerate}
\item[$(\alpha)$] \label{3.2-alpha}
$E$ is Archimedean
%
\item[$(\beta)$] \label{3.2-beta}
If $a,b\in E^+$ and $na\leq b$ for all $n\in \N$, then $a=0$.
%
\item[$(\gamma)$] \label{3.2-gamma}
For all $a$ in $E^+$, 
$\inf \{ \frac{1}{n} a\colon n\in\N \} = 0$.
%
\item[$(\delta)$] \label{3.2-delta}
For all $a,b\in E^+$ with $a\neq 0$ 
the set $\{ \lambda\in \R\colon \lambda a \leq b \}$
has a largest element.
\end{enumerate}
\end{psec}
%
%                  3.3
%
\begin{psec}{3.3}{Examples}
\begin{enumerate}
\item \label{3.3-1}
$\R^X$ is Archimedean for every set $X$.
Riesz subspaces of Archimedean spaces are Archimedean,
so~$\Cont{X}$, $\ell^\infty$, $\cseq_0$, \ldots{} 
are all Archimedean.
%
\item \label{3.3-2}%
\newcommand{\leqlex}{\leq_{\mathrm{lex}}}
Defining a binary relation $\leqlex$ on $\R^2$ by
\begin{align*}
(x_1,x_2) \leqlex (y_1,y_2)   \quad \iff \quad
& \htam{ either }x_1<y_2 \\
& \htam{ or }x_1 = y_1\htam{, }\ x_2 \leq y_2\htam{, }
\end{align*}
we have a total ordering,
turning $\R^2$ into a Riesz space.
The element~$(0,1)$ is infinitesimal
(\,$(0,n)\leqlex(1,0)$ for all $n$ in $\N$).
The element $(1,0)$ is a strong unit.
$(0,1)$ is a weak but not a strong unit.

$\R^2$ with the ordering $\leqlex$ is called
the \keyword{lexicographic plane}.
%
\item \label{3.3-3}%
(Quotients of Archimedean spaces may fail to be Archimedean.)
Let~$E$ be the quotient Riesz space $\R^X / \ell^\infty$
and let~$Q$ be the quotient map $\R^\N\ra E$.
For $a=(a_1, a_2, \dotsc)\in \R^\N$,
let $a^* := (a_1,2a_2,3a_3,\dotsc)$.
If $a\in \smash{{(\R^\N)}^+}$, then for all~$n$ in~$\N$
\begin{equation*}
na \leq a^* + (na_1,\, na_2,\, \dotsc, na_n,\, 0,\, 0,\, \dotsc)
\end{equation*}
whence $n\, Qa\leq Qa^*$.

It follows that all elements of $E$ are infinitesimal.
\end{enumerate}
\end{psec}
%
%                  3.4
%
\begin{psec}{3.4}{Exercise}
Let $E$ be a Riesz space;
let $J$ be the set of its infinitesimal elements.
Prove the following:

$J$ is a Riesz ideal of $E$, 
and $E/J$ is Archimedean.
If~$T$ is a Riesz homomorphism of~$E$
into an Archimedean Riesz  space~$F$,
then there is a unique
$T_*\colon E/J\ra F$ with
\begin{equation*}
T_*(x+J) = Tx \qquad (x\in E)\htam{; }
\end{equation*}
this $T_*$ is a Riesz homomorphism.
\begin{equation*}
\xymatrix{
 E \ar[d]_{\htam{Quotient map}}\ar[r]^T & F \\
 E/J\ar@{.>}[ru]_{T_*} &
}
\end{equation*}
\end{psec}
%
%                  3.5
%
\begin{psec}{3.5}{Lemma}
\statement{
Let $E$ be a Riesz space, $E\neq \{0\}$
and $E$ are the only Riesz ideals of~$E$.
Then $E$ is Riesz isomorphic to $\R$.
}
\end{psec}
\begin{proof}
\begin{enumerate}[label=(\Roman*)]
\item \label{3.5-I} 
($E$ is totally ordered.) 
Let $a\in E$; 
we prove that $a\geq 0$ or $a\leq 0$.
The set $\{ x\colon x\perp a^+ \}$
is a Riesz ideal (see~\textbf{?!?}),
hence is either~$\{0\}$ or~$E$.
In the first case,
$a^-=0$, 
which implies $a=a^+\geq 0$.
In the second case
we have $a^+ \perp a^+$, so $a^+=0$ 
and $a=-a^-\leq 0$.
%
\item \label{3.5-II}
($E$ is Archimedean.)
Let $a,b\in E$, $n|a|\leq b$ for all $n\in \N$;
we prove $a=0$. 
That is easy. 
Suppose~$a\neq 0$.
The set $J:=\{ x\colon n|x|\leq b\htam{ for all }n\}$
is a Riesz ideal containing~$a$,
hence unequal to~$\{0\}$.
Then $J=E$,
so $|a|+b\in J$,
$|a|+b\leq b$,
$|a|\leq 0$,
and $a=0$.
%
\item \label{3.5-III} 
(A totally ordered Archimedean space is at most $1$-dimensional.)
Choose $a\in E^+$, $a\neq 0$.
The map $\lambda \mapsto \lambda a$ 
is a linear order isomorphism
of~\R{} onto a $1$-dimensional linear subspace of~$E$.
We are done if this subspace is~$E$ itself.
Thus, take~$b$ in~$E^+\backslash\{0\}$;
it suffices to show that~$b=\lambda a$
for some $\lambda \in \R$.

According to Exercise~\ref{3.2} there exist
$\sigma_0,\tau_0\in[0,\infty)$ such that
\begin{alignat*}{2}
&\htam{ if }\sigma\in\R\htam{, then }
  & \sigma a\leq b 
  &\iff \sigma\leq\sigma_0\htam{, } \\
&\htam{ if }\tau\in\R\htam{, then }
  & \tau b \leq a 
  &\iff \tau \leq \tau_0\htam{.}
\end{alignat*}
If $\sigma\in\R$ with $\sigma>\sigma_0$
we have $\sigma a\nleq b$,
so $b\leq \sigma a$,
$\sigma^{-1} b\leq a$
(note that $\sigma > \sigma_0 \geq 0$),
and therefore $\sigma^{-1}\leq \tau_0$
and $1\leq \sigma\tau_0$.
Hence, $1\leq \sigma_0 \tau_0$.
This yields $\sigma_0 a \leq b \leq \sigma_0 \tau_0 b = \sigma_0 a$,
so that $b = \sigma_0 a$. \xqed
\end{enumerate}
\end{proof}
\clearpage
\end{document}
