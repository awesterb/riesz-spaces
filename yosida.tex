\documentclass[main.tex]{subfiles}
\begin{document}
% 3
\section{The Yosida Theorem}
%
%                  3.1
%
\begin{psec}{3.1}{Definitions}
We call an element $x$ 
of a Riesz space $E$
\keyword{infinitesimal}
if the set
\begin{equation*}
\{ n x\colon n\in \mathbb Z \}
\end{equation*}
has an upper bound in $E$.
A Riesz space~$E$ is said to be
\keyword{Archimedean}
if~$0$ is the only infinitesimal element of~$E$.
\end{psec}
%
%                  3.2
%
\begin{psec}{3.2}{Exercise}
Let $E$ be a Riesz space.
Prove the equivalence of:
\begin{enumerate}
\item[$(\alpha)$] \label{3.2-alpha}
$E$ is Archimedean.
%
\item[$(\beta)$] \label{3.2-beta}
If $a,b\in E^+$ and $na\leq b$ for all $n\in \N$, then $a=0$.
%
\item[$(\gamma)$] \label{3.2-gamma}
For all $a$ in $E^+$, 
$\inf \{ \frac{1}{n} a\colon n\in\N \} = 0$.
%
\item[$(\delta)$] \label{3.2-delta}
For all $a,b\in E^+$ with $a\neq 0$ 
the set $\{ \lambda\in \R\colon \lambda a \leq b \}$
has a largest element.
\end{enumerate}
\end{psec}
%
%                  3.3
%
\begin{psec}{3.3}{Examples}
\begin{enumerate}
\item \label{3.3-1}
$\R^X$ is Archimedean for every set $X$.
Riesz subspaces of Archimedean spaces are Archimedean,
so~$\Cont{X}$, $\ell^\infty$, $\cseq_0$, \ldots{} 
are all Archimedean.
%
\item \label{3.3-2}%
\newcommand{\leqlex}{\leq_{\mathrm{lex}}}
Defining a binary relation $\leqlex$ on $\R^2$ by
\begin{align*}
(x_1,x_2) \leqlex (y_1,y_2)   \quad \iff \quad
& \htam{ either }x_1<y_2 \\
& \htam{ or }x_1 = y_1\htam{, }\ x_2 \leq y_2\htam{, }
\end{align*}
we have a total ordering,
turning $\R^2$ into a Riesz space.
The element~$(0,1)$ is infinitesimal
(\,$(0,n)\leqlex(1,0)$ for all $n$ in $\N$).
The element $(1,0)$ is a strong unit.
$(0,1)$ is a weak but not a strong unit.

$\R^2$ with the ordering $\leqlex$ is called
the \keyword{lexicographic plane}.
%
\item \label{3.3-3}%
(Quotients of Archimedean spaces may fail to be Archimedean.)
Let~$E$ be the quotient Riesz space $\R^\N / \ell^\infty$
and let~$Q$ be the quotient map $\R^\N\ra E$.
For $a=(a_1, a_2, \dotsc)\in \R^\N$,
let $a^* := (a_1,2a_2,3a_3,\dotsc)$.
If $a\in \smash{{(\R^\N)}^+}$, then for all~$n$ in~$\N$
\begin{equation*}
na \leq a^* + (na_1,\, na_2,\, \dotsc, na_n,\, 0,\, 0,\, \dotsc)
\end{equation*}
whence $n\, Qa\leq Qa^*$.

It follows that all elements of $E$ are infinitesimal.
\end{enumerate}
\end{psec}
%
%                  3.4
%
\begin{psec}{3.4}{Exercise}
Let $E$ be a Riesz space;
let $J$ be the set of its infinitesimal elements.
Prove the following:

$J$ is a Riesz ideal of $E$, 
and $E/J$ is Archimedean.
If~$T$ is a Riesz homomorphism of~$E$
into an Archimedean Riesz  space~$F$,
then there is a unique
$T_*\colon E/J\ra F$ with
\begin{equation*}
T_*(x+J) = Tx \qquad (x\in E)\htam{; }
\end{equation*}
this $T_*$ is a Riesz homomorphism.
\begin{equation*}
\xymatrix{
 E \ar[d]_{\htam{Quotient map}}\ar[r]^T & F \\
 E/J\ar@{.>}[ru]_{T_*} &
}
\end{equation*}
\end{psec}
%
%                  3.5
%
\begin{psec}{3.5}{Lemma}
\statement{
Let $E$ be a Riesz space, $E\neq \{0\}$.
Suppose $\{0\}$ and $E$ are the only Riesz ideals of~$E$.
Then $E$ is Riesz isomorphic to $\R$.
}
\end{psec}
\begin{proof}
\begin{enumerate}[label=(\Roman*)]
\item \label{3.5-I} 
($E$ is totally ordered.) 
Let $a\in E$; 
we prove that $a\geq 0$ or $a\leq 0$.
The set $\{ x\colon x\perp a^+ \}$
is a Riesz ideal (see~\ref{2.14}),
hence is either~$\{0\}$ or~$E$.
In the first case,
$a^-=0$, 
which implies $a=a^+\geq 0$.
In the second case
we have $a^+ \perp a^+$, so $a^+=0$ 
and $a=-a^-\leq 0$.
%
\item \label{3.5-II}
($E$ is Archimedean.)
Let $a,b\in E$, $n|a|\leq b$ for all $n\in \N$;
we prove $a=0$. 
That is easy. 
Suppose~$a\neq 0$.
The set $J:=\{ x\colon n|x|\leq b\htam{ for all }n\}$
is a Riesz ideal containing~$a$,
hence unequal to~$\{0\}$.
Then $J=E$,
so $|a|+b\in J$,
$|a|+b\leq b$,
$|a|\leq 0$,
and $a=0$.
%
\item \label{3.5-III} 
(A totally ordered Archimedean space is at most $1$-dimensional.)
Choose $a\in E^+$, $a\neq 0$.
The map $\lambda \mapsto \lambda a$ 
is a linear order isomorphism
of~\R{} onto a $1$-dimensional linear subspace of~$E$.
We are done if this subspace is~$E$ itself.
Thus, take~$b$ in~$E^+\backslash\{0\}$;
it suffices to show that~$b=\lambda a$
for some $\lambda \in \R$.

According to Exercise~\ref{3.2} there exist
$\sigma_0,\tau_0\in[0,\infty)$ such that
\begin{alignat*}{2}
&\htam{ if }\sigma\in\R\htam{, then }
  & \sigma a\leq b 
  &\iff \sigma\leq\sigma_0\htam{, } \\
&\htam{ if }\tau\in\R\htam{, then }
  & \tau b \leq a 
  &\iff \tau \leq \tau_0\htam{.}
\end{alignat*}
If $\sigma\in\R$ with $\sigma>\sigma_0$
we have $\sigma a\nleq b$,
so $b\leq \sigma a$,
$\sigma^{-1} b\leq a$
(note that $\sigma > \sigma_0 \geq 0$),
and therefore $\sigma^{-1}\leq \tau_0$
and $1\leq \sigma\tau_0$.
Hence, $1\leq \inf_{\sigma > \sigma_0}\sigma \tau_0 = \sigma_0 \tau_0$.
This yields $\sigma_0 a \leq b \leq \sigma_0 \tau_0 b \leq \sigma_0 a$,
so that $b = \sigma_0 a$. \xqed
\end{enumerate}
\end{proof}
%
%                  3.6
%
\begin{psec}{3.6}%
Let $E$ be a Riesz space.
A Riesz ideal $D$ of $E$ is called \keyword{proper}
if $D\neq E$.
Under inclusion,
the proper Riesz ideals of~$E$
form an ordered set.
The maximal elements of this set are,
by definition,
the \keyword{maximal Riesz ideals} of~$E$.

Let $M$ be such a maximal ideal and 
let $Q\colon E\ra E/M$ be the quotient map.
$E/M$ is a Riesz space.
If $D$ is a Riesz ideal in $E/M$,
then one easily sees 
that $Q^{-1}(D)$ is a Riesz ideal in~$E$
that contains~$M$.
The maximality of~$M$ 
implies that~$Q^{-1}(D)$ is either $M$ or $E$,
so that~$D$ is either~$\{ 0\}$ or~$E/M$.

Thus, $E/M$ is Riesz isomorphic to~$\R$,
and~$M$ is the kernel of a Riesz homomorphism
$E\ra \R$.
\end{psec}
%
%                  3.7
%
\begin{psec}{3.7}%
Let $E$ be a unitary Riesz space.
\begin{enumerate}
\item \label{3.7-1}
Choose a unit $e$ in $E$. 
The only Riesz ideal that contains~$e$ is~$E$ itself.
Hence, if~$D$ is any Riesz ideal, then
\begin{equation*}
D\htam{ is proper }\quad \iff \quad e\notin D\htam{.}
\end{equation*}
%
\item \label{3.7-2}
Let $Z$ be a subset of $E$.
Zorn's Lemma implies:
If $D_0$ is a Riesz ideal
and~$D_0\subseteq Z$,
then among the Riesz ideals~$D$ 
with $D_0\subseteq D\subseteq Z$
there is a maximal one.
%
\item \label{3.7-3}
Consequently:
Every proper Riesz ideal of~$E$
is contained in a maximal proper Riesz ideal.
%
\item \label{3.7-4}
Therefore,
with the last part of~\ref{3.6}:
If $D$ is any proper Riesz ideal of~$E$,
there exists a nonzero Riesz homomorphism
$\varphi\colon E\ra\R$ whose kernel contains~$D$.
\end{enumerate}
\end{psec}
%
%                  3.8
%
\begin{psec}{3.8}{Example}
Consider the (unitary) space $\ell^\infty$
of~\ref{2.8}\ref{2.8-2}.
The evaluations $x\mapsto x_n$ 
are Riesz homomorphisms $\ell^\infty\ra \R$,
as are their positive scalar multiples
(and the zero function, of course).

It follows from~\ref{3.7}\ref{3.7-4}
that there are other Riesz homomorphisms
$\ell^\infty \ra \R$.
Indeed,
$\cseq_0$ (see \ref{1.5}\ref{1.5-5}) 
is a proper Riesz ideal in~$\ell^\infty$,
so there must be nonzero Riesz homomorphisms
$\ell^\infty\ra \R$ that vanish on~$\cseq_0$.
\end{psec}
%
%                  3.8b
%
\begin{psec}{3.8b}{Exercise}
Prove the following theorem:%
\statement{
Let~$E$ be an Archimedean Riesz space
with a unit~$e$,
and let~$D$ be a Riesz subspace of~$E$
that contains~$e$.
Then every Riesz homomorphism $D\ra \R$
extends to a Riesz homomorphism $E\ra \R$.
}
\end{psec}
%
%                  3.9
%
\begin{psec}{3.9}{Lemma}
\statement{
Let $E$ be an Archimedean unitary Riesz space
and let $a\in E$, $a>0$.
Then there is a Riesz homomorphism
$\varphi\colon E \ra \R$ with $\varphi(a)>0$.
}
\end{psec}
\begin{proof}
Choose a unit $e$ in $E$.
As $E$ is Archimedean
there is an $n$ in $\N$
with $na\nleq e$,
i.e. $na-e\nleq 0$.
But $na-e\leq(na-e)^+$,
so $(na-e)^+\neq 0$.

By exercise~\ref{2.14},
$J:=\{x\colon x\perp(na-e)^+\}$
is a Riesz ideal of~$E$;
it is proper because it does not contain $(na-e)^+$.
By~\ref{3.7}\ref{3.7-4}
there exists a nonzero Riesz homomorphism 
$f\colon E\ra \R$
whose kernel contains $J$.
This kernel is a Riesz ideal,
proper because~$f$ is nonzero.
Hence, $f(e)>0$.

Now $(na-e)^-\in J$,
so $\smash{0=f\bigl((na-e)^-\bigr)=\bigl(nf(a)-f(e)\bigr)^-}$,
which implies that
$nf(a)-f(e)=(nf(a)-f(e))^+\geq 0$.
Thus,
$nf(a)\geq f(e)>0$ 
and~$f(a)>0$.  \xqed
\end{proof}
%
%                  3.10
%
\begin{psec}{3.10}%
For the main theorem of this chapter
we need some results from functional analysis.
\begin{enumerate}
\item \label{3.10-1}
Let $X$ be a topological space.

For $f\in \BCont{X}$ (\ref{1.5}\ref{1.5-3}) we set
\begin{equation*}
\| f \|_\infty := \sup \bigl\{ | f(x) | \colon x \in X\bigr\}\htam{.}
\end{equation*}
The function $\|\cdot \|_\infty$ is a norm on $\BCont{X}$.
The convergence it induces is uniform convergence:
$\|f_n-f\|_\infty\ra 0$ if and only if for every~$\varepsilon >0$
there is an $N\in \N$ with
\begin{equation*}
\htam{ if }n\geq N
\htam{, then }|f_n(x)-f(x)|\leq \varepsilon
\htam{ for all }x\htam{.}
\end{equation*}
%
\item \label{3.10-2}
\textbf{Stone--Weierstrass Theorem}\  \statement{%
Let $X$ be a compact Hausdorff space,
$D$ a Riesz subspace of $\Cont{X}$($=\BCont{X}$)
with the properties
\begin{align*}
& \mathbb{1}\in D\htam{, } \\
& \htam{if }x,y\in X\htam{ and }x\neq y
\htam{, there is an }f\htam{ in }D
\htam{ with }f(x)\neq f(y)\htam{.}
\end{align*}
Then $D$ is dense in $\Cont{X}$.
}
\end{enumerate}
\end{psec}
%
%                  3.11
%
\begin{psec}{3.11}{Definition}
Let $E$ be an Archimedean Riesz space
with a (strong) unit~$e$.
For every $x\in E$
there exist positive numbers~$\lambda$
with $|x|\leq \lambda e$;
define
\begin{equation*}
\| x \|_e := \inf\bigl\{ \lambda\in[0,\infty): |x|\leq \lambda e\bigr\}\htam{.}
\end{equation*}
Thus, we obtain a function $\|\cdot\|\colon E \ra [0,\infty)$.
\end{psec}
%
%                  3.12
%
\begin{psec}{3.12}{Exercise}
(Situation of~\ref{3.11}) Prove the following.
\begin{enumerate}
\item \label{3.12-1}
If $x\in E$, then 
\begin{equation*}
|x|\leq \|x\|_e e\htam{.}
\end{equation*}
(See exercise~\ref{3.2}\ref{3.2-alpha}$\implies$\ref{3.2-delta}.)
%
\item \label{3.12-2}
$\|\cdot\|_e$ is a norm.
%
\item \label{3.12-3}
If $a,x_1,x_2,\dotsc\in E$,
the sequence $(x_n)_n$
converges to~$a$ in the sense
of the norm~$\|\cdot\|_e$
if and only if 
for every $\varepsilon>0$ there exists 
an~$N$ with
\begin{equation*}
|x_n-a|\leq \varepsilon e \quad\htam{ if }\quad n\geq N\htam{.} 
\end{equation*}
%
\item \label{3.12-4}
If $x,y\in E$ and $0\leq x\leq y$,
then $\|x\|_e \leq \|y\|_e$.
%
\item \label{3.12-5}
$\| \, | x|\, \|_e=\| x \|_e$ for all $x\in E$.
%
\item \label{3.12-6}
If $X$ is a set and $E=\ell^\infty(X)$
(\ref{1.5}\ref{1.5-2}),
or if~$X$ is a topological space 
and $E=\BCont{X}$,
then $\|\cdot\|_\mathbb{1}$ 
is the supremum norm $\|\cdot\|_\infty$.
\end{enumerate}
\end{psec}
%
%                  3.13
%
\begin{psec}{3.13}{Exercise}
(Situation of~\ref{3.11} and~\ref{3.12})
Show that for all $x,y\in E$
\begin{align*}
\|\,|x|-|y|\,\|_e & \leq \|x-y\|_e\htam{,}\\
\| x^+ - y^+ \|_e & \leq \|x-y\|_e\htam{.}
\end{align*}
Thus, the operations $x\mapsto |x|$
and $x\mapsto x^+$
are (uniformly) continuous.
\end{psec}
%
%                  3.14
%
\begin{psec}{3.14}{Yosida Representation Theorem}
\statement{
Let $E$ be an Archimedean Riesz space
with a unit, $e$.
Let $\Phi$ be the set of all Riesz homomorphisms
$\varphi\colon E \ra \R$ with $\varphi(e)=1$,
topologized as a subset of $\R^E$ (product topology).
For $x\in E$ define a function~$\hat{x}$ on~$\Phi$ by
\begin{equation*}
\hat{x}(\varphi):=\varphi(x)\qquad (\varphi\in \Phi)\htam{,}
\end{equation*}
(so that $\hat{e}=\mathbb{1}$),
and put $\hat{E}:=\{\hat{x}\colon x\in E\}.$
Then:}

\statement{
$\Phi$ is a compact Hausdorff space,
$\hat{E}$ is a Riesz subspace of $\Cont{\Phi}$,
dense in the sense of~$\|\cdot\|_\infty$.
The map $x\mapsto \hat{x}$
is a Riesz isomorphism $E\ra \hat{E}$,
and $\|x\|_e = \|\hat{x}\|_\infty$
for all~$x$ in~$E$.
}

The space~$\Phi$ is called 
the \keyword{spectrum} of the Riesz space~$E$.
\end{psec}
\begin{proof}
\begin{enumerate}[label=(\Roman*)]
\item \label{3.14-I}
If $x\in E$, then $|x|\leq \| x\|_e\, e$\ 
 (\ref{3.12}\ref{3.12-1}),
so that for all~$\varphi$ in~$\Phi$
\begin{equation*}
|\varphi(x)|=\varphi(|x|)\leq \|x\|_e\, \varphi(e)=\|x\|_e\htam{.}
\end{equation*}
It follows that (also topologically)
\begin{equation*}
\Phi 
\ \subseteq  \ 
\prod_{x\in E} \bigl[-\| x \|_e , \|x \|_e \bigr] 
\ \subseteq \ \R^E\htam{.}
\end{equation*}
Hence,
by Tychonoff's Theorem,
$\Phi$ will be compact if it is closed in~$\R^E$.
Thus, let $(\varphi_i)_{i\in I}$
be a net in~$\Phi$,
converging to some $\varphi\in\R^E$;
we need $\varphi\in\Phi$.
From the definition of the product topology we obtain
\begin{equation}
\label{eq3.14-I} \tag{$*$}
\varphi_i(x) \longrightarrow \varphi(x) \qquad (x\in E)\htam{.}
\end{equation}
Hence, 
if $x,y\in E$,
then $\varphi_i(x)\rightarrow \varphi(x)$,
$\varphi_i(y)\rightarrow \varphi(y)$,
$\varphi_i(x+y)\rightarrow \varphi(x+y)$,
whereas $\varphi_i(x)+\varphi_i(y)=\varphi_i(x+y)$ for all $i$;
it follows that $\varphi(x)+\varphi(y) = \varphi(x+y)$.
Similarly, $\varphi(\lambda x)=\lambda \varphi(x)
\quad (x\in E,\ \lambda\in \R)$,
$\varphi(|x|)=|\varphi(x)|\quad(x\in E)$,
and $\varphi(e)=1$.
Thus~$\varphi\in\Phi$.
%
\item \label{3.14-II}
Formula~\eqref{eq3.14-I},
read as ``$\hat{x}(\varphi_i)\rightarrow\hat{x}(\varphi)\quad (x\in E)$'',
shows that $\hat{E}\subseteq\Cont{\Phi}$.
From the definitions it is apparent that $x\mapsto\hat{x}$
is a Riesz homomorphism $E\ra\Cont{\Phi}$.
Then its range, $\hat{E}$, 
is a Riesz subspace of~$\Cont{\Phi}$.
%
\item \label{3.14-III}
If $\varphi,\psi\in\Phi$ and $\varphi\neq\psi$,
then trivially there is an~$x$ in~$E$ with $\varphi(x)\neq\psi(x)$,
i.e. $\hat{x}(\varphi) \neq\hat{x}(\psi)$.
Hence, $\hat{E}$ is dense in~$\Cont{\Phi}$ by the 
Stone--Weierstrass Theorem.
%
\item \label{3.14-IV}
It remains to prove that the map $x\mapsto \hat{x}$
is injective.
Let $a\in E$, $a\neq 0$;
it suffices to show that $\hat{a}\neq 0$.
By lemma~\ref{3.9},
there is a Riesz homomorphism $\omega\colon E\ra \R$
with $\omega(|a|)>0$.
As $|a|\leq ne$ for some $n\in \N$
we have $\omega(e)>0$.
Setting $\varphi:=\omega(e)^{-1} \omega$
we find $\varphi\in \Phi$ 
and $|\hat{a}(\varphi)| = |\varphi(a)| = \varphi(|a|)>0$.
%
\item \label{3.14-V}
Finally,
$x\mapsto \hat{x}$
being an isomorphism $E\ra \hat{E}$:
\begin{equation*}
\|x\|_e = \|\hat{x}\|_{\hat{e}} = \|\hat{x}\|_\mathbb{1} 
= \|\hat{x}\|_\infty\qquad (x\in E)\mathrlap{\htam{. \qquad \xqed}}
\end{equation*}
\end{enumerate}
\end{proof}
%
%                  3.15
%
\begin{psec}{3.15}{Exercise}
Let $X$ be a compact Hausdorff space.
An application of Yosida's Theorem with $E=\Cont{X}$,
$e=\mathbb{1}$ yields
a compact Hausdorff space~$\Phi$
and a map $f\mapsto\hat{f}$ of~$\Cont{X}$ into~$\Cont{\Phi}$.
Show that~$\Phi$ is homeomorphic to~$X$
and that, in fact, there exists a homeomorphism
$\tau\colon \Phi\ra X$ with $\hat{f}=f\circ \tau\quad(f\in \Cont{X}\, )$.
Infer that $\hat{E}$ is all of $\Cont{\Phi}$.
\end{psec}
%
%                  3.16
%
\begin{psec}{3.16}{Exercise}
(Situation as in~\ref{3.14}.)
A natural question is,
whether the map $x\mapsto \hat{x}$ is surjective.
(We return to this problem later on.)
Prove that $\hat{E}=C(\Phi)$ if $E$ has this completeness property:
\begin{quote}
If $(x_n)_n$ and $(y_n)_n$ are sequences in~$E$
such that $x_n \leq y_m\quad (n,m\in \N)$
and $(y_n-x_n)\leq n^{-1} e\quad (n\in \N)$,
there is a~$z\in E$ with $x_n\leq z\leq y_m \quad (n,m\in \N)$.
\end{quote}
\end{psec}
%
%                  3.17
%
We use Yosida's theorem
(which, by the way, 
is due not only to 
K.~Yosida
but also to 
S.~Kakutani,
M.~and S.~Krein,
and H.~Nakano)
to prove a representation theorem
for abstract Boolean algebras.
\begin{psec}{3.17}{Definitions}
Let $X$ be a topological space.
A subset~$Y$ of~$X$
is called \keyword{clopen} if it is both closed and open.
The clopen subsets of~$X$ form an algebra,
\begin{equation*}
\clopen(X)\htam{.}
\end{equation*}
$X$ is said to be \keyword{zerodimensional}
if the clopen sets form a base 
for the topology (e.g.,~$\mathbb{Q}$, 
or $\{0,1,\sfrac{1}{2},\sfrac{1}{3},\sfrac{1}{4},\dotsc\}$).
\end{psec}
%
%                  3.18
%
\begin{psec}{3.18}{Exercise}
Let $X$ be a compact topological space,
$\mathcal{R}$ an algebra of clopen subsets of~$X$.
Suppose $\mathcal{R}$ ``separates the points of~$X$''
in the sense that for any two distinct points of~$X$
there is a set in~$\mathcal{R}$
containing one of them but not the other.
Prove that~$X$ is zerodimensional
and $\mathcal{R}=\clopen(X)$.
\end{psec}
%
%                  3.19
%
\begin{psec}{3.19}{Exercise}
Applying Yosida's Theorem to $\ell^\infty$ (\ref{2.8}\ref{2.8-2})
we obtain a compact Hausdorff space~$\Phi$
and a Riesz isomorphism $f\mapsto \hat{f}$
of~$\ell^\infty$ onto a Riesz subspace of~$\Cont{\Phi}$
sending~$\mathbb{1}_\N$ to~$\mathbb{1}_\Phi$,
implying
\begin{equation*}
\|\hat{f}\|_\infty = \|f\|_\infty \qquad (f\in \ell^\infty)\htam{.}
\end{equation*}
\begin{enumerate}
\item \label{3.19-1}
If $A\subseteq \N$,
then $\mathbb{1}_A \wedge (\mathbb{1}_\N - \mathbb{1}_A)=0$
in~$\ell^\infty$.
Deduce that~$\hat{\mathbb{1}}_A$ is the indicator
of some clopen subset~$\hat{A}$ of~$\Phi$.
%
\item \label{3.19-2}
By $\mathcal{P}(\N)$ we denote the algebra (\ref{1.1}\ref{1.1-4})
of all subsets of~$\N$.
Put
\begin{equation*}
\mathcal{R}:=\{ \hat{A}\colon A \subseteq \N \}\htam{.}
\end{equation*}
Show that $\mathcal{R}$ is an algebra of sets
(and a sublattice of $\clopen(\Phi)$)
and that $A\mapsto \hat{A}$
is a lattice isomorphism $\mathcal{P}(\N)\ra \mathcal{R}$.
%
\item \label{3.19-3}
Prove: If $f\in\Cont{\Phi}$ and $\varepsilon>0$,
there exist pairwise disjoint $B_1,\dotsc,B_N\in\mathcal{R}$
and real numbers $s_1,\dotsc,s_N$
such that $\bigcup B_n = \Phi$
and $| f- \sum s_n \mathbb{1}_{B_n}| \leq \varepsilon \mathbb{1}_\Phi$.
%
\item \label{3.19-4}
Infer that $\mathcal{R}$ is a base for the topology of~$\Phi$.
%
\item \label{3.19-5}
We prove the following remarkable property of~$\Phi$:
If~$(\varphi_n)_n$ is a converging sequence in~$\Phi$,
then there is an~$N$ with $\varphi_N=\varphi_{N+1}=\varphi_{N+2}=\dotsb$.

Indeed,
suppose we have a sequence~$(\varphi_n)_n$ in~$\Phi$
with limit~$\varphi$
for which no such~$N$ exists.
Show that there is a sequence 
$n_1,B_1,n_2,B_2,n_3,\dotsc$ 
such that
\begin{equation*}
\left\{
\begin{aligned}
&n_1 < n_2 < \dotsb\htam{ in }\N\htam{, } \\
&B_1\supset B_2 \supset \dotsb\htam{ in }\mathcal{R}\htam{,} \\
&\varphi\in B_i\htam{ and } \varphi_{n_i} \notin B_i \quad 
    &&(i\in \N)\htam{,} \\
&\varphi_n\in B_i\htam{ if }n\geq n_{i+1} 
    && (i,n\in \N)\htam{.}
\end{aligned}
\right. 
\end{equation*}

Use~\ref{3.19-2} to obtain a set~$C$ in~$\mathcal{R}$
such that $B_i \backslash B_{i+1}\subseteq C$
if~$i$ is even,
$B_i \backslash B_{i+1} \subseteq \Phi\backslash C$
if~$i$ is odd, 
implying that $\varphi_{n_i}\in C$
if~$i$ is odd,
$\varphi_{n_i}\in\Phi\backslash C$
if~$i$ is even.
Show that this is false.
\end{enumerate}
\end{psec}
%
%                  3.20
%
\begin{psec}{3.20}{Stone Representation Theorem}
\statement{
Every Boolean algebra is isomorphic
to the Boolean algebra of all clopen subsets of some 
zerodimensional compact Hausdorff space.
}
\end{psec}
The proof of this theorem
is a complicated and abstract affair.
The following lines may give an idea 
of what is really going on.

The basic idea is to use Yosida's Theorem,
applied to a suitable Riesz space.
Such a space is easily found if~$\Lambda$
is not an abstract Boolean algebra
but an algebra of subsets of a set~$X$:
Take the space~$\lbb\Lambda\rbb$
of all $\Lambda$-step functions,
with~$\mathbb{1}_X$ as the unit.
It is not difficult to show that then,
under Yosida's Theorem,
indicators of subsets of~$X$ correspond 
to indicators of subsets of~$\Phi$,
yielding a bijection $\Lambda \ra \clopen(X)$.

What we need is a proof that can be applied
to general Boolean algebras,
not just to algebras of sets.
Therefore,
we reconsider the above:
Can we describe~$\lbb \Lambda \rbb$,
as an abstract Riesz space,
in terms of~$\Lambda$
without explicit mention of the points of~$X$?

There are in fact,
various ways to do so.
The following idea is quite natural:
view a $\Lambda$-step function
\begin{equation*}
\label{eq3.20rem} \tag{$*$}
\lambda_1 \mathbb{1}_{A_1} + \dotsb + \lambda_N \mathbb{1}_{A_N}
\end{equation*}
(with $A_1,\dotsc,A_N$ disjoint)
as a function defined on the finite set
$\{A_1,\dotsc,A_N\}$, sending~$A_n$ to~$\lambda_n\quad (n=1,\dotsc,N)$.
Using this idea in the Boolean algebra situation
one can define a, say,
``abstract step function''
to be a function on a finite subset $\{a_1,\dotsc,a_N\}$
of the Boolean algebra
with $a_n\wedge a_m=0\quad(n\neq m)$.
With sensibly defined addition and ordering
this yields a Riesz space,
and a proof of the theorem.

The drawback of this approach is that checking the details
turns out to be very laborious.
Accordingly,
we prefer the following variation:
We represent the step function~\eqref{eq3.20rem}
by a function defined on all of~$\Lambda$,
assigning the value~$\lambda_n$ to~$A_n$
\emph{and to all subsets of}~$A_n$,
with arbitrary values elsewhere.
We now have a ready-made Riesz space structure
that precisely fits our purposes,
except that the same step function will have many
representations.
That difficulty we overcome by dividing by
the right Riesz ideal.
(The reader will have noticed that
a function cannot really for each~$n$ assign the value~$\lambda_n$
to~$A_n$ and all of its subsets.
Our proof deals with the functions defined on~$\Lambda\backslash\{0\}$
rather than the entire~$\Lambda$.)
\begin{proof}
of the Stone Representation Theorem.
Let $\Lambda$ be a Boolean algebra with $0\neq 1$.
Put $\Lambda^*:=\Lambda \backslash\{0\}$.
\begin{enumerate}[label=(\Roman*),itemindent=3em,labelwidth=3em]
\item \label{3.20-I}
By a ``partition'' of~$\Lambda^*$
we mean a finite subset~$S$ of~$\Lambda^*$
such that
\begin{alignat*}{2}
& \htam{if }s_1,s_2\in S\htam{ and }s_1\neq s_2
   \htam{, then }s_1\wedge s_2=0\htam{;} \\
& \sup S=1\htam{.}
\end{alignat*}
A partition~$T$ is a ``refinement'' of a partition~$S$
if for every~$t\in T$ there is an~$s\in S$ with $t\leq s$.

For a partition $S$ let $E_S$ be the collection
of all functions $f\colon \Lambda^* \ra \R$ with the property
\begin{align*}
s\in S\htam{, }\ x\in\Lambda^*\htam{, }\ x\leq s
\quad \implies \quad f(x)=f(s)\htam{.}
\end{align*}

As an example,
for each $a\in\Lambda$ 
we define $e_a\colon \Lambda^* \ra \R$ by
\begin{alignat*}{2}
e_a(x) := 1 \quad & \htam{ if } x\leq a\htam{,} \\
e_a(x) := 0 \quad & \htam{ if } x\nleq a\htam{.}
\end{alignat*}
If $a\neq0,1$, we have $e_a\in E_S$ 
where $S$ is the partition $\{a, a'\}$,
$a'$ being the complement of~$a$.
The functions~$e_0$ and~$e_1$ lie in~$E_{\{1\}}$,
and
\begin{equation*}
e_0 = 0\htam{, } \qquad e_1 = \mathbb{1}_{\Lambda^*}\htam{.}
\end{equation*}
%
\item \label{3.20-II}
Every $E_S$ is a Riesz subspace of $\R^{\Lambda^*}$.
If~$S$ and~$T$ are partitions
and~$T$ is a refinement of~$S$,
then $E_S\subseteq E_T$.
Any two partitions $S_1,S_2$ 
have a common refinement,
e.g., $\{ s_1 \wedge s_2\colon s_1\in S_1\htam{, }s_2\in S_2\htam{, }\htam{, }
  s_1 \wedge s_2 \neq 0\}$.

It follows from these considerations 
that the union of all spaces~$E_S$
is a Riesz subspace of~$\R^{\Lambda^*}$.
We call this union~$E$.
%
\item \label{3.20-III}
Let $D$ be the collection of all functions $f$ on~$\Lambda^*$
for which there is a partition~$S$ such that
\begin{equation*}
f\in E_S \quad \htam{ and } \quad f(s)=0 \qquad (s\in S)\htam{.}
\end{equation*}
Observe: if $f\in D$ 
and if~$T$ is \keyword{any} 
partition with $f\in E_T$,
then $f(t)=0$ for all $t\in T$.
(Indeed, if $t\in T$,
there is an $s\in S$ with $s\wedge t\neq 0$;
then $f(t)=f(s\wedge t)=f(s)=0$.)

$D$ is easily seen to be a Riesz ideal in~$E$.
We are going to apply the Yosida Theorem to~$E/D$.
To this end, 
we prove~$E/D$
to be unitary and Archimedean.
%
\item \label{3.20-IV}
First we describe the ordering on $E/D$.
Let $Q\colon E\ra E/D$ be the quotient map.

Let $f,g\in E$. 
Choose a partition $S$ with $f,g\in S$.
\begin{align*}
Qf \leq Qg 
& \iff Qg = Qf \wedge Qg \\
& \iff Q(g-f\wedge g)=0 \\
& \iff g-f\wedge g \in D \\
& \relnote{\iff}{(*)} g-f\wedge g =0\quad\htam { on }S \\
& \iff f\leq g\quad\htam{ on }S
\end{align*}
(Step $(*)$ uses the fact that $g-f\wedge g\in E_S$.)
%
\item \label{3.20-V}
($E/D$ is unitary.)
Take $f\in E$.
Choose a partition $S$ with $f\in E_S$,
and observe that also
$\mathbb{1}_{\Lambda^*}\in E_S$.
As $S$ is finite, 
there is an~$n$ in~$\N$
with $|f|\leq n \mathbb{1}_{\Lambda^*}$ on $S$.
Then $|Qf|\leq n Q\mathbb{1}_{\Lambda^*}$.

Thus, $Q\mathbb{1}_{\Lambda^*}$ is a unit in $E/D$.
%
\item \label{3.20-VI}
($E/D$ is Archimedean.)
Let $f,g\in E^+$ be so that $nQf\leq Qg$ for all $n\in \N$;
we prove $Qf=0$.
Take a partition~$S$ with $f,g\in E_S$.
As also $nf\in E_S$
for all~$n$
we have $nf(s)\leq g(s)\quad(n\in \N,\ s\in S)$.
It follows that $f(s)=0\quad (s\in S)$,
so $f\in D$ and $Qf=0$.
%
\item \label{3.20-VII}
Yosida's Theorem produces a compact Hausdorff space~$\Phi$
and a Riesz isomorphism of~$E/D$ 
onto a dense Riesz subspace of~$\Cont{\Phi}$
that sends~$Q\mathbb{1}_{\Lambda^*}$ to~$\mathbb{1}_\Phi$.
Composition of this isomorphism with~$Q$ 
yields a Riesz homomorphism $f\mapsto \tilde{f}$
of~$E$ onto~$\Cont{\Phi}$ for which
\begin{alignat*}{2}
& \tilde{f} = 0 \quad \iff \quad f\in D\qquad (f\in E) \\
& \{ \tilde{f}\colon f\in E\} \htam{ is dense in } \Cont{\Phi}\htam{, } \\
& \tilde{\mathbb{1}}_{\Lambda^*} = \mathbb{1}_\Phi\htam{.}
\end{alignat*}

It is elementary that the maps $a\mapsto Qe_a$ 
and, therefore, $a\mapsto \tilde{e}_a$,
are lattice homomorphisms.
It follows in particular that for every~$a$
\begin{equation*}
\tilde{e}_a \vee \tilde{e}_{a'} = \mathbb{1}_\Phi\htam{, }
\qquad \tilde{e}_a \wedge \tilde{e}_{a'}=0\htam{, }
\end{equation*}
so that~$\tilde{e}_a$ is the indicator of a certain subset,
$\Phi_a$, of~$\Phi$.
As~$\tilde{e}_a$ is continuous,
the set~$\Phi_a$ is clopen.

We now have a lattice homomorphism $a\mapsto \Phi_a$
of~$\Lambda$ into~$\clopen(\Phi)$.
%
\item \label{3.20-VIII}
The proofs of the following two statements 
we leave to the reader.
\begin{itemize}
\item 
If $a,b\in \Lambda$ and $e_a - e_b \in D$, then $a=b$.
\item
If $S$ is a partition and $f\in E_S$, 
then $f-\sum_{s\in S} f(s)e_s \in D$.
\end{itemize}
The first of these statements implies
that the map $a\mapsto \Phi_a$ is injective.
Denoting by~$\mathcal{R}$ 
the algebra of sets $\{ \Phi_a\colon a\in \Lambda \}$
we see from the second statement that every~$\tilde{f}\quad(f\in E)$
is an $\mathcal{R}$-step function.
But $\{\tilde{f}\colon f\in E\}$ is dense in~$\Cont{\Phi}$.
Then~$\mathcal{R}$ separates the points of~$\Phi$
(see~\ref{3.18});
so~$\Phi$ is zerodimensional and $\mathcal{R}=\clopen(\Phi)$.~\xqed
\end{enumerate}
\end{proof}
%
%                  3.21
%
\begin{psec}{3.21}{Definition}
Given a Boolean algebra $\Lambda$,
a zerodimensional compact Hausdorff space~$X$
for which $\clopen(X)$ is isomorphic to~$\Lambda$
is called a \keyword{Stone space} for~$\Lambda$.

The Stone space is essentially unique:
\end{psec}
%
%                  3.22
%
\begin{psec}{3.22}{Exercise}
Let $X$ and $Y$ be zerodimensional compact Hausdorff spaces.
Assume~$T$ is a lattice isomorphism $\clopen(X)\ra\clopen(Y)$.
Prove that there is a homeomorphism $\tau\colon X\ra Y$
such that for $x\in X$ and $A\in\clopen(X)$
\begin{equation*}
x\in A\quad \iff \quad \tau(x)\in T(A)\htam{.}
\end{equation*}
\end{psec}
\clearpage
\end{document}
