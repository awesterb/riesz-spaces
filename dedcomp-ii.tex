\documentclass[main.tex]{subfiles}
\begin{document}
% 6
\section{Dedekind Completeness, II}
Another application concerns an extension of the
representation theory for Hermitian operators.
First, some topology.
%
%                  7.1,  Baire Category Theorem
%
\begin{psec}{7.1}{Baire Category Theorem}\statement{
Let $X$ be either a compact Hausdorff space
or a complete metric space.
\begin{enumerate}
\item\label{7.1-1}
If $U_1, U_2,\dotsc$ are dense open sets,
then $\bigcap U_n$ is dense.
%
\item\label{7.1-2}
If $C_1,C_2,\dotsc$ are closed sets with empty interior,
then $\bigcup C_n$ has empty interior.
%
\item\label{7.1-3}
If $C_1,C_2,\dotsc$ are closed sets and $\bigcup C_n=X$,
then at least one~$C_n$ has nonempty interior
---
provided~$X$ is nonempty.
\end{enumerate}
}\end{psec}
\begin{proof}
For $Y\subseteq X$
the statements ``$Y$ is dense''
and ``$X\backslash Y$ has empty interior''
are equivalent.
Consequently,
we have \ref{7.1-1}$\iff$\ref{7.1-2}$\implies$\ref{7.1-3},
and it suffices to prove~\ref{7.1-1}.
Let $U_1,U_2,\dotsc$ be dense open sets
and let~$W$ be open, nonempty;
we wish to show that $W\cap \bigcap U_n\neq \varnothing$.

As $U_1$ is open and dense, $W\cap U_1$
is a nonempty open set.
Then there is a nonempty open~$V_1$
with $\overline{V}_1\subseteq W\cap U_1$.
Similarly,
$V_1\cap U_2$ is nonempty and open:
Choose a nonempty open~$V_2$ with $\overline{V}_2\subseteq V_1\cap U_2$.
Continuing in this fashion
one obtains the following scheme:
\begin{equation*}
\xymatrix@-1em{
W 
  \ar@{}[r]|*{\supseteq}&
\overline{V}_1 
  \ar@{}[r]|*{\supseteq}&
V_1
 \ar@{}[r]|*{\supseteq}&
\overline{V}_2 
 \ar@{}[r]|*{\supseteq}  &
V_2
  \ar@{}[r]|*{\supseteq} &
\overline{V}_3
  \ar@{}[r]|*{\supseteq}  &
\dotsb \\
&
U_1
 \ar@{}[u]|*{\begin{turn}{90}$\supseteq$\end{turn}} &
&
U_2 \ar@{}[u]|*{\begin{turn}{90}$\supseteq$\end{turn}} &
&
U_3 \ar@{}[u]|*{\begin{turn}{90}$\supseteq$\end{turn}} &
}\end{equation*}
Now $\bigcap \overline{V}_n \subseteq W\cap \bigcap U_n$;
and we are done in case $\bigcap \overline{V}_n \neq \varnothing$.

If $X$ is compact,
this is automatic.
If~$X$ is a complete metric space,
we insert one clause in the above construction,
viz., that~$V_n$ be a ball with radius~$\leq n^{-1}$.
\xqed
\end{proof}
%
%                  7.2
%
\begin{psec}{7.2}{Definition}
Let $X$ be a compact Hausdorff space.

A subset~$Y$ of~$X$ is called \keyword{meager}
if there exist subsets $C_1,C_2,\dotsc$ of~$X$ such that
\begin{equation*}
\left\{\ 
\begin{aligned}
& \htam{every }C_n\htam{ is closed and has empty interior,} \\
& \textstyle Y\subseteq \bigcup C_n\htam{.}
\end{aligned}
\right.
\end{equation*}
\end{psec}
%
%                  7.3
%
\begin{psec}{7.3}%
In a compact Hausdorff space~$X$,
meager sets are small in a sense illustrated by the following observations.
\begin{itemize}[itemindent=1.2em]
\item
Meager sets have empty interior;
in other words:
the complement of a meager set is dense.
(This is Baire's Theorem~\ref{7.1}.)
%
\item
The only meager open set is~$\varnothing$.
%
\item
A closed set is meager
if and only if its interior is empty.
%
\item
Subsets of meager sets are meager.
%
\item
A union of countably many meager sets is meager.
\end{itemize}

Meager sets are in topology
more or less what null sets are in measure theory.
\end{psec}
%
%                  7.4
%
\begin{psec}{7.4}{Examples}
In $[0,1]$,
every countable set is meager.

Let $(q_1,q_2,\dotsc)$ be an enumeration of $[0,1]\cap\mathbb{Q}$,
and for $\varepsilon>0$ let
\begin{equation*}
U_\varepsilon\ =\ [0,1]\,\cap\,{\bigcup}_n
  \bigl(q_n-\varepsilon\, 2^{-n},\, q_n +\varepsilon\, 2^{-n}\bigr)\htam{.}
\end{equation*}
$U_\varepsilon$ is a dense open subset of~$[0,1]$.

Then $[0,1]\,\backslash\, U_{\sfrac{1}{5}}$ is meager
but its Lebesgue measure is positive.
The set
\begin{equation*}
\bigcap_{p\in\N} U_{\sfrac{1}{p}}
\end{equation*}
has measure~$0$ but its complement is meager.
\end{psec}
%
%
%
\begin{psec}{7.5}{Exercise}
Let $X$ be a compact Hausdorff space
and let $(f_n)_n$ be a sequence in $\Cont{X}^+$.
Prove the equivalence of:
\begin{enumerate}[label=(\greek*)]
\item\label{7.5-alpha}
In $\Cont{X}$,
the set $\{ f_1,f_2,\dotsc \}$ has infimum~$0$.
%
\item\label{7.5-beta}
The set $\{x\in X\colon \inf_n f_n(x)>0\}$ is meager.
\end{enumerate}
Hint for \ref{7.5-alpha}$\implies$\ref{7.5-beta}:
For $\varepsilon>0$,
consider $X_\varepsilon :=\{x\colon \inf_n f_n(x)\geq\varepsilon\}$.

Now prove:
There is no sequence $(g_n)_n$ in $\Cont{[0,1]}^+$ with
\begin{equation*}
\inf_n g_n(x)\ =\ 0 \quad\iff\quad x\in[0,1]\cap\mathbb{Q}\htam{.}
\end{equation*}
\end{psec}
%
%                  7.6
%
\begin{psec}{7.6}{Definitions}
Let $X$ be a compact Hausdorff space.
For functions~$f$ and~$g$ on~$X$ we put
\begin{align*}
f&\equiv g \\
\shortintertext{%
if the set $\{x\colon f(x)\neq g(x)\}$ is meager.
For subsets~$A$ and~$B$ of~$X$
we put}
A&\equiv B
\end{align*}
if $\mathbb{1}_A\equiv\mathbb{1}_B$,
which is the case if and only if $A\backslash B$
and $B\backslash A$ are meager.

Note that both relations are equivalences.
\end{psec}
%
%                  7.7
%
\begin{psec}{7.7}{Exercise}
Let $X$ be a compact Hausdorff space.
Prove:
\begin{enumerate}
\item\label{7.7-1}
If $U\subseteq X$ is open,
then $U\equiv\overline U$.
%
\item\label{7.7-2}
If $A,B\subseteq X$ and $A\equiv B$,
then $X\backslash A\equiv X\backslash B$.
%
\item\label{7.7-3}
If $A_1,A_2,\dotsc,B_1,B_2,\dotsc\subseteq X$
and $A_n\equiv B_n\quad(n\in\N)$,
then $\bigcup A_n\equiv \bigcup B_n$.
%
\item\label{7.7-4}
If $U$ and $V$ are clopen subsets of~$X$
and $U\equiv V$,
then $U=V$.
(Observe, however, that,
if~$U$ and~$V$ are only open,
then $U\equiv V$ as soon as $\overline U = \overline V$.)
%
\item\label{7.7-5}
If $f_1,f_2,\dotsc,g_1,g_2,\dotsc$ are functions on~$X$
with $f_n\equiv g_n\quad (n\in\N)$,
and if both sequences~$(f_n)_n$ and~$(g_n)_n$
converge pointwise,
then $\lim f_n\equiv \lim g_n$.
%
\item\label{7.7-6}
If $f$ and $g$ are continuous functions on~$X$
and $f\equiv g$, then $f=g$.
\end{enumerate}
\end{psec}
%
%                  7.8
%
\begin{psec}{7.8}{Lemma}\statement{
Let $X$ be a compact Hausdorff space.}

\statement{The subsets $Y$  of $X$ with the property
\begin{equation}\label{eq7.8-1}
Y\equiv U\quad\htam{ for some open set }\ U
\end{equation}
form a $\sigma$-algebra containing all open sets.}

\statement{The condition~\eqref{eq7.8-1}
is equivalent to
\begin{equation}\label{eq7.8-2}
Y\equiv C\quad\htam{ for some closed set }\ C\htam{.}
\end{equation}}

\statement{If $X$ is extremally disconnected,
\eqref{eq7.8-1} and~\eqref{eq7.8-2}
are equivalent to
\begin{equation}\label{eq7.8-3}
Y\equiv W\quad \htam{ for some clopen set }\ W\htam{.}
\end{equation}}
\end{psec}
\begin{proof}
Let $\mathcal A$ be the collection
of all sets with Property~\eqref{eq7.8-1}.
Obviously, $\mathcal A$ contains~$\varnothing$
and all open sets.
If $Y_1,Y_2,\dotsc \in\mathcal A$
and if $U_1,U_2,\dotsc$ are open sets with $A_n\equiv U_n\quad(n\in\N)$,
then $\bigcup Y_n\equiv \bigcup U_n$~(\ref{7.7}\ref{7.7-3}),
so $\bigcup Y_n\in\mathcal A$.
If $Y\in\mathcal A$ and~$U$ is open,
$Y\equiv U$,
then $Y\equiv \overline U$~(\ref{7.7}\ref{7.7-1}),
so that $X\backslash Y\equiv X\backslash\overline U$~(\ref{7.7}\ref{7.7-2}),
and $X\backslash Y\in\mathcal A$.

Thus, $\mathcal A$ is a $\sigma$-algebra.
A set~$Y$ has Property~\eqref{eq7.8-2}
if and only if $X\backslash Y\in\mathcal A$,
if and only if $Y\in\mathcal A$.
Thus, \eqref{eq7.8-1} and~\eqref{eq7.8-2} are equivalent.

If $X$ is extremally disconnected,
then~\eqref{eq7.8-1} implies~\eqref{eq7.8-3}
because for any open set~$U$
we have $U\equiv\overline U$ and~$\overline U$ is clopen. \xqed
\end{proof}
%
%                  7.9
%
\begin{psec}{7.9}{Definition}
Let $X$ be a compact Hausdorff space.
We say that a subset~$Y$ of~$X$ is \keyword{of the first Baire class}
if~$\mathbb{1}_A$ is the (pointwise) limit of a sequence
of continuous functions.
These sets form a collection
\begin{equation*}
\BaireI(X)
\end{equation*}
that is easily seen to be an algebra.
\end{psec}
%
%                  7.10
%
\begin{psec}{7.10}{Comments}
\begin{enumerate}
\item\label{7.10-1}
If $f\in\Cont{X}$,
then the set $\{x\colon f(x)>0\}$ is of the first Baire class,
as its indicator is $\lim \smash{\sqrt[n]{f^+}}$.
Therefore,
if $f\in\Cont{X}$,
sets such as $\{x\colon f(x)>3\}$,
$\{x\colon f(x)\leq 3 \}$,
$\{x\colon 2\leq f(x)\leq 3\}$ lie in~$\BaireI(X)$.
%
\item\label{7.10-2}
It follows that
\begin{equation*}
\Cont{X}\ \subseteq\ \BNL(\BaireI(X))\htam{.}
\end{equation*}
Indeed,
let $f\in\Cont{X}$, $\varepsilon>0$.
Choose $s_0<s_1<\dotsb<s_N$ in~$\R$
with $s_n<s_{n-1}+\varepsilon$
for $n=1,\dotsc,N$
and such that all values of~$f$ lie in the interval $(s_0,s_N)$.
Setting $A_n:=\{x\colon s_{n-1}<f(x)\leq s_n\}$
we obtain $A_1,\dotsc, A_N$ in~$\BaireI(X)$,
pairwise disjoint, and with
\begin{equation*}
0\ \leq\ f-{\textstyle\sum}\,
 s_n\,\mathbb{1}_{A_n}\ \leq\ \varepsilon\,\mathbb{1}\htam{.}
\end{equation*}
%
\item\label{7.10-3}
In the case $X$ is metrizable,
one can show that every open set is of the form $\{x\colon f(x)>0\}$
for some~$f$ in~$\Cont{X}$,
so all open sets are of the first Baire class.
For, e.g., the space $[0,1]^\R$ this is not true.
%
\item\label{7.10-4}
Let $A\in\BaireI(X)$.
Choose $f_1,f_2,\dotsc$ in~$\Cont{X}$
with $f_n\ra \mathbb{1}_A$.
Then an element~$x$ of~$X$ belongs to~$A$
if and only if there is an~$N$ with
\begin{equation*}
\bigl\{ f_N(x),\, f_{N+1}(x),\, \dotsc\,\bigr\}
\ \subseteq\ \bigl(\textstyle\frac{1}{2},\infty\bigr)\htam{.}
\end{equation*}
Hence,
\begin{equation*}
A\ =\ \bigcup_N \bigcap_{n\geq N} \bigl\{ 
x\colon f_n(x)>\textstyle \frac{1}{2} \bigr\}\htam{.}
\end{equation*}
As $\{x\colon f_n(x)>\sfrac{1}{2}\}$ is always open,
it follows that~$A$ lies in the $\sigma$-algebra
described in Lemma~\ref{6.19}.
What we will use farther on is this:
\statement{If~$A$ is a set of the first Baire class
in an extremally disconnected compact Hausdorff space,
then there is a unique clopen set~$U$ with $A\equiv U$.}
\end{enumerate}
\end{psec}
%
%                  7.11
%
\begin{psec}{7.11}%
Next,
we extend the integration theory of~\ref{6.10}.
Instead of an additive $\mu\colon \mathcal A\ra\R^+$
we now consider an additive $\mu\colon \mathcal A\ra E^+$
where~$E$ is a suitable Riesz space:

\vspace{.5em}
\noindent\textbf{Theorem}\  \statement{
Let $\mathcal A$ be an algebra of subsets of a set~$X$.
Let~$E$ be a uniformly complete
unitary Riesz space,
and let~$\mu$ be an additive map $\mathcal A\ra E^+$.
Then there is a unique linear map
\begin{equation*}
f\mapsto \int f\,\mathrm{d}\mu
\end{equation*}
of $\BLeb(\mathcal A)$ into~$E$ with the properties
\begin{equation*}
\left\{\ 
\begin{aligned}
\int \mathbb{1}_A\,\mathrm{d}\mu\ &=\ \mu(A)
  \qquad& (A\in\mathcal A)\htam{,} \\
\int f\,\mathrm{d}\mu \ &\geq\ 0
  \qquad & (f\geq 0)\htam{.}
\end{aligned}
\right.
\end{equation*}}
\end{psec}
\begin{proof}
Let~$e$ be a unit for~$E$;
then under the norm~$\|\cdot\|_e$,
$E$ is complete.
The proof is now nothing but aping our proof of Theorem~\ref{6.10},
replacing~$\R$ by the complete metric space~$E$. \xqed
\end{proof}
%
%
%
\clearpage
\end{document}
