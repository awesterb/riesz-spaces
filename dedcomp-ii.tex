\documentclass[main.tex]{subfiles}
\begin{document}
% 6
\section{Dedekind Completeness, II}
Another application concerns an extension of the
representation theory for Hermitian operators.
First, some topology.
%
%                  7.1,  Baire Category Theorem
%
\begin{psec}{7.1}{Baire Category Theorem}\statement{
Let $X$ be either a compact Hausdorff space
or a complete metric space.
\begin{enumerate}
\item\label{7.1-1}
If $U_1, U_2,\dotsc$ are dense open sets,
then $\bigcap U_n$ is dense.
%
\item\label{7.1-2}
If $C_1,C_2,\dotsc$ are closed sets with empty interior,
then $\bigcup C_n$ has empty interior.
%
\item\label{7.1-3}
If $C_1,C_2,\dotsc$ are closed sets and $\bigcup C_n=X$,
then at least one~$C_n$ has nonempty interior
---
provided~$X$ is nonempty.
\end{enumerate}
}\end{psec}
\begin{proof}
For $Y\subseteq X$
the statements ``$Y$ is dense''
and ``$X\backslash Y$ has empty interior''
are equivalent.
Consequently,
we have \ref{7.1-1}$\iff$\ref{7.1-2}$\implies$\ref{7.1-3},
and it suffices to prove~\ref{7.1-1}.
Let $U_1,U_2,\dotsc$ be dense open sets
and let~$W$ be open, nonempty;
we wish to show that $W\cap \bigcap U_n\neq \varnothing$.

As $U_1$ is open and dense, $W\cap U_1$
is a nonempty open set.
Then there is a nonempty open~$V_1$
with $\overline{V}_1\subseteq W\cap U_1$.
Similarly,
$V_1\cap U_2$ is nonempty and open:
Choose a nonempty open~$V_2$ with $\overline{V}_2\subseteq V_1\cap U_2$.
Continuing in this fashion
one obtains the following scheme:
\begin{equation*}
\xymatrix@-1em{
W 
  \ar@{}[r]|*{\supseteq}&
\overline{V}_1 
  \ar@{}[r]|*{\supseteq}&
V_1
 \ar@{}[r]|*{\supseteq}&
\overline{V}_2 
 \ar@{}[r]|*{\supseteq}  &
V_2
  \ar@{}[r]|*{\supseteq} &
\overline{V}_3
  \ar@{}[r]|*{\supseteq}  &
\dotsb \\
&
U_1
 \ar@{}[u]|*{\begin{turn}{90}$\supseteq$\end{turn}} &
&
U_2 \ar@{}[u]|*{\begin{turn}{90}$\supseteq$\end{turn}} &
&
U_3 \ar@{}[u]|*{\begin{turn}{90}$\supseteq$\end{turn}} &
}\end{equation*}
Now $\bigcap \overline{V}_n \subseteq W\cap \bigcap U_n$;
and we are done in case $\bigcap \overline{V}_n \neq \varnothing$.

If $X$ is compact,
this is automatic.
If~$X$ is a complete metric space,
we insert one clause in the above construction,
viz., that~$V_n$ be a ball with radius~$\leq n^{-1}$.
\xqed
\end{proof}
%
%                  7.2
%
\begin{psec}{7.2}{Definition}
Let $X$ be a compact Hausdorff space.

A subset~$Y$ of~$X$ is called \keyword{meager}
if there exist subsets $C_1,C_2,\dotsc$ of~$X$ such that
\begin{equation*}
\left\{\ 
\begin{aligned}
& \htam{every }C_n\htam{ is closed and has empty interior,} \\
& \textstyle Y\subseteq \bigcup C_n\htam{.}
\end{aligned}
\right.
\end{equation*}
\end{psec}
%
%                  7.3
%
\begin{psec}{7.3}%
In a compact Hausdorff space~$X$,
meager sets are small in a sense illustrated by the following observations.
\begin{itemize}[itemindent=1.2em]
\item
Meager sets have empty interior;
in other words:
the complement of a meager set is dense.
(This is Baire's Theorem~\ref{7.1}.)
%
\item
The only meager open set is~$\varnothing$.
%
\item
A closed set is meager
if and only if its interior is empty.
%
\item
Subsets of meager sets are meager.
%
\item
A union of countably many meager sets is meager.
\end{itemize}

Meager sets are in topology
more or less what null sets are in measure theory.
\end{psec}
%
%                  7.4
%
\begin{psec}{7.4}{Examples}
In $[0,1]$,
every countable set is meager.

Let $(q_1,q_2,\dotsc)$ be an enumeration of $[0,1]\cap\mathbb{Q}$,
and for $\varepsilon>0$ let
\begin{equation*}
U_\varepsilon\ =\ [0,1]\,\cap\,{\bigcup}_n
  \bigl(q_n-\varepsilon\, 2^{-n},\, q_n +\varepsilon\, 2^{-n}\bigr)\htam{.}
\end{equation*}
$U_\varepsilon$ is a dense open subset of~$[0,1]$.

Then $[0,1]\,\backslash\, U_{\sfrac{1}{5}}$ is meager
but its Lebesgue measure is positive.
The set
\begin{equation*}
\bigcap_{p\in\N} U_{\sfrac{1}{p}}
\end{equation*}
has measure~$0$ but its complement is meager.
\end{psec}
%
%
%
\begin{psec}{7.5}{Exercise}
Let $X$ be a compact Hausdorff space
and let $(f_n)_n$ be a sequence in $\Cont{X}^+$.
Prove the equivalence of:
\begin{enumerate}[label=(\greek*)]
\item\label{7.5-alpha}
In $\Cont{X}$,
the set $\{ f_1,f_2,\dotsc \}$ has infimum~$0$.
%
\item\label{7.5-beta}
The set $\{x\in X\colon \inf_n f_n(x)>0\}$ is meager.
\end{enumerate}
Hint for \ref{7.5-alpha}$\implies$\ref{7.5-beta}:
For $\varepsilon>0$,
consider $X_\varepsilon :=\{x\colon \inf_n f_n(x)\geq\varepsilon\}$.

Now prove:
There is no sequence $(g_n)_n$ in $\Cont{[0,1]}^+$ with
\begin{equation*}
\inf_n g_n(x)\ =\ 0 \quad\iff\quad x\in[0,1]\cap\mathbb{Q}\htam{.}
\end{equation*}
\end{psec}
%
%                  7.6
%
\begin{psec}{7.6}{Definitions}
Let $X$ be a compact Hausdorff space.
For functions~$f$ and~$g$ on~$X$ we put
\begin{align*}
f&\equiv g \\
\shortintertext{%
if the set $\{x\colon f(x)\neq g(x)\}$ is meager.
For subsets~$A$ and~$B$ of~$X$
we put}
A&\equiv B
\end{align*}
if $\mathbb{1}_A\equiv\mathbb{1}_B$,
which is the case if and only if $A\backslash B$
and $B\backslash A$ are meager.

Note that both relations are equivalences.
\end{psec}
%
%                  7.7
%
\begin{psec}{7.7}{Exercise}
Let $X$ be a compact Hausdorff space.
Prove:
\begin{enumerate}
\item\label{7.7-1}
If $U\subseteq X$ is open,
then $U\equiv\overline U$.
%
\item\label{7.7-2}
If $A,B\subseteq X$ and $A\equiv B$,
then $X\backslash A\equiv X\backslash B$.
%
\item\label{7.7-3}
If $A_1,A_2,\dotsc,B_1,B_2,\dotsc\subseteq X$
and $A_n\equiv B_n\quad(n\in\N)$,
then $\bigcup A_n\equiv \bigcup B_n$.
%
\item\label{7.7-4}
If $U$ and $V$ are clopen subsets of~$X$
and $U\equiv V$,
then $U=V$.
(Observe, however, that,
if~$U$ and~$V$ are only open,
then $U\equiv V$ as soon as $\overline U = \overline V$.)
%
\item\label{7.7-5}
If $f_1,f_2,\dotsc,g_1,g_2,\dotsc$ are functions on~$X$
with $f_n\equiv g_n\quad (n\in\N)$,
and if both sequences~$(f_n)_n$ and~$(g_n)_n$
converge pointwise,
then $\lim f_n\equiv \lim g_n$.
%
\item\label{7.7-6}
If $f$ and $g$ are continuous functions on~$X$
and $f\equiv g$, then $f=g$.
\end{enumerate}
\end{psec}
%
%                  7.8
%
\begin{psec}{7.8}{Lemma}\statement{
Let $X$ be a compact Hausdorff space.}

\statement{The subsets $Y$  of $X$ with the property
\begin{equation}\label{eq7.8-1}
Y\equiv U\quad\htam{ for some open set }\ U
\end{equation}
form a $\sigma$-algebra containing all open sets.}

\statement{The condition~\eqref{eq7.8-1}
is equivalent to
\begin{equation}\label{eq7.8-2}
Y\equiv C\quad\htam{ for some closed set }\ C\htam{.}
\end{equation}}

\statement{If $X$ is extremally disconnected,
\eqref{eq7.8-1} and~\eqref{eq7.8-2}
are equivalent to
\begin{equation}\label{eq7.8-3}
Y\equiv W\quad \htam{ for some clopen set }\ W\htam{.}
\end{equation}}
\end{psec}
\begin{proof}
Let $\mathcal A$ be the collection
of all sets with Property~\eqref{eq7.8-1}.
Obviously, $\mathcal A$ contains~$\varnothing$
and all open sets.
If $Y_1,Y_2,\dotsc \in\mathcal A$
and if $U_1,U_2,\dotsc$ are open sets with $A_n\equiv U_n\quad(n\in\N)$,
then $\bigcup Y_n\equiv \bigcup U_n$~(\ref{7.7}\ref{7.7-3}),
so $\bigcup Y_n\in\mathcal A$.
If $Y\in\mathcal A$ and~$U$ is open,
$Y\equiv U$,
then $Y\equiv \overline U$~(\ref{7.7}\ref{7.7-1}),
so that $X\backslash Y\equiv X\backslash\overline U$~(\ref{7.7}\ref{7.7-2}),
and $X\backslash Y\in\mathcal A$.

Thus, $\mathcal A$ is a $\sigma$-algebra.
A set~$Y$ has Property~\eqref{eq7.8-2}
if and only if $X\backslash Y\in\mathcal A$,
if and only if $Y\in\mathcal A$.
Thus, \eqref{eq7.8-1} and~\eqref{eq7.8-2} are equivalent.

If $X$ is extremally disconnected,
then~\eqref{eq7.8-1} implies~\eqref{eq7.8-3}
because for any open set~$U$
we have $U\equiv\overline U$ and~$\overline U$ is clopen. \xqed
\end{proof}
%
%                  7.9
%
\begin{psec}{7.9}{Definition}
Let $X$ be a compact Hausdorff space.
We say that a subset~$Y$ of~$X$ is \keyword{of the first Baire class}
if~$\mathbb{1}_A$ is the (pointwise) limit of a sequence
of continuous functions.
These sets form a collection
\begin{equation*}
\BaireI(X)
\end{equation*}
that is easily seen to be an algebra.
\end{psec}
%
%                  7.10
%
\begin{psec}{7.10}{Comments}
\begin{enumerate}
\item\label{7.10-1}
If $f\in\Cont{X}$,
then the set $\{x\colon f(x)>0\}$ is of the first Baire class,
as its indicator is $\lim \smash{\sqrt[n]{f^+}}$.
Therefore,
if $f\in\Cont{X}$,
sets such as $\{x\colon f(x)>3\}$,
$\{x\colon f(x)\leq 3 \}$,
$\{x\colon 2\leq f(x)\leq 3\}$ lie in~$\BaireI(X)$.
%
\item\label{7.10-2}
It follows that
\begin{equation*}
\Cont{X}\ \subseteq\ \BLeb(\BaireI(X))\htam{.}
\end{equation*}
Indeed,
let $f\in\Cont{X}$, $\varepsilon>0$.
Choose $s_0<s_1<\dotsb<s_N$ in~$\R$
with $s_n<s_{n-1}+\varepsilon$
for $n=1,\dotsc,N$
and such that all values of~$f$ lie in the interval $(s_0,s_N)$.
Setting $A_n:=\{x\colon s_{n-1}<f(x)\leq s_n\}$
we obtain $A_1,\dotsc, A_N$ in~$\BaireI(X)$,
pairwise disjoint, and with
\begin{equation*}
0\ \leq\ f-{\textstyle\sum}\,
 s_n\,\mathbb{1}_{A_n}\ \leq\ \varepsilon\,\mathbb{1}\htam{.}
\end{equation*}
%
\item\label{7.10-3}
In the case $X$ is metrizable,
one can show that every open set is of the form $\{x\colon f(x)>0\}$
for some~$f$ in~$\Cont{X}$,
so all open sets are of the first Baire class.
For, e.g., the space $[0,1]^\R$ this is not true.
%
\item\label{7.10-4}
Let $A\in\BaireI(X)$.
Choose $f_1,f_2,\dotsc$ in~$\Cont{X}$
with $f_n\ra \mathbb{1}_A$.
Then an element~$x$ of~$X$ belongs to~$A$
if and only if there is an~$N$ with
\begin{equation*}
\bigl\{ f_N(x),\, f_{N+1}(x),\, \dotsc\,\bigr\}
\ \subseteq\ \bigl(\textstyle\frac{1}{2},\infty\bigr)\htam{.}
\end{equation*}
Hence,
\begin{equation*}
A\ =\ \bigcup_N \bigcap_{n\geq N} \bigl\{ 
x\colon f_n(x)>\textstyle \frac{1}{2} \bigr\}\htam{.}
\end{equation*}
As $\{x\colon f_n(x)>\sfrac{1}{2}\}$ is always open,
Lemma~\ref{7.8} implies:
\statement{If~$A$ is a set of the first Baire class
in an extremally disconnected compact Hausdorff space,
then there is a unique clopen set~$U$ with $A\equiv U$.}
\end{enumerate}
\end{psec}
%
%                  7.11
%
\begin{psec}{7.11}%
Next,
we extend the integration theory of~\ref{6.10}.
Instead of an additive $\mu\colon \mathcal A\ra\R^+$
we now consider an additive $\mu\colon \mathcal A\ra E^+$
where~$E$ is a suitable Riesz space:

\vspace{.5em}
\noindent\textbf{Theorem}\  \statement{
Let $\mathcal A$ be an algebra of subsets of a set~$X$.
Let~$E$ be a uniformly complete
unitary Riesz space,
and let~$\mu$ be an additive map $\mathcal A\ra E^+$.
Then there is a unique linear map
\begin{equation*}
f\mapsto \int f\,\mathrm{d}\mu
\end{equation*}
of $\BLeb(\mathcal A)$ into~$E$ with the properties
\begin{equation*}
\left\{\ 
\begin{aligned}
\int \mathbb{1}_A\,\mathrm{d}\mu\ &=\ \mu(A)
  \qquad& (A\in\mathcal A)\htam{,} \\
\int f\,\mathrm{d}\mu \ &\geq\ 0
  \qquad & (f\geq 0)\htam{.}
\end{aligned}
\right.
\end{equation*}}
\end{psec}
\begin{proof}
Let~$e$ be a unit for~$E$;
then under the norm~$\|\cdot\|_e$,
$E$ is complete.
The proof is now nothing but aping our proof of Theorem~\ref{6.10},
replacing~$\R$ by the complete metric space~$E$. \xqed
\end{proof}
%
%                  7.11
%
\begin{psec}{7.12}{Exercise}
The functions~$f$ on~$\N$
for which $\lim_{n\ra \infty} f(n)$ exists (in~$\R$)
form a Riesz space~$\cseq$.
(See~\ref{1.5}\ref{1.5-4}, \ref{4.20}.)
Let~$\mathcal A$ be the collection of all subsets of~$\N$
that either are finite or have a finite complement.
\begin{enumerate}
\item\label{7.12-1}
Show that~$\mathcal A$ is an algebra
and that $\BLeb(\mathcal A)=\cseq$.
%
\item\label{7.12-2}
Define $\mu\colon\mathcal A\ra\R$ by
\begin{equation*}
\mu(A)\ :=\ \sum_{n\in A} \,2^{-n}\htam{.}
\end{equation*}
``Describe'' (more directly than by quoting the definition)
$\int \cdot\,\mathrm{d}\mu$.
%
\item\label{7.12-3}
Define $\pi\colon\mathcal A\ra\R$ by
\begin{alignat*}{2}
\pi(A)&:=0 \qquad&&\htam{if }A\htam{ is finite,} \\
\pi(A)&:=1 \qquad&&\htam{if }\N\backslash A\htam{ is finite.}
\end{alignat*}
``Describe'' $\int\cdot\,\mathrm{d}\pi$.
%
\item\label{7.12-4}
Define $\tau\colon \mathcal A\ra\ell^\infty$ ($:=\ell^\infty(\N)$) by
\begin{equation*}
\tau(A)\ :=\ \mathbb{1}_A\htam{.}
\end{equation*}
``Describe'' $\int\cdot\,\mathrm{d}\tau$.
\end{enumerate}
\end{psec}
%
%                  7.13
%
We return to the Hermitian operators.
In the following pages
our notation are as in Chapter~\ref{5}.
\begin{psec}{7.13}{Theorem}\statement{
Every strongly closed subalgebra of~$\mathscr H$
is a Dedekind complete Riesz space.}
(See~\ref{5.17} for the definition of ``strongly closed''.)
\end{psec}
\begin{proof}
Let $\mathscr A$ be a strongly closed subalgebra of~$\mathscr H$.
Being closed in the sense of the norm,
$\mathscr A$ is a Riesz space (Lemma~\ref{5.25}).
We show Dedekind completeness with the aid of~\ref{4.17}\ref{4.17-2}:

Let $\mathscr T$ be a nonempty subset of~${\mathscr A}^+$,
upwards directed,
with an upper bound~$A$ in~${\mathscr A}^+$.
Without loss of generality we assume~$\|A\|\leq 1$;
then $A\leq I$ (Lemma~\ref{5.10})
so that~$I$ is an upper bound for~$\mathscr T$.
(Possibly, $I\notin \mathscr A$,
but that turns out to be irrelevant.)

$\mathscr T$ is a directed set. For $x\in H$,
\begin{equation*}
\bigl(\left< Tx, x\right>\bigr)_{T\in\mathscr T}
\end{equation*}
is an increasing net in~$\R$,
bounded from above by~$\left<x,x\right>$,
and therefore convergent to some number, $\alpha(x)$, say:
\begin{equation*}
\alpha(x)\ =\ \lim_{T\in\mathscr T} \left<Tx,x\right>
\ =\ \sup_{T\in\mathscr T} \left<Tx,x\right>
\qquad(x\in H)\htam{.}
\end{equation*}

If $S,T\in\mathscr T$ and $S\geq T$,
then $0\leq S-T\leq I-0=I$,
so that (\ref{5.20}\ref{5.20-2})
\begin{equation*}
(S-T)^2\ \leq\ S-T\htam{.}
\end{equation*}
Then for all~$x$ in~$H$
\begin{alignat*}{2}
\bigl\|Sx - Tx\bigr\|^2\ 
  &=\ \bigl<(S-T)x,(S-T)x\bigr> \\
  &=\ \bigl<(S-T)^2x,x\bigr> \\
  &=\ \bigl<(S-T)x,x\bigr> \\
  &=\ \bigl<Sx,x\bigr> - \bigl<Tx,x\bigr> \\
  &\leq\ \alpha(x) - \bigl<Tx,x\bigr>\htam{.}
\end{alignat*}
It follows without too much trouble
that the net $(Tx)_{T\in\mathscr T}$ converges:
Choose $T_1\leq T_2\leq \dotsb$ in~$\mathscr T$
with $\left<T_nx,x\right>\geq\alpha(x)-n^{-2}$ for all~$n$; then
\begin{equation*}
\| Sx-T_nx\|\ \leq\ n^{-1}
\qquad(S\in\mathscr T,\ n\in\N,\ S\geq T_n)\htam{.}
\end{equation*}
In particular, $(T_nx)_n$ is a Cauchy sequence.
If $x^*$ is its limit,
we have $\|x^* - T_nx\|\leq n^{-1}$ for all~$n$,
so that $\|Tx-x^*\|\leq
\|Tx-T_nx\| + \|x^*-T_nx\|\leq 2n^{-1}$
as soon as $T\geq T_n$:
Thus, $x^* = \lim_{T\in\mathscr T} Tx$.

Consequently,
we can define $T_\infty\colon H\ra H$ by
\begin{equation*}
T_\infty x\ =\ \lim_{T\in \mathscr T} Tx\qquad (x\in H)\htam{.}
\end{equation*}

We have $T\leq I\quad(T\in\mathscr H)$,
hence, by Lemma~\ref{5.10}, $\|Tx\|\leq\|x\|\quad(T\in\mathscr T,\ x\in H)$,
hence $\|T_\infty x\|\leq\|x\|$.
Then~$T_\infty$ is a continuous linear map.
As $\left<Tx,y\right>=\left<x,Ty\right>\quad(T\in\mathscr T; x,y\in H)$,
we also have $\left<T_\infty x,y\right> = \left<x,T_\infty y\right>$
for all~$x$ and~$y$, so $T_\infty\in\mathscr H$.
Thanks to the strong closedness of~$\mathscr A$ in~$\mathscr H$, \ 
$T_\infty\in\mathscr A$, and the formula
\begin{equation*}
\left<T_\infty x,x\right>\ 
=\ \sup_{T\in\mathscr T} \left<Tx,x\right>\qquad(x\in H)
\end{equation*}
implies $T_\infty = \sup \mathscr T$. \xqed
\end{proof}
%
%                  7.14, Spectral Theorem for Hermitian operators
%
\begin{psec}{7.14}{Spectral Theorem for Hermitian operators}\statement{
Let $A$ be a Hermitian operator in~$H$.
Then there exists an additive
$P\colon \BaireI(\Sp{A})\ra(A^{\square\square})^+$
with the properties
\begin{equation*}
\left\{\quad 
\begin{aligned}
f(A)\ &=\ \smash{\int} f\,\mathrm{d}P\qquad&&(f\in\Cont{\Sp{A}}\,)\htam{,} \\
P(Y)\ &\htam{\,is \  a projection }
  \qquad&&(Y\in\BaireI(\Sp{A})\,)\htam{,}\\
P(\Sp{A})\ &=\ I\htam{,} \\
P(Y)\,P(Z)\ &=\ P(Y\cap Z)\qquad&&(Y,Z\in\BaireI(\Sp{A})\,)\htam{.}
\end{aligned}
\right.
\end{equation*}
}\end{psec}
\begin{proof}
\begin{enumerate}[label=(\Roman*)]
\item\label{7.14-I}
With $A$ we have associated two closed subalgebras of~$\mathscr H$,
viz. $\smash{\overline{\R[A]}}$ (see~\ref{5.32})
and $A^{\square\square}$ ($:=\smash{\bigl(\{A\}^\square\bigr)^\square}$;
see~\ref{5.16}\ref{5.16-5}).
We have a Riesz isomorphism $f\mapsto f(A)$
of~$\Cont{\sigma_A}$ onto $\smash{\overline{\R[A]}}$ (\ref{5.33}),
and a Riesz isomorphism $T\mapsto \hat{T}$ of~$A^{\square\square}$
onto~$\Cont{\Phi}$
for some extremally disconnected compact Hausdorff space~$\Cont{\Phi}$
(\ref{5.17},\ref{6.5}, \ref{7.13}).
\begin{equation*}
\xymatrix@=4em{
\overline{\R[A]}
  \ar@{}[r]|*{\subseteq} & 
A^{\square\square}
  \ar[d]^{T\mapsto \hat{T}}\\
\Cont{\Sp{A}}
  \ar[r]_{f\mapsto f\circ \omega} 
  \ar[u]^{f\mapsto f(A)}&
\Cont{\Phi}
}
\end{equation*}
The inclusion $\smash{\overline{\R[A]}}\subseteq A^{\square\square}$
induces a Riesz homomorphism
\begin{equation*}
f\mapsto \widehat{f(A)}
\end{equation*}
of~$\Cont{\Sp{A}}$ into~$\Cont{\Phi}$,
with $\mathbb{1}_{\Sp{A}}\mapsto\mathbb{1}_\Phi$.

According to Exercise~\ref{7.15},
there is a continuous $\omega\colon \Phi\ra\Sp{A}$
with
\begin{equation*}
\widehat{f(A)}\ =\ f\circ \omega\qquad (f\in \Cont{\Sp{A}}\,)\htam{.}
\end{equation*}
%
\item\label{7.14-II}
If $Y\in\BaireI(\Sp{A})$,
there exists a sequence $(f_n)_n$ in~$\Cont{\Sp{A}}$
such that $f_n \ra\mathbb{1}_Y$
pointwise on~$\Sp{A}$.
Then the sequence $(f_n\circ\omega)_n$
converges to $\mathbb{1}_Y\circ\omega$
--- which is $\mathbb{1}_{\omega^{-1}(Y)}$ ---
pointwise on~$\Phi$.
Thus,
$\omega^{-1}(Y)\in\BaireI(\Phi)$.
Then, by~\ref{7.10}\ref{7.10-4}
there is a clopen set $U\subseteq \Phi$
with $\omega^{-1}(Y)\equiv U$.
As $\mathbb{1}_U\in\Cont{\Phi}$,
there is an element~$P(Y)$ of~$A^{\square\square}$
with $\mathbb{1}_U=\smash{\widehat{P(Y)}}$:
\begin{equation*}
\label{eq7.14-II}\tag{$*$}
\mathbb{1}_Y\circ\omega\ \equiv\ \widehat{P(Y)}\htam{.}
\end{equation*}
($P(Y)$ is determined by this relation:
there is only one $T\in A^{\square\square}$ with
$\mathbb{1}_Y\circ\omega=\hat{T}$.)
%
\item\label{7.14-III}
We obtain a map $P\colon \BaireI(\Sp{A})\ra A^{\square\square}$
satisfying~\eqref{eq7.14-II} for every~$Y$.
Clearly this~$P$ is additive,
and $P(Y)\,P(Z)=P(Y\cap Z)$ if $Y,Z\in\BaireI(\Sp{A})$.

In particular,
$P(Y)^2=P(Y)$,
so each~$P(Y)$ is a projection~(\ref{5.7}).

Of course, $P(\sigma_A)=I$.
%
\item\label{7.14-IV}
Take~$f$ in~$\Cont{\Sp{A}}$.
By~\ref{7.10}\ref{7.10-4},
$\int f\,\mathrm{d}P$ exists;
we prove that it equals~$f(A)$.

Let $\varepsilon>0$.
Following~\ref{7.10}\ref{7.10-2},
choose $N\in\N$,
$s_1,\dotsc,s_N\in\R$
and $Y_1,\dotsc,Y_N\in\BaireI(\Sp{A})$ so that
\begin{equation}
\label{eq7.14-IV-*}\tag{$**$}
0\ \leq\ f-{\textstyle\sum}\,s_n\,\mathbb{1}_{Y_n}
\ \leq\ \varepsilon\,\mathbb{1}_{\Sp{A}}\qquad(\htam{on }\Sp{A})\htam{.}
\end{equation}

On the one hand,
integration with respect to~$P$ yields
\begin{equation}
\label{eq7.14-IV-1}\tag{$1$}
0\ \leq\ {\textstyle\int}f\,\mathrm{d}P - 
{\textstyle\sum}\,s_n\,P(Y_n)\ \leq\ \varepsilon\,I\htam{.}
\end{equation}

On the other,
\eqref{eq7.14-IV-*}~implies
\begin{equation*}
0\ \leq\ f\circ\omega -{\textstyle\sum}\,s_n\,\mathbb{1}_{Y_n}\circ\omega
\ \leq\ \varepsilon\,\mathbb{1}_\Phi\qquad(\htam{on }\Phi)\htam{.}
\end{equation*}
But
\begin{equation*}
f\circ\omega - {\textstyle\sum}\,s_n\,\mathbb{1}_{Y_n}\circ\omega
\ \equiv\ \widehat{f(A)} - {\textstyle\sum}\,s_n\,\widehat{P(Y_n)}\htam{.}
\end{equation*}
The right hand member denoting a continuous function,
we obtain
\begin{equation*}
0\ \leq\ \widehat{f(A)} - {\textstyle\sum}\,s_n\,\widehat{P(Y_n)}
\ \leq\ \varepsilon\,\mathbb{1}_\Phi\qquad(\htam{on }\Phi)\htam{.}
\end{equation*}
whence
\begin{equation}
\label{eq7.14-IV-2}\tag{$2$}
0\ \leq\ f(A)-{\textstyle\sum}\,s_n\,P(Y_n)\ \leq\ \varepsilon\,I
\qquad(\htam{in }A^{\square\square})\htam{.}
\end{equation}

It follows from~\eqref{eq7.14-IV-1} and~\eqref{eq7.14-IV-2} that
\begin{equation*}
-\varepsilon\,I\ \leq\ f(A)-{\textstyle\int}f\,\mathrm{d}P
\ \leq\ \varepsilon\,I\htam{.}
\end{equation*}

As this inequality holds for every~$\varepsilon$,
we have: $f(A)=\int f\,\mathrm{d}P$. \xqed
\end{enumerate}
\end{proof}
%
%                  7.15
%
\begin{psec}{7.15}{Exercise}
Let $X$ and~$Z$ be compact Hausdorff spaces
and~$T$ a Riesz homomorphism $\Cont{X}\ra\Cont{Z}$,
sending~$\mathbb{1}_X$ to~$\mathbb{1}_Z$.
Prove that there is a continuous $\tau\colon Z\ra X$
such that
\begin{equation*}
Tf\ =\ f\circ\tau\qquad (f\in\Cont{X}\,)\htam{.}
\end{equation*}
\end{psec}
%
%                  7.16
%
\begin{psec}{7.16}{Comment}
Suppose $\dim H<\infty$.
Then $\Sp{A}$ consists of the eigenvalues of~$A$~(\ref{5.36}),
and is a finite set $\{s_1,\dotsc,s_N\}\quad(s_n\neq s_m\htam{ if }n\neq m)$.
It follows that~$\BaireI(\Sp{A})$ is
the algebra of all subsets of~$\Sp{A}$.
In the terminology of~\ref{7.14},
put $P_n:=P(\{s_n\})\quad(n=1,\dotsc,N)$.
Then $P_1,\dotsc,P_N$ are projections
on mutually orthogonal linear subspaces of~$H$, and
\begin{align*}
I\ &=\ {\textstyle\sum}\,P_n &
A\ &=\ {\textstyle\sum}\,s_n\,P_n\htam{.}
\end{align*}
\end{psec}
%
%                  7.17
%
\begin{psec}{7.17}{Exercise}
Prove the following weak version of the Spectral Theorem:
If $A\in\mathscr H$
and $\varepsilon\in(0,\infty)$,
there exist $N\in\N$,
$s_1,\dotsc,s_N\in\R$
and closed linear subspaces
$H_1,\dotsc,H_N$ of~$H$ such that
\begin{equation*}
\left\{\ 
\begin{aligned}
&H_n\perp H_m \htam{ whenever }n\neq m\htam{; }H_1+\dotsb+H_N=H\htam{;} \\
&\htam{ if }x\in H_n\htam{, then }Ax\in H_n\htam{ and }
\|Ax -s_nx\|\leq\varepsilon\,\|x\|\htam{.}
\end{aligned}
\right.
\end{equation*}
\end{psec}
%
%
%
\clearpage
\end{document}
