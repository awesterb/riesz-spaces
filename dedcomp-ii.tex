\documentclass[main.tex]{subfiles}
\begin{document}
% 6
\section{Dedekind Completeness, II}
Another application concerns an extension of the
representation theory for Hermitian operators.
First, some topology.
%
%                  7.1,  Baire Category Theorem
%
\begin{psec}{7.1}{Baire Category Theorem}\statement{
Let $X$ be either a compact Hausdorff space
or a complete metric space.
\begin{enumerate}
\item\label{7.1-1}
If $U_1, U_2,\dotsc$ are dense open sets,
then $\bigcap U_n$ is dense.
%
\item\label{7.1-2}
If $C_1,C_2,\dotsc$ are closed sets with empty interior,
then $\bigcup C_n$ has empty interior.
%
\item\label{7.1-3}
If $C_1,C_2,\dotsc$ are closed sets and $\bigcup C_n=X$,
then at least one~$C_n$ has nonempty interior
---
provided~$X$ is nonempty.
\end{enumerate}
}\end{psec}
\begin{proof}
For $Y\subseteq X$
the statements ``$Y$ is dense''
and ``$X\backslash Y$ has empty interior''
are equivalent.
Consequently,
we have \ref{7.1-1}$\iff$\ref{7.1-2}$\implies$\ref{7.1-3},
and it suffices to prove~\ref{7.1-1}.
Let $U_1,U_2,\dotsc$ be dense open sets
and let~$W$ be open, nonempty;
we wish to show that $W\cap \bigcap U_n\neq \varnothing$.

As $U_1$ is open and dense, $W\cap U_1$
is a nonempty open set.
Then there is a nonempty open~$V_1$
with $\overline{V}_1\subseteq W\cap U_1$.
Similarly,
$V_1\cap U_2$ is nonempty and open:
Choose a nonempty open~$V_2$ with $\overline{V}_2\subseteq V_1\cap U_2$.
Continuing in this fashion
one obtains the following scheme:
\begin{equation*}
\xymatrix@-1em{
W 
  \ar@{}[r]|*{\supseteq}&
\overline{V}_1 
  \ar@{}[r]|*{\supseteq}&
V_1
 \ar@{}[r]|*{\supseteq}&
\overline{V}_2 
 \ar@{}[r]|*{\supseteq}  &
V_2
  \ar@{}[r]|*{\supseteq} &
\overline{V}_3
  \ar@{}[r]|*{\supseteq}  &
\dotsb \\
&
U_1
 \ar@{}[u]|*{\begin{turn}{90}$\supseteq$\end{turn}} &
&
U_2 \ar@{}[u]|*{\begin{turn}{90}$\supseteq$\end{turn}} &
&
U_3 \ar@{}[u]|*{\begin{turn}{90}$\supseteq$\end{turn}} &
}\end{equation*}
Now $\bigcap \overline{V}_n \subseteq W\cap \bigcap U_n$;
and we are done in case $\bigcap \overline{V}_n \neq \varnothing$.

If $X$ is compact,
this is automatic.
If~$X$ is a complete metric space,
we insert one clause in the above construction,
viz., that~$V_n$ be a ball with radius~$\leq n^{-1}$.
\xqed
\end{proof}
%
%                  7.2
%
\begin{psec}{7.2}{Definition}
Let $X$ be a compact Hausdorff space.

A subset~$Y$ of~$X$ is called \keyword{meager}
if there exist subsets $C_1,C_2,\dotsc$ of~$X$ such that
\begin{equation*}
\left\{\ 
\begin{aligned}
& \htam{every }C_n\htam{ is closed and has empty interior,} \\
& \textstyle Y\subseteq \bigcup C_n\htam{.}
\end{aligned}
\right.
\end{equation*}
\end{psec}
%
%                  7.3
%
\begin{psec}{7.3}%
In a compact Hausdorff space~$X$,
meager sets are small in a sense illustrated by the following observations.
\begin{itemize}[itemindent=1.2em]
\item
Meager sets have empty interior;
in other words:
the complement of a meager set is dense.
(This is Baire's Theorem~\ref{7.1}.)
%
\item
The only meager open set is~$\varnothing$.
%
\item
A closed set is meager
if and only if its interior is empty.
%
\item
Subsets of meager sets are meager.
%
\item
A union of countably many meager sets is meager.
\end{itemize}

Meager sets are in topology
more or less what null sets are in measure theory.
\end{psec}
%
%                  7.4
%
\begin{psec}{7.4}{Examples}
In $[0,1]$,
every countable set is meager.

Let $(q_1,q_2,\dotsc)$ be an enumeration of $[0,1]\cap\mathbb{Q}$,
and for $\varepsilon>0$ let
\begin{equation*}
U_\varepsilon\ =\ [0,1]\,\cap\,{\bigcup}_n
  \bigl(q_n-\varepsilon\, 2^{-n},\, q_n +\varepsilon\, 2^{-n}\bigr)\htam{.}
\end{equation*}
$U_\varepsilon$ is a dense open subset of~$[0,1]$.

Then $[0,1]\,\backslash\, U_{\sfrac{1}{5}}$ is meager
but its Lebesgue measure is positive.
The set
\begin{equation*}
\bigcap_{p\in\N} U_{\sfrac{1}{p}}
\end{equation*}
has measure~$0$ but its complement is meager.
\end{psec}
%
%
%
\begin{psec}{7.5}{Exercise}
Let $X$ be a compact Hausdorff space
and let $(f_n)_n$ be a sequence in $\Cont{X}^+$.
Prove the equivalence of:
\begin{enumerate}[label=(\greek*)]
\item\label{7.5-alpha}
In $\Cont{X}$,
the set $\{ f_1,f_2,\dotsc \}$ has infimum~$0$.
%
\item\label{7.5-beta}
The set $\{x\in X\colon \inf_n f_n(x)>0\}$ is meager.
\end{enumerate}
Hint for \ref{7.5-alpha}$\implies$\ref{7.5-beta}:
For $\varepsilon>0$,
consider $X_\varepsilon :=\{x\colon \inf_n f_n(x)\geq\varepsilon\}$.

Now prove:
There is no sequence $(g_n)_n$ in $\Cont{[0,1]}^+$ with
\begin{equation*}
\inf_n g_n(x)\ =\ 0 \quad\iff\quad x\in[0,1]\cap\mathbb{Q}\htam{.}
\end{equation*}
\end{psec}
%
%
%
\clearpage
\end{document}
