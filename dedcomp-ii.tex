\documentclass[main.tex]{subfiles}
\begin{document}
% 6
\section{Dedekind Completeness, II}
Another application concerns an extension of the
representation theory for Hermitian operators.
First, some topology.
%
%                  7.1,  Baire Category Theorem
%
\begin{psec}{7.1}{Baire Category Theorem}\statement{
Let $X$ be either a compact Hausdorff space
or a complete metric space.
\begin{enumerate}
\item\label{7.1-1}
If $U_1, U_2,\dotsc$ are dense open sets,
then $\bigcap U_n$ is dense.
%
\item\label{7.1-2}
If $C_1,C_2,\dotsc$ are closed sets with empty interior,
then $\bigcup C_n$ has empty interior.
%
\item\label{7.1-3}
If $C_1,C_2,\dotsc$ are closed sets and $\bigcup C_n=X$,
then at least one~$C_n$ has nonempty interior
---
provided~$X$ is nonempty.
\end{enumerate}
}\end{psec}
\begin{proof}
For $Y\subseteq X$
the statements ``$Y$ is dense''
and ``$X\backslash Y$ has empty interior''
are equivalent.
Consequently,
we have \ref{7.1-1}$\iff$\ref{7.1-2}$\implies$\ref{7.1-3},
and it suffices to prove~\ref{7.1-1}.
Let $U_1,U_2,\dotsc$ be dense open sets
and let~$W$ be open, nonempty;
we wish to show that $W\cap \bigcap U_n\neq \varnothing$.

As $U_1$ is open and dense, $W\cap U_1$
is a nonempty open set.
Then there is a nonempty open~$V_1$
with $\overline{V}_1\subseteq W\cap U_1$.
Similarly,
$V_1\cap U_2$ is nonempty and open:
Choose a nonempty open~$V_2$ with $\overline{V}_2\subseteq V_1\cap U_2$.
Continuing in this fashion
one obtains the following scheme:
\begin{equation*}
\xymatrix@-1em{
W 
  \ar@{}[r]|*{\supseteq}&
\overline{V}_1 
  \ar@{}[r]|*{\supseteq}&
V_1
 \ar@{}[r]|*{\supseteq}&
\overline{V}_2 
 \ar@{}[r]|*{\supseteq}  &
V_2
  \ar@{}[r]|*{\supseteq} &
\overline{V}_3
  \ar@{}[r]|*{\supseteq}  &
\dotsb \\
&
U_1
 \ar@{}[u]|*{\begin{turn}{90}$\supseteq$\end{turn}} &
&
U_2 \ar@{}[u]|*{\begin{turn}{90}$\supseteq$\end{turn}} &
&
U_3 \ar@{}[u]|*{\begin{turn}{90}$\supseteq$\end{turn}} &
}\end{equation*}
Now $\bigcap \overline{V}_n \subseteq W\cap \bigcap U_n$;
and we are done in case $\bigcap \overline{V}_n \neq \varnothing$.

If $X$ is compact,
this is automatic.
If~$X$ is a complete metric space,
we insert one clause in the above construction,
viz., that~$V_n$ be a ball with radius~$\leq n^{-1}$.
\xqed
\end{proof}
%
%
%
\clearpage
\end{document}
