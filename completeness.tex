\documentclass[main.tex]{subfiles}
\begin{document}
% 4
\section{Completeness}
After Yosida's Theorem~\ref{3.14},
a reasonable question is:
When is the map~$E\ra \Cont{\Phi}$ surjective?
%
%                  4.1
%
\begin{psec}{4.1}{Exercise}
Let $E$ be an Archimedean Riesz space with a unit, $e$.
\begin{enumerate}
\item \label{4.1-1}
Let $u\in E^+$ 
and let $E_u$ denote the principal ideal of~$E$ 
generated by~$u$ (\ref{2.11}). 
Then~$E_u$ is Archimedean and has~$u$ as a unit.
Show that
\begin{equation*}
\| x \|_e \leq \| u \|_e \, \| x \|_u \qquad (x\in E)
\end{equation*}
so that the identity map $E_u \ra E$
is continuous relative to~$\|\cdot\|_u$ and~$\|\cdot \|_e$.
%
\item \label{4.1-2}
Now let~$u$ be a unit of~$E$.
Show that~$\|\cdot\|_e$ and~$\|\cdot\|_u$
determine the same topology on~$E$,
and, less trivially:
$E$ is complete under (the metric induced by) 
the norm~$\|\cdot\|_u$
if and only if it is complete under~$\|\cdot\|_e$.
\end{enumerate}
\end{psec}
%
%                  4.2
%
\begin{psec}{4.2}{Definitions}
Let $E$ be an Archimedean unitary Riesz space.
The topology induced by~$\|\cdot\|_e$,
where~$e$ is any unit,
is called the \keyword{(relatively) uniform topology}.
$E$ is said to be
\keyword{uniformly complete}
if it is complete relative to~$\|\cdot\|_e$
where~$e$ is any unit.

We use terms such as ``relatively uniformly closed''
and ``relatively uniformly convergent''
with the obvious meanings.
For instance,
if~$e$ is a unit in~$E$,
a sequence $(x_n)_n$
``converges relatively uniformly''
to an element $a$ 
if and only if for every $\varepsilon>0$
there is an~$N$ with
\begin{equation*}
|x_n - a| \leq \varepsilon e \qquad (n\geq N)\htam{.}
\end{equation*}

In spaces such as~$\ell^\infty$
and $\BCont{X}$
the constant function $\mathbb{1}$
serves as a unit.
As~$\|\cdot\|_\mathbb{1}$ coincides
with the sup-norm,
\emph{in these spaces,
relatively uniform convergence is just uniform convergence.}

Accordingly,
the qualification ``relatively'' is often dropped.
(However,
if~$E$ is the space of 
all scalar multiples of the function 
$i\colon x \mapsto x\quad (x\in [0,\infty)\,)$,
then the sequence $(n^{-1}i)_n$
converges to~$0$
relatively uniformly,
but not uniformly.
\end{psec}
%
%                  4.3
%
\begin{psec}{4.3}{Exercise}
Let $E$ be an Archimedean unitary Riesz space.
Show that,
with respect to the uniform topology,
the map $x\mapsto|x|$
is continuous
and every ``interval'' $[a,b]$ is closed.
\end{psec}
%
%                  4.4
%
\begin{psec}{4.4}{Exercise}
Show that in the Riesz space $\Cont{[0,1]}$
the principal Riesz ideal
generated by the function $t\mapsto t$
is not uniformly closed in $\Cont{[0,1]}$.
\end{psec}
%
%                  4.5
%
\begin{psec}{4.5}{Theorem}
(Continuation of~\ref{3.41})
\statement{%
If~$E$ is uniformly complete,
then~$\hat{E}=\Cont{\Phi}$.
Thus:
Every uniformly complete (Archimedean, unitary) Riesz space
is Riesz isomorphic to $\Cont{\Phi}$
for some compact Hausdorff space~$\Phi$.
}
\end{psec}
\begin{proof}
$E$ is complete relative to~$\|\cdot\|_e$.
Hence, $\hat{E}$ is complete relative to~$\|\cdot\|_{\hat{e}}$,
which is~$\|\cdot\|_\infty$.
Then~$\hat{E}$ is closed in~$\Cont{\Phi}$
(relative to~$\|\cdot\|_\infty$).
But it is also dense,
so~$\hat{E}$ equals~$\Cont{\Phi}$. \xqed
\end{proof}
%
%                  4.6
%
\begin{psec}{4.6}{Examples}
\begin{enumerate}
\item \label{4.6-1}
For every set~$X$
the space $\ell^\infty(X)$ (\ref{1.5}\ref{1.5-2})
is uniformly complete.
(A brief proof:
Let $f_1,f_2,\dotsc\in \ell^\infty(X)$
and let $(\varepsilon_n)_n$ be a sequence in~$(0,\infty)$,
tending to~$0$ and with
\begin{equation*}
\|f_n-f_N\|_\infty \leq \varepsilon_N \qquad (n\geq N)\htam{.}
\end{equation*}
For every $x\in X$ we have $|f_n(x)-f_N(x)|\leq\varepsilon_N\quad (n\geq N)$,
so that $(f_n(x))_n$ is a Cauchy sequence in~$\R$;
let~$f(x)$ be its limit.
Thus we obtain a function~$f$.
For all~$x$,
$|f(x)-f_1(x)|=\lim_{n\ra \infty} |f_n(x)-f_1(x)|\leq \varepsilon_1$,
so~$f$ is bounded: $f\in \ell^\infty(X)$.
Take $N\in\N$;
then for all~$x$,
$|f(x)-f_N(x)|
=\lim_{n\ra\infty}|f_n(x)-f_N(x)|
\leq\varepsilon_N$.
Thus, $\| f-f_N \|_\infty \leq \varepsilon_N \ra 0$.)
%
\item \label{4.6-2}
If $X$ is a topological space,
$\BCont{X}$ is a $\|\cdot\|_\infty$-closed subset of~$\ell^\infty(X)$,
hence is uniformly complete.
(Proof of the closedness:
Let $f\in \ell^\infty(X)$ lie in the closure of~$\BCont{X}$.
Let $x_i\ra a$ in~$X$.
Question:
must $f(x_i)\ra f(a)$?
Take $\varepsilon>0$.
Question:
Is there an~$i_0$
such that $|f(x_i)-f(a)|<\varepsilon\quad(i\geq i_0)$?
Choose $g\in\BCont{X}$ 
with $\|f-g\|_\infty\leq \sfrac{\varepsilon}{3}$.
There is an~$i_0$ such that $|g(x_i)-g(a)|<\sfrac{\varepsilon}{3}$
for all $i\geq i_0$.
For such~$i$ we have
$|f(x_i)-f(a)|
\leq|f(x_i)-g(x_i)| + |g(x_i)-g(a)| + |g(a)-f(a)|
\leq \varepsilon$.)
%
\item \label{4.6-3}
In particular,
of course,
the~$\Cont{\Phi}$ of Yosida's Theorem is uniformly complete.
%
\item \label{4.6-4}
More examples follow later on in this chapter.
\end{enumerate}
\end{psec}
%
%                  4.7
%
\begin{psec}{4.7}{Lemma}
\statement{%
Let $E$ be a unitary Archimedean Riesz space.
If~$E$ is uniformly complete,
then so is every principal ideal of~$E$.
}
\end{psec}
\begin{proof}
By Theorem~\ref{4.5},
$E$ is Riesz isomorphic to~$\Cont{X}$
for some compact space~$X$,
so we may as well assume that~$E$ \emph{is}~$\Cont{X}$.
Let~$E_u$ be the principal ideal of~$E$
generated by an element~$u$ of~$\Cont{X}^+$.

Let~$(f_n)_n$ and~$(\varepsilon_n)_n$
be sequences in~$E_u$ and~$(0,\infty)$,
respectively,
with
\begin{equation*}
\| f_n - f_N \|_u \leq \varepsilon_N \qquad (n\geq N)\htam{,}
\end{equation*}
i.e.,
\begin{equation*}
\label{eq4.7} \tag{$*$}
|f_n(x)-f_N(x)|\leq \varepsilon_N \,u(x) \qquad (x\in X;\ n\geq N)\htam{.}
\end{equation*}

For $n,N\in\N$ with $n\geq N$
we have $\|f_n-f_N\|_\infty \leq \varepsilon_N \|u\|_\infty$,
by~\ref{4.1}\ref{4.1-1};
so that~$(f_n)_n$ is a Cauchy sequence relative to~$\|\cdot\|_\infty$.
As~$\Cont{X}$ is complete,
this sequence converges to some~$f$ in~$\Cont{X}$.
By letting~$n$ tend to~$\infty$
in~\eqref{eq4.7} we obtain,
for all~$N$,
\begin{equation*}
|f(x)-f_N(x)|\leq\varepsilon_N\,u(x)\qquad(x\in X)\htam{.}
\end{equation*}
Then $|f|\leq |f_1|+\varepsilon_1 u$,
so that $f\in E_u$;
and $\| f - f_N\|_u\leq \varepsilon_N$,
so that~$(f_n)_n$
converges to~$f$ in the sense of~$\|\cdot\|_u$. \xqed
\end{proof}
\clearpage
\end{document}
