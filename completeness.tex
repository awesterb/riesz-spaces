\documentclass[main.tex]{subfiles}
\begin{document}
% 4
\section{Completeness}
After Yosida's Theorem~\ref{3.14},
a reasonable question is:
When is the map~$E\ra \Cont{\Phi}$ surjective?
%
%                  4.1
%
\begin{psec}{4.1}{Exercise}
Let $E$ be an Archimedean Riesz space with a unit, $e$.
\begin{enumerate}
\item \label{4.1-1}
Let $u\in E^+$ 
and let $E_u$ denote the principal ideal of~$E$ 
generated by~$u$ (\ref{2.11}). 
Then~$E_u$ is Archimedean and has~$u$ as a unit.
Show that
\begin{equation*}
\| x \|_e \leq \| u \|_e \, \| x \|_u \qquad (x\in E)
\end{equation*}
so that the identity map $E_u \ra E$
is continuous relative to~$\|\cdot\|_u$ and~$\|\cdot \|_e$.
%
\item \label{4.1-2}
Now let~$u$ be a unit of~$E$.
Show that~$\|\cdot\|_e$ and~$\|\cdot\|_u$
determine the same topology on~$E$,
and, less trivially:
$E$ is complete under (the metric induced by) 
the norm~$\|\cdot\|_u$
if and only if it is complete under~$\|\cdot\|_e$.
\end{enumerate}
\end{psec}
\clearpage
\end{document}
