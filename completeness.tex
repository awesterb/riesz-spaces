\documentclass[main.tex]{subfiles}
\begin{document}
% 4
\section{Completeness}
After Yosida's Theorem~\ref{3.14},
a reasonable question is:
When is the map~$E\ra \Cont{\Phi}$ surjective?
%
%                  4.1
%
\begin{psec}{4.1}{Exercise}
Let $E$ be an Archimedean Riesz space with a unit, $e$.
\begin{enumerate}
\item \label{4.1-1}
Let $u\in E^+$ 
and let $E_u$ denote the principal ideal of~$E$ 
generated by~$u$ (\ref{2.11}). 
Then~$E_u$ is Archimedean and has~$u$ as a unit.
Show that
\begin{equation*}
\| x \|_e \leq \| u \|_e \, \| x \|_u \qquad (x\in E)
\end{equation*}
so that the identity map $E_u \ra E$
is continuous relative to~$\|\cdot\|_u$ and~$\|\cdot \|_e$.
%
\item \label{4.1-2}
Now let~$u$ be a unit of~$E$.
Show that~$\|\cdot\|_e$ and~$\|\cdot\|_u$
determine the same topology on~$E$,
and, less trivially:
$E$ is complete under (the metric induced by) 
the norm~$\|\cdot\|_u$
if and only if it is complete under~$\|\cdot\|_e$.
\end{enumerate}
\end{psec}
%
%                  4.2
%
\begin{psec}{4.2}{Definitions}
Let $E$ be an Archimedean unitary Riesz space.
The topology induced by~$\|\cdot\|_e$,
where~$e$ is any unit,
is called the \keyword{(relatively) uniform topology}.
$E$ is said to be
\keyword{uniformly complete}
if it is complete relative to~$\|\cdot\|_e$
where~$e$ is any unit.

We use terms such as ``relatively uniformly closed''
and ``relatively uniformly convergent''
with the obvious meanings.
For instance,
if~$e$ is a unit in~$E$,
a sequence $(x_n)_n$
``converges relatively uniformly''
to an element $a$ 
if and only if for every $\varepsilon>0$
there is an~$N$ with
\begin{equation*}
|x_n - a| \leq \varepsilon e \qquad (n\geq N)\htam{.}
\end{equation*}

In spaces such as~$\ell^\infty$
and $\BCont{X}$
the constant function $\mathbb{1}$
serves as a unit.
As~$\|\cdot\|_\mathbb{1}$ coincides
with the sup-norm,
\emph{in these spaces,
relatively uniform convergence is just uniform convergence.}

Accordingly,
the qualification ``relatively'' is often dropped.
(However,
if~$E$ is the space of 
all scalar multiples of the function 
$i\colon x \mapsto x\quad (x\in [0,\infty)\,)$,
then the sequence $(n^{-1}i)_n$
converges to~$0$
relatively uniformly,
but not uniformly.
\end{psec}
%
%                  4.3
%
\begin{psec}{4.3}{Exercise}
Let $E$ be an Archimedean unitary Riesz space.
Show that,
with respect to the uniform topology,
the map $x\mapsto|x|$
is continuous
and every ``interval'' $[a,b]$ is closed.
\end{psec}
%
%                  4.4
%
\begin{psec}{4.4}{Exercise}
Show that in the Riesz space $\Cont{[0,1]}$
the principal Riesz ideal
generated by the function $t\mapsto t$
is not uniformly closed in $\Cont{[0,1]}$.
\end{psec}
%
%                  4.5
%
\begin{psec}{4.5}{Theorem}
(Continuation of~\ref{3.41})
\statement{%
If~$E$ is uniformly complete,
then~$\hat{E}=\Cont{\Phi}$.
Thus:
Every uniformly complete (Archimedean, unitary) Riesz space
is Riesz isomorphic to $\Cont{\Phi}$
for some compact Hausdorff space~$\Phi$.
}
\end{psec}
\begin{proof}
$E$ is complete relative to~$\|\cdot\|_e$.
Hence, $\hat{E}$ is complete relative to~$\|\cdot\|_{\hat{e}}$,
which is~$\|\cdot\|_\infty$.
Then~$\hat{E}$ is closed in~$\Cont{\Phi}$
(relative to~$\|\cdot\|_\infty$).
But it is also dense,
so~$\hat{E}$ equals~$\Cont{\Phi}$. \xqed
\end{proof}
\clearpage
\end{document}
