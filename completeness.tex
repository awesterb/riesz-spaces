\documentclass[main.tex]{subfiles}
\begin{document}
% 4
\section{Uniform Completeness}
After Yosida's Theorem~\ref{3.14},
a reasonable question is:
When is the map~$E\ra \Cont{\Phi}$ surjective?
%
%                  4.1
%
\begin{psec}{4.1}{Exercise}
Let $E$ be an Archimedean Riesz space with a unit, $e$.
\begin{enumerate}
\item \label{4.1-1}
Let $u\in E^+$ 
and let $E_u$ denote the principal ideal of~$E$ 
generated by~$u$ (\ref{2.11}). 
Then~$E_u$ is Archimedean and has~$u$ as a unit.
Show that
\begin{equation*}
\| x \|_e \leq \| u \|_e \, \| x \|_u \qquad (x\in E)
\end{equation*}
so that the identity map $E_u \ra E$
is continuous relative to~$\|\cdot\|_u$ and~$\|\cdot \|_e$.
%
\item \label{4.1-2}
Now let~$u$ be a unit of~$E$.
Show that~$\|\cdot\|_e$ and~$\|\cdot\|_u$
determine the same topology on~$E$,
and, less trivially:
$E$ is complete under (the metric induced by) 
the norm~$\|\cdot\|_u$
if and only if it is complete under~$\|\cdot\|_e$.
\end{enumerate}
\end{psec}
%
%                  4.2
%
\begin{psec}{4.2}{Definitions}
Let $E$ be an Archimedean unitary Riesz space.
The topology induced by~$\|\cdot\|_e$,
where~$e$ is any unit,
is called the \keyword{(relatively) uniform topology}.
$E$ is said to be
\keyword{uniformly complete}
if it is complete relative to~$\|\cdot\|_e$
where~$e$ is any unit.

We use terms such as ``relatively uniformly closed''
and ``relatively uniformly convergent''
with the obvious meanings.
For instance,
if~$e$ is a unit in~$E$,
a sequence $(x_n)_n$
``converges relatively uniformly''
to an element $a$ 
if and only if for every $\varepsilon>0$
there is an~$N$ with
\begin{equation*}
|x_n - a| \leq \varepsilon e \qquad (n\geq N)\htam{.}
\end{equation*}

In spaces such as~$\ell^\infty$
and $\BCont{X}$
the constant function $\mathbb{1}$
serves as a unit.
As~$\|\cdot\|_\mathbb{1}$ coincides
with the sup-norm,
\emph{in these spaces,
relatively uniform convergence is just uniform convergence.}

Accordingly,
the qualification ``relatively'' is often dropped.
(However,
if~$E$ is the space of 
all scalar multiples of the function 
$i\colon x \mapsto x\quad (x\in [0,\infty)\,)$,
then the sequence $(n^{-1}i)_n$
converges to~$0$
relatively uniformly,
but not uniformly.
\end{psec}
%
%                  4.3
%
\begin{psec}{4.3}{Exercise}
Let $E$ be an Archimedean unitary Riesz space.
Show that,
with respect to the uniform topology,
the map $x\mapsto|x|$
is continuous
and every ``interval'' $[a,b]$ is closed.
\end{psec}
%
%                  4.4
%
\begin{psec}{4.4}{Exercise}
Show that in the Riesz space $\Cont{[0,1]}$
the principal Riesz ideal
generated by the function $t\mapsto t$
is not uniformly closed in $\Cont{[0,1]}$.
\end{psec}
%
%                  4.5
%
\begin{psec}{4.5}{Theorem}
(Continuation of~\ref{3.14})
\statement{%
If~$E$ is uniformly complete,
then~$\hat{E}=\Cont{\Phi}$.
Thus:
Every uniformly complete (Archimedean, unitary) Riesz space
is Riesz isomorphic to $\Cont{\Phi}$
for some compact Hausdorff space~$\Phi$.
}
\end{psec}
\begin{proof}
$E$ is complete relative to~$\|\cdot\|_e$.
Hence, $\hat{E}$ is complete relative to~$\|\cdot\|_{\hat{e}}$,
which is~$\|\cdot\|_\infty$.
Then~$\hat{E}$ is closed in~$\Cont{\Phi}$
(relative to~$\|\cdot\|_\infty$).
But it is also dense,
so~$\hat{E}$ equals~$\Cont{\Phi}$. \xqed
\end{proof}
%
%                  4.6
%
\begin{psec}{4.6}{Examples}
\begin{enumerate}
\item \label{4.6-1}
For every set~$X$
the space $\ell^\infty(X)$ (\ref{1.5}\ref{1.5-2})
is uniformly complete.
(A brief proof:
Let $f_1,f_2,\dotsc\in \ell^\infty(X)$
and let $(\varepsilon_n)_n$ be a sequence in~$(0,\infty)$,
tending to~$0$ and with
\begin{equation*}
\|f_n-f_N\|_\infty \leq \varepsilon_N \qquad (n\geq N)\htam{.}
\end{equation*}
For every $x\in X$ we have $|f_n(x)-f_N(x)|\leq\varepsilon_N\quad (n\geq N)$,
so that $(f_n(x))_n$ is a Cauchy sequence in~$\R$;
let~$f(x)$ be its limit.
Thus we obtain a function~$f$.
For all~$x$,
$|f(x)-f_1(x)|=\lim_{n\ra \infty} |f_n(x)-f_1(x)|\leq \varepsilon_1$,
so~$f$ is bounded: $f\in \ell^\infty(X)$.
Take $N\in\N$;
then for all~$x$,
$|f(x)-f_N(x)|
=\lim_{n\ra\infty}|f_n(x)-f_N(x)|
\leq\varepsilon_N$.
Thus, $\| f-f_N \|_\infty \leq \varepsilon_N \ra 0$.)
%
\item \label{4.6-2}
If $X$ is a topological space,
$\BCont{X}$ is a $\|\cdot\|_\infty$-closed subset of~$\ell^\infty(X)$,
hence is uniformly complete.
(Proof of the closedness:
Let $f\in \ell^\infty(X)$ lie in the closure of~$\BCont{X}$.
Let $x_i\ra a$ in~$X$.
Question:
must $f(x_i)\ra f(a)$?
Take $\varepsilon>0$.
Question:
Is there an~$i_0$
such that $|f(x_i)-f(a)|<\varepsilon\quad(i\geq i_0)$?
Choose $g\in\BCont{X}$ 
with $\|f-g\|_\infty\leq \sfrac{\varepsilon}{3}$.
There is an~$i_0$ such that $|g(x_i)-g(a)|<\sfrac{\varepsilon}{3}$
for all $i\geq i_0$.
For such~$i$ we have
$|f(x_i)-f(a)|
\leq|f(x_i)-g(x_i)| + |g(x_i)-g(a)| + |g(a)-f(a)|
\leq \varepsilon$.)
%
\item \label{4.6-3}
In particular,
of course,
the~$\Cont{\Phi}$ of Yosida's Theorem is uniformly complete.
%
\item \label{4.6-4}
More examples follow later on in this chapter.
\end{enumerate}
\end{psec}
%
%                  4.7
%
\begin{psec}{4.7}{Lemma}
\statement{%
Let $E$ be a unitary Archimedean Riesz space.
If~$E$ is uniformly complete,
then so is every principal ideal of~$E$.
}
\end{psec}
\begin{proof}
By Theorem~\ref{4.5},
$E$ is Riesz isomorphic to~$\Cont{X}$
for some compact space~$X$,
so we may as well assume that~$E$ \emph{is}~$\Cont{X}$.
Let~$E_u$ be the principal ideal of~$E$
generated by an element~$u$ of~$\Cont{X}^+$.

Let~$(f_n)_n$ and~$(\varepsilon_n)_n$
be sequences in~$E_u$ and~$(0,\infty)$,
respectively,
with $\varepsilon_n\downarrow 0$, and
\begin{equation*}
\| f_n - f_N \|_u \leq \varepsilon_N \qquad (n\geq N)\htam{,}
\end{equation*}
i.e.,
\begin{equation*}
\label{eq4.7} \tag{$*$}
|f_n(x)-f_N(x)|\leq \varepsilon_N \,u(x) \qquad (x\in X;\ n\geq N)\htam{.}
\end{equation*}

For $n,N\in\N$ with $n\geq N$
we have $\|f_n-f_N\|_\infty \leq \varepsilon_N \|u\|_\infty$,
by~\ref{4.1}\ref{4.1-1};
so that~$(f_n)_n$ is a Cauchy sequence relative to~$\|\cdot\|_\infty$.
As~$\Cont{X}$ is complete,
this sequence converges to some~$f$ in~$\Cont{X}$.
By letting~$n$ tend to~$\infty$
in~\eqref{eq4.7} we obtain,
for all~$N$,
\begin{equation*}
|f(x)-f_N(x)|\leq\varepsilon_N\,u(x)\qquad(x\in X)\htam{.}
\end{equation*}
Then $|f|\leq |f_1|+\varepsilon_1 u$,
so that $f\in E_u$;
and $\| f - f_N\|_u\leq \varepsilon_N$,
so that~$(f_n)_n$
converges to~$f$ in the sense of~$\|\cdot\|_u$. \xqed
\end{proof}
%
%                  4.8
%
\begin{psec}{4.8}{Comment}
There is room for confusion here:
It may happen that~$E_u$ is uniformly complete
but not uniformly closed;
see, e.g., \ref{4.4}.
The tricky part is that the word ``uniformly''
refers first to the unit of~$E_u$,
later to the unit of~$E$.
\end{psec}
%
%                  4.9
%
\noindent
Let us turn to uniform completeness of quotients.
There we hit a snag:
a quotient of a uniformly complete Riesz space
need not even be Archimedean.
We have the following result.
\begin{psec}{4.9}{Theorem}
\statement{Let $D$ be a Riesz ideal
in a unitary Riesz space~$E$.
Then the quotient Riesz space~$E/D$
is Archimedean if and only if~$D$
is a uniformly closed subset of~$E$.}
\end{psec}
\begin{proof}
Let $Q$ be the quotient map $E\ra E/D$.
Choose a unit~$e$ in~$E$.
Observe that~$Qe$ is a unit in~$E/D$.
\begin{enumerate}[label=(\Roman*)]
\item \label{4.9-I}
Suppose $E/D$ is Archimedean.
Then~$Qe$ defines a norm~$\|\cdot\|_{Qe}$ on~$E/D$,
and from the definitions it is apparent that
\begin{equation*}
\|Qx\|_{Qe} \leq \| x \|_e \qquad (x\in E)\htam{.}
\end{equation*}
(Actually,
$\|\cdot\|_{Qe}$ is the quotient norm:
$\|Qx\|_{Qe}=\inf\{ \|y\|_e\colon Qy=Qx\}$.)
If~$a\in E$ lies in the $\|\cdot\|_e$-closure of~$D$,
then~$Qa$ lies in the $\|\cdot\|_{Qe}$-closure of~$Q(D)=\{0\}$,
so that~$Qa=0$ and~$a\in D$.
%
\item \label{4.9-II}
Suppose $D$ is $\|\cdot\|_e$-closed in~$E$.
Let $a,b\in E^+$, 
$Qa\leq n^{-1} Qb$ for all $n$;
we prove $Qa=0$.
As~$b$ is smaller than some multiple of~$e$,
we may assume~$b\leq e$.
For each~$n$, 
put $x_n := (a-n^{-1}b)^+$;
then $x_n \in D$
because $Qx_n=0$.
We have $a=a^+ \geq x_n \geq a-n^{-1} b\quad (n\in \N)$
and $b\leq e$;
it follows that $\| a - x_n \|_e\leq n^{-1}$ for each $n$.
Hence, $a$ lies in the $\|\cdot\|_e$-closure of~$D$,
which is~$D$. \xqed
\end{enumerate}
\end{proof}
%
%                  4.10
%
\noindent For uniformly closed ideals everything is fine:
\begin{psec}{4.10}{Theorem}
\statement{Let $D$ be a uniformly closed Riesz ideal
in a uniformly complete (unitary Archimedean) Riesz space~$E$.
Then~$E/D$ is (unitary, Archimedean, and) uniformly complete.}
\end{psec}
%
%                  4.11
%
\noindent Before turning to the proof
we consider a special situation:
\begin{psec}{4.11}{Theorem}
\statement{%
Let $X$ be a compact Hausdorff space,
$D$ a uniformly closed Riesz ideal of~$\Cont{X}$.
Then there exists a closed subset~$A$ of~$X$ such that
\begin{equation*}
D = \{ f\in\Cont{X}\colon f|A=0\}\htam{.}
\end{equation*}}%
\end{psec}
\begin{proof}
Define
\begin{equation*}
A := \bigcap_{f\in D} \bigl\{ x\colon f(x)=0 \bigr\}\htam{.}
\end{equation*}
$A$ is a closed set (possibly $\varnothing$).
Trivially, $D\subseteq\{ f\colon f|A=0\}$.
Conversely, suppose
\begin{equation*}
f\in \Cont{X}\htam{, }\qquad f|A=0\htam{; }
\end{equation*}
we prove $f\in D$.
For convenience,
assume~$f\geq 0$
(or replace~$f$ by~$|f|$).
Let~$\varepsilon\in(0,\infty)$;
we are done if we can find an~$f_\varepsilon$ in~$D$
with~$|f-f_\varepsilon|\leq \varepsilon\mathbb{1}$.

Claim:
$(f-\varepsilon\mathbb{1})^+$ is such an~$f_\varepsilon$.

First,
note that $f=f^+\geq (f-\varepsilon\mathbb{1})^+ 
\geq f-\varepsilon\mathbb{1}$,
so that $0\leq f - (f-\varepsilon \mathbb{1})^+ 
\leq \varepsilon \mathbb{1}$.
We are done if $(f-\varepsilon \mathbb{1})^+\in D$.

The set $B:=\{ x\colon f(x)\geq \varepsilon \}$ is closed.
For every point~$b$ of~$B$
we have~$b\notin A$,
so there is a~$g$ in~$D^+$ with~$g(b)>0$;
then there is also a~$g$ in~$D^+$
with $g(b)>f(b)$.
Hence:
\begin{align*}
B \subseteq &\bigcup_{\lsub{g\in D^+}{g\in D}} 
\bigl\{ x\colon g(x) > f(x) \bigr\}\htam{.}\\
\intertext{%
By the compactness of~$B$,
there exist $g_1,\dotsc,g_N$ in~$D^+$ such that}
B \subseteq &\bigcup_{\lsub{\phantom{g}\,n}{g\in D}} 
\bigl\{ x\colon g_n(x)>f(x) \bigr\}\htam{.}%
\end{align*}
Setting $g:=g_1\vee \dotsb \vee g_N$,
we get~$g\in D^+$
and $g>f$ on~$B$;
then $g\geq(f-\varepsilon\mathbb{1})^+$,
and $(f-\varepsilon \mathbb{1})^+ \in D$. \xqed
\end{proof}
\begin{proof} of Theorem~\ref{4.10}:
As in our proof of Lemma~\ref{4.7},
we assume $E=\Cont{X}$ for some compact~$X$.
Then $D=\{ f\colon |A = 0 \}$ for some closed $A\subseteq X$,
and $E/D$ is Riesz isomorphic to~$\Cont{A}$~(\ref{2.9}).
But $\Cont{A}$ is uniformly complete (\ref{4.6}\ref{4.6-1}). \xqed
\end{proof}
%
%                  4.12
%
\noindent The following is an application of Theorem~\ref{4.5}.
\begin{psec}{4.12}{The Stone--\v{C}ech compactification}
Let~$X$ be a topological space.
\begin{enumerate}[label=(\Roman*)]
\item \label{4.12-I}
The Stone--\v{C}ech compactification of~$X$,
denoted
\begin{equation*}
\beta X
\end{equation*}
is the compact Hausdorff space
obtained from~$\BCont{X}$
by applying the Yosida Theorem.
Explicitly:
$\beta X$ is 
the set of all Riesz homomorphisms
$\BCont{X}\ra \R$ that send~$\mathbb{1}_X$ to~$1$,
topologized as a subset of~$\R^{\BCont{X}}$.
It is,
indeed,
a compact Hausdorff space,
and we have a map
$\beta\colon X \ra \beta X$ given by
\begin{equation*}
\tag{$*$} \label{eq4.12*}
\beta(x)\ \htam{ is the evaluation }\ 
f\mapsto f(x) 
\quad(f\in \BCont{X}\,)\htam{.}
\end{equation*}
(The notation is standard but deplorable:
the range of~$\beta$ is~$\beta(X)$, not $\beta X$.)

The map $f\mapsto \hat{f}$ of~$\BCont{X}$ into~$\Cont{\beta X}$
satisfies $\hat{f}(\beta(x))=(\beta(x))(f)$, i.e.,
\begin{alignat*}{2}
\tag{$**$} \label{eq4.12**}
&\hat{f}(\beta(x)) = f(x) \qquad (x\in X\htam{,}\ f\in\BCont{X}\,)\htam{.}\\
&\xymatrix{
X\ar[r]^f \ar[d]_\beta & \R \\
\beta X \ar[ru]_{\hat f}}
\end{alignat*}
As $\BCont{X}$ is uniformly complete (\ref{4.6}\ref{4.6-2}),
the correspondence $f\mapsto \hat{f}$
is a Riesz isomorphism of~$\BCont{X}$ onto~$\Cont{\beta X}$.
%
\item \label{4.12-II}
$\beta(X)$ is a subset of~$\beta X$.
If $x,y\in X$ and $\beta(x)=\beta(y)$,
then $f(x)=f(y)$ for all $f$~in~$\BCont{X}$.
Hence,
for every~$f$ in~$\BCont{X}$
the formula $\beta(x)\mapsto f(x)$ 
determines a function on~$\beta(X)$.
By the above,
this function (is continuous and)
extends to the continuous 
function $\hat{f}\colon \beta X \ra \R$.

Often (see below) $\beta$ is injective
and a homeomorphism of~$X$ onto~$\beta(X)$;
then~$X$ and~$\beta(X)$ may be identified,
and one may view~$X$ as a subset of~$\beta X$,
and~$\hat f$ as an extension of~$f$.
%
\item \label{4.12-III}
If $X$ itself is a compact Hausdorff space,
then~$\beta$ is a homeomorphism of~$X$ onto~$\beta X$;
that is~\ref{3.12}.
%
\item \label{4.12-IV}
$\beta(X)$ is always a dense subset of~$\beta X$.
Indeed, suppose it is not.
By Urysohn's Lemma
there exists a nonzero~$g$ in $\Cont{\beta X}$
such that $g=0$ on (the closure of) $\beta(X)$;
then we have a nonzero~$f$ in $\BCont{X}$
with $\hat f = 0$ on~$\beta (X)$;
but then (with \eqref{eq4.12**}) 
$f = \hat f \circ \beta = 0$:
contradiction.
\item \label{4.12-V}
Let~$(x_i)_i$ be a net in~$X$
and let $a\in X$.
As $\{\hat f\colon f\in\BCont{X}\}=\Cont{\beta X}$,
we have,
thanks to~\ref{2.18},
\begin{alignat*}{2}
\beta(x_i) \longrightarrow \beta(a) 
& \quad\iff\quad \hat f(\beta(x_i))\longrightarrow \hat f(\beta(a))\qquad
 (f\in \BCont{X}\,)\\
& \quad\iff\quad f(x_i)\longrightarrow f(a) \qquad
 (f\in \BCont{X}\,)
\end{alignat*}%
In particular,
this means that~$\beta$ is continuous.
%
\item \label{4.12-VI}
Since $\beta X$ is completely regular, 
$\beta(X)$ is completely regular too.
So if $\beta$ is a homeomorphism $X\ra \beta(X)$,
$X$ is completely regular.
Conversely, $\beta$ is a homeomorphism $X\rightarrow \beta(X)$,
if $X$ is completely regular (e.g., metrizable).

Indeed, suppose $X$ is completely regular.
If $a,b\in X$ and $a\neq b$,
there are functions~$f$ in $\BCont{X}$
with $f(a)\neq f(b)$,
then $(\beta(a))(f)\neq (\beta(b))(f)$,
so $\beta(a)\neq\beta(b)$.
Thus,
$\beta$ is injective,
hence a bijection $X\rightarrow \beta(X)$.
If~$(x_i)_i$ is a net in~$X$
and $a\in X$,
and if $\beta(x_i)\rightarrow \beta(a)$,
then $f(x_i)\rightarrow f(a)$ for all~$f$ in $\BCont{X}$;
Thus by lemma~\ref{2.18},
we obtain $x_i\rightarrow a$.
Thus,
$\beta^{-1}$ is continuous.
\end{enumerate}
\end{psec}
%
%                  4.13
%
\begin{psec}{4.13}{Exercise}
One might expect the Stone--\v{C}ech compactification of
$(0,1]$ to be (homeomorphic to) $[0,1]$.
Prove that this is false 
by showing that the sequence $(\beta(n^{-1}))_n$ in $\beta(0,1]$
has no converging subsequence.
\end{psec}
%
%                  4.14
%
\begin{psec}{4.14}{Exercise}
\begin{enumerate}
\item \label{4.14-1}
Let $X,Y$ be topological spaces,
$\tau$ a continuous map $X\rightarrow Y$.
Prove that the following formula
\begin{equation*}
\beta\tau\colon\varphi\mapsto\varphi\circ\tau\qquad(\varphi\in\BCont{X}\,)
\end{equation*}
defines a continuous map $\beta\tau\colon\beta X\ra\beta Y$,
satisfying $(\beta\tau)\circ\beta=\beta\circ\tau$:
\begin{equation*}
\xymatrix{
X\ar[r]^\tau \ar[d]_\beta& Y\ar[d]_\beta \\
\beta X\ar[r]^{\beta\tau} & \beta Y
}
\end{equation*}
%
\item \label{4.14-2}
Let~$X$ be a topological space, 
$Y$ a compact Hausdorff space.
Show that every continuous $\tau\colon X\ra Y$
induces a continuous $\hat\tau\colon\beta X\ra Y$
with $\hat\tau\circ\beta=\tau$.
\end{enumerate}
\end{psec}
%
%                  4.15
%
\begin{psec}{4.15}{Exercise}
$\ell^\infty$ is $\BCont{\N}$,
so the space~$\Phi$ considered in~\ref{3.19} is~$\beta\N$
(on which a book has been written).
More about~$\beta\N$:
\begin{enumerate}
\item \label{4.15-1}
Show that for every $n\in\N$
the singleton set $\{\beta(n)\}$ is open in~$\beta \N$.

Thus~$\beta(\N)$ is a cloud of isolated points,
dense in~$\beta\N$.
It turns out that~$\beta\N$ is very large:
its cardinality is
that of the power set of the power set of~$\N$.
As a modest step in this direction
\emph{we make an injection} $\R\ra\beta\N$:
\item \label{4.15-2}
Show: For every $A\subseteq \N$, 
$\hat{\mathbb 1}_A$ is~${\mathbb 1}_{\hat A}$
for some compact open subset~$\hat A$ of~$\beta\N$.
(This was also in~\ref{3.19}\ref{3.19-1}.) We have
\begin{equation*}
\hat A \subseteq \beta(\N)
\quad\htam{ if and only if }\quad 
A\htam{ is finite.}
\end{equation*}
%
\item \label{4.14-3}
For each $\sr\in \R$,
choose a sequence $(\sr_n)_n$ in $\mathbb{Q}$
that converges to~$\sr$,
such that the set $S_\sr:=\{\sr_1,\sr_2,\dotsc\}$
is infinite.
Then $S_\sr \cap S_s$ is finite whenever $\sr,s\in \R$, $\sr\neq s$.
Via a bijection $\mathbb{Q}\leftrightarrow \N$ we see:
there is a family $(A_\sr)_{\sr\in\R}$
of infinite subsets of~$\N$
such that $A_\sr\cap A_s$ is finite if $\sr\neq s$.

Now prove:
The sets ${\hat A}_\sr \backslash \beta(\N)$,
where~$\sr$ runs through $\R$,
are nonempty and pairwise disjoint.
\end{enumerate}
\end{psec}
%
%                  4.16
%
\noindent We return to the general theory
for a discussion of a property that,
for unitary spaces,
implies completeness.
\begin{psec}{4.16}{Definitions}
A Riesz space is called \keyword{Dedekind complete}
if every nonempty subset that is bounded from above
has a supremum;
it is \keyword{$\sigma$-Dedekind complete}
if every countable nonempty subset
that is bounded from above
has a supremum.
\end{psec}
%
%                  4.17
%
\begin{psec}{4.17}{Comments}
\begin{enumerate}
\item \label{4.17-1}
Of course,
a Riesz space is [$\sigma$-]Dedekind complete
if and only if every [countable] nonempty subset
that is bounded from below has an infimum.
%
\item \label{4.17-2}
There are various alternative ways to describe
[$\sigma$-]Dedekind completeness.
One of them:

Call a subset~$W$ of~$E$ \keyword{upwards directed}
if for all $w_1,w_2\in W$
there exists a $w\in W$ with $w_1\leq w$, $w_2\leq w$.
Claim:
\statement{$E$ is already [$\sigma$-]Dedekind complete
if every [countable] nonempty subset of~$E^+$ that is upwards
directed and bounded from above has a supremum.}
Indeed,
suppose the latter condition is satisfied.
Let $\varnothing\neq W\subseteq E$ 
and let~$W$ be bounded from above;
for the $\sigma$-completeness,
assume~$W$ is countable.
Choose~$w_0$ in~$W$.
The set
\begin{equation*}
W_0 :=\bigl\{w_0\vee w_1\vee \dotsb \vee w_N\colon
 N\in \N,\ w_1,\dotsc,w_N\in W\bigr\}
\end{equation*}
is upwards directed and has the same upper bounds~$W$ has.
By the given property of~$E$,
the set $W_0 - w_0$ has a supremum.
Then $\sup(W_0-w_0)+w_0$ is the supremum of~$W_0$ and of $W$.
\end{enumerate}
\end{psec}
%
%                  4.18
%
\begin{psec}{4.18}{Theorem} \statement{
\begin{enumerate}
\item \label{4.18-1}
Dedekind complete spaces are $\sigma$-Dedekind complete.
%
\item \label{4.18-2}
$\sigma$-Dedekind complete spaces are Archimedean.
%
\item \label{4.18-3}
Unitary $\sigma$-Dedekind complete spaces are uniformly complete.
\end{enumerate} 
} \end{psec}
\begin{proof}
\begin{enumerate}
\item is trivial.
\item Let~$E$ be a $\sigma$-Dedekind complete Riesz space.
Let $a,b\in E^+$ be such that $na\leq b$ for all~$n$ in~$\N$;
we prove $a=0$. (See~\ref{3.2}.)
The set $\{na\colon n\in\N\}$ has a supremum, $c$, say.
For all~$n$ in~$\N$ we have $(n+1)a\leq c$,
so that $na\leq c-a$.
Then $c-a$ is an upper bound for $\{na\colon n\in \N\}$,
$c-a\geq c$, $a\leq 0$, and $a=0$.
%
\item We turn the argument of~\ref{4.5} inside out
and use Yosida's Theorem to prove uniform completeness.
Let~$E$ be a unitary $\sigma$-Dedekind complete Riesz space;
we show that it is uniformly complete.
By Yosida's Theorem we may assume
that~$E$ is a dense Riesz subspace of some $\Cont{X}$,
with $\mathbb{1}_X\in E$, 
and by~\ref{4.6}\ref{4.6-2}
it suffices to prove $E=\Cont{X}$.

Take $f\in\Cont{X}$.
For each $n\in\N$ there is an~$f_n$ in~$E$
with $\|f-f_n\|_\infty \leq n^{-1}$, i.e.,
\begin{equation*}
f_n-n^{-1}\mathbb{1}\ \leq\ f\ \leq\ f_n+n^{-1}\mathbb{1}\htam{.}
\end{equation*}
The set $\{f_n - n^{-1}\mathbb{1}\colon n\in\N\}$ has a supremum, 
$u$, in~$E$.
The set $\{f_n+n^{-1} \mathbb{1}\colon n\in \N\}$ has an infimum,
$v$, in~$E$.
For all~$n$ we have
\begin{equation*}
f_n-n^{-1}\mathbb{1}\ \leq\ u\ \leq\ v\ \leq\ f_n+n^{-1}\mathbb{1}\htam{.}
\end{equation*}
Then $f=u=v\in E$. \xqed
\end{enumerate}
\end{proof}
%
%                  4.19
%
\begin{psec}{4.19}{Examples}
\begin{enumerate}
\item \label{4.19-1}
It is perfectly clear that~$\R^X$ and $\ell^\infty(X)$
are Dedekind complete.
\item \label{4.19-2}
Riesz ideals of a [$\sigma$-]Dedekind complete Riesz spaces
are [$\sigma$-]Dedekind complete.
\item \label{4.19-3}
Thus, the spaces~$\ell^1$ and~$\cseq_0$ (\ref{1.5}\ref{1.5-5})
are Dedekind complete.
\end{enumerate}
\end{psec}
%
%                  4.20
%
\begin{psec}{4.20}{Exercise}
\begin{enumerate}
\item \label{4.20-1}
Let $\cseq$ be the Riesz space of all converging number sequences.
(See~\ref{1.5}\ref{1.5-5}.)
Show that~$\cseq$ is unitary and uniformly complete
but not $\sigma$-Dedekind complete.
(It may be convenient to view~$\cseq$ as the space of all Cauchy sequences.)
%
\item \label{4.20-2}
Let $E$ be the Riesz space of all bounded functions~$f$ on~$\R$
with the property:
\begin{alignat*}{2}
&\htam{There exists a countable set }A\subset\R \\
&\htam{such that }f\htam{ is constant on }\R\backslash A\htam{.}
\end{alignat*}
Show that~$E$ is $\sigma$-Dedekind complete but not Dedekind complete.
\end{enumerate}
\end{psec}
%
%                  4.21
%
We return to Dedekind completeness in Chapter~6. 
%  chapter 6 does not exist yet
For the moment,
by way of nontrivial examples,
we prove the following:
\begin{psec}{4.21}{Theorem}\statement{
$\Mod(\Lambda)$ (\ref{1.10}),
$\BA(\mathcal{A})$ (\ref{1.12}) 
and $\sigma(\mathcal{A})$ (\ref{1.14})
are Dedekind complete.
}\end{psec}
\begin{proof}
$\BA(\mathcal A)$ is a special case of $\Mod(\Lambda)$,
and $\sigma(\mathcal A)$ is an ideal in $\BA(\mathcal A)$,
so we only consider $\Mod(\Lambda)$:
Let~$\Lambda$ be a distributive lattice with a smallest element, $0$.

If $\varphi,\psi\in\Mod(\Lambda)$ and $\varphi\leq\psi$,
then $\psi-\varphi$ is increasing,
so that for all $s\in\Lambda$, 
$(\psi-\varphi)(s)\geq(\psi-\varphi)(0)=0$.
Thus:
\begin{equation}
\label{eq4.21} \tag{$*$}
\begin{alignedat}{2}
&\htam{For }s\in\Lambda\htam{ the function }\varphi \mapsto \varphi(s) \\
&\htam{on }\Mod(\Lambda)\htam{ is increasing.}
\end{alignedat}
\end{equation}

Let $\Omega$ be a nonempty subset of $\Mod(\Lambda)^+$,
upwards directed and bounded from above.
Invoking~\ref{4.17}\ref{4.17-2},
we see that we are done if we prove that~$\Omega$ has a supremum.

Every $s\in\Lambda$ induces 
a net $(\omega(s))_{\omega\in\Omega}$ in~$\R$ that
by~\eqref{eq4.21} is increasing and bounded from above,
hence converging.
Define $\omega_0\colon\Lambda\ra\R$ by
\begin{equation*}
\omega_0(s):=\lim_{\omega\in\Omega}\omega(s)
=\sup_{\omega\in\Omega}(\omega(s))\qquad(s\in \Lambda)\htam{.}
\end{equation*}

From this, it is clear that $\omega_0(0)=0$
and that~$\omega_0$ is modular (\ref{1.8}\ref{1.8-2}) and increasing;
hence $\omega_0 \in\Mod(\Lambda)^+$.

Take~$\varphi$ in~$\Omega$.
Then $\Omega_\varphi:=\{\omega\in \Omega\colon \omega\geq \varphi\}$
is cofinal in~$\Omega$, so
\begin{equation*}
\omega_0(s)=\lim_{\omega\in\Omega_\varphi} \omega(s) 
\qquad (s\in\Lambda)\htam{.}
\end{equation*}
By the definition of the ordering in $\Mod(\Lambda)$,
for every~$\omega$ in~$\Omega_\varphi$
the function $\omega-\varphi$ is increasing:
then so is $\omega_0-\varphi$.
Thus, $\omega_0$ is an upper bound for~$\Omega$ in $\Mod(\Lambda)$.

Finally,
let $\omega'$ be any upper bound for~$\Omega$.
Then for all~$\omega$ in~$\Omega$ we have 
that the function $\omega'-\omega$ is increasing.
From the definition of~$\omega_0$ it follows that $\omega'-\omega_0$
is increasing, i.e., $\omega'\geq\omega_0$ in $\Mod(\Lambda)$.

Thus, $\omega_0=\sup\Omega$. \xqed
\end{proof}
\clearpage
\emptypage
\end{document}
