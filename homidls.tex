\documentclass[main.tex]{subfiles}
\begin{document}
% 2
\section{Homomorphisms, ideals}
%
%                  2.1
%
\begin{psec}{2.1}%
(See \ref{1.5}\ref{1.5-8}.) 
Let $(X,\mathcal A, \mu)$ be a measure space.
For an $\mathcal A$-measurable 
function $f\colon X\ra \R$
denote by~$(f)_\mu$
the set of all $\mathcal A$-measurable functions
that are equal to~$f$ $\mu$-a.e.,
and put 
$F_{\mathcal A}(\mu):=\{(f)_\mu\colon f\in\mathcal{F}_{\mathcal A}\}$.
Then~$F_{\mathcal A}(\mu)$ 
is a Riesz-space
under the ordering
\begin{align*}
& (f)_\mu \leq (g)_\mu \quad \iff \quad f\leq g\ \mu\htam{-a.e.,} \\
\intertext{%
the lattice operations begin given by}%
& (f)_\mu \vee (g)_\mu = (f\vee g)_\mu \htam{, }
  \qquad (f)_\mu \wedge (g)_\mu = (f\wedge g)_\mu
\htam{.}
\end{align*}
For a general approach to similar constructions
it pays to investigate ``Riesz homomorphisms'' 
and ``Riesz ideals''.
\end{psec}
%
%                  2.2
%
\begin{psec}{2.2}{Definitions}
Let $E$ and $F$ be Riesz spaces 
and $T$ a linear map $E\ra F$.
We call~$T$ \keyword{positive} 
if it is order preserving ($=\htam{increasing}$),
which is the case if and only if
\begin{equation*}
T(E^+) \subseteq F^+\htam{, }
\end{equation*}
whereas $T$ is a \keyword{Riesz homomorphism} 
it is (linear and) a 
lattice homomorphism, i.e.,
\begin{equation*}
T(x\vee y) = Tx \vee Ty\htam{,}
\quad T(x\wedge y)=Tx\wedge Ty
\qquad (x,y\in E)\htam{.}
\end{equation*}
Every Riesz homomorphism is positive. 
($x\in E^+\iff x=x\vee 0$)

We leave it to the reader to define ``Riesz isomorphism''
and ``Riesz isomorphic''.

A linear (bijective) order isomorphism is,
of course,
a Riesz isomorphism,
but some positive linear bijections are not
(e.g., $(x_1,x_2)\mapsto(x_1,x_1+x_2)$ as a map
$\R^2\ra\R^2$).
\end{psec}
%
%                  2.3
%
\begin{psec}{2.3}{Exercise}
Let $E$ and $F$ be Riesz spaces,
$T$ a linear map $E\ra F$.
Prove the equivalence of:
\begin{enumerate}
\item[$(\alpha)$]  \label{2.3-alpha}
$T$ is a Riesz homomorphism.
%
\item[$(\beta)$]  \label{2.3-beta}
$T(|x|)=|Tx| \quad (x\in E)$.
%
\item[$(\gamma)$]  \label{2.3-gamma}
$T(x^+) = \lsub{(Tx)^+}{(Tx)} \quad (x\in E)$.
\end{enumerate}
\end{psec}
%
%                  2.4
%
\begin{psec}{2.4}%
Let $E$ be a Riesz space and $T$ a Riesz homomorphism of $E$
into a Riesz space~$F$.
Consider
 $D:=\{x\in E\colon Tx=0\}$.
This~$D$ is a linear subspace of~$E$ 
with the additional property
\begin{equation}
\tag{$*$}
\label{eq2.4}
a\in D\htam{, }\; x\in E\htam{, }\; |x|\leq |a| 
  \quad \implies \quad x\in D\htam{. }
\end{equation}
($|Tx|=T(|x|)\leq T(|a|)=|Ta|=0$.)  
Conversely:
\end{psec}
%
%                  2.5
%
\begin{psec}{2.5}{Exercise}
Let $E$ be a Riesz space and $D$ a linear subspace of~$E$
with the property~\eqref{eq2.4} of~\ref{2.4}.
Show that there is one (and only one)
ordering on the quotient vector space $E/D$ 
that turns $E/D$ into a Riesz space 
and the quotient map
$E\ra E/D$ into a Riesz homomorphism.
\end{psec}
%
%                  2.6
%
\begin{psec}{2.6}{Definition}
A \keyword{Riesz ideal} of a Riesz space $E$ 
is a linear subspace~$D$ of~$E$ for which
\begin{equation*}
a\in D\htam{, }\; x\in E\htam{, }\; |x|\leq |a| 
  \quad \implies \quad x\in D\htam{. }
\end{equation*}
\end{psec}

Observe that a Riesz ideal is a Riesz subspace. 
(Take $x=|a|$.)
A Riesz subspace (not just a linear subspace) $D$ of $E$
is a Riesz ideal if and only if 
\begin{equation*}
a\in D^+\htam{, }\; x\in E^+\htam{, }\; x\leq a 
  \quad \implies \quad x\in D\htam{. }
\end{equation*}
%
%                  2.7
%
\begin{psec}{2.7}{Theorem}
\statement{
A subset of a Riesz space
is a Riesz ideal 
if and only if
it is the kernel of a Riesz homomorphism.
}

It will be clear
how one defines
``quotient Riesz space''.
\end{psec}
%
%                  2.8
%
\begin{psec}{2.8}{Examples}
\begin{enumerate}
\item \label{2.8-1}%
(Back to~\ref{2.1}.)
Let $(X,\mathcal A, \mu)$ be a measure space
$D:=\{f\in {\mathcal F}_{\mathcal A}\colon f=0\ \mu\htam{-a.e.} \}$
is a Riesz ideal of~${\mathcal F}_{\mathcal A}$;
the quotient Riesz space is~$\mathrm{F}_{\mathcal A} (\mu)$.

The same~$D$ is a Riesz ideal of~$\mathcal{L}_{\mathcal A}[\mu]$;
the quotient~${\mathcal L}_{\mathcal A}[\mu]/I$ is denoted
\begin{equation*}
\mathrm{L}^1(\mu)\htam{.}
\end{equation*}

Further,
$\mathcal{L}_{\mathcal A}[\mu]$ 
is a Riesz ideal of~${\mathcal F}_{\mathcal A}$
(Maat en Integraal),
and~$\mathrm{L}^1(\mu)$ is a Riesz ideal of~$\mathrm{F}_{\mathcal A}(\mu)$.
%
\item \label{2.8-2}
$\ell^\infty(X)$ is a Riesz ideal of $\R^X$ (see~\ref{1.5}\ref{1.5-1}).
For a topological space~$X$, 
$\BCont{X}$ is a Riesz ideal 
of~$\Cont{X}$ (\ref{1.5}\ref{1.5-3}).
$\cseq_0$ and $\ell^1$ are Riesz ideals 
in~$\ell^\infty$ ($:=\ell^\infty(X)$).
%
\item \label{2.8-3}
Let $\mathcal A$ be a $\sigma$-algebra in a set $X$.
Then $\sigma(\mathcal A)$ is a Riesz ideal of~$\BA(\mathcal A)$.
Indeed, as we already know~$\sigma(\mathcal A)$ 
to be a Riesz subspace of~$\BA(\mathcal A)$,
it will be an ideal if
\begin{equation*}
\mu\in \sigma(\mathcal A)\htam{, }\ 
\tau\in\BA(\mathcal A)\htam{, }\ 
0\leq \tau\leq \mu
\quad \implies \quad
\tau \in \sigma(\mathcal A)
\htam{.}
\end{equation*}
But this follows from the observation
that an additive $\alpha\colon\mathcal A\ra [0,\infty)$
is a finite measure if and only if
\begin{equation*}
A_n \downarrow \eset\ \htam{ in }\mathcal A 
\quad \implies \quad
\alpha(A_n)\ra 0
\htam{.}
\end{equation*}
%
\item \label{2.8-4}
If $X$ is a topological space and $Y\subseteq X$, then
\begin{equation*}
\{f\in\Cont{X}\colon f | Y=0 \}
\end{equation*}
is a Riesz ideal of $\Cont{X}$.

Not all Riesz ideals of $\Cont{X}$ are of this type.
For instance,
in~$\Cont{\R}$ the functions~$f$
with $\lim_{x\ra \infty} x^3 f(x)=0$
form a Riesz ideal.
\end{enumerate}
\end{psec}
%
%                  2.9
%
\begin{psec}{2.9}{Exercise}
Let $X$ be a topological space
and $A$ a closed subset of~$X$ 
such that every continuous function on~$A$
can be extended to a continuous function on~$X$ 
(e.g., $X$ is compact Hausdorff or $X$ is metrizable,
and $A$ is any closed set).
Let~$J_A$ be the Riesz ideal
$\{f\in\Cont{X}\colon f|A=0\}$.
Show that the quotient Riesz space
$\Cont{X}/J_A$ is Riesz isomorphic to~$\Cont{A}$.
\end{psec}
%
%                  2.10
%
\begin{psec}{2.10}{Exercise}
\begin{enumerate}
\item \label{2.10-1}
Show that the sum of two Riesz subspaces 
of a Riesz space~$E$ need not be a Riesz subspace of~$E$.
(The space of all affine functions on~$[0,1]$ (see~\ref{1.7})
is a sum of two $1$-dimensional Riesz subspaces of~$\Cont{[0,1]}$.)
%
\item \label{2.10-2}
Prove that the sum of two Riesz ideals
of a Riesz space~$E$
is a Riesz ideal of~$E$.
(This is not so easy.
It may be useful to prove
the ``Riesz Decomposition Lemma'':
if~$x,a,b\in E^+$ and $x\leq a+b$,
then there exist $y,z\in E^+$ 
with $x=y+z$, $y\leq a$, $z\leq b$;
e.g., $y=x\wedge a$ and $z=x-y=x\vee a-a$.)
\end{enumerate}
\end{psec}
%
%                  2.11
%
\begin{psec}{2.11}{Definitions}
The intersection of any collection of Riesz subspaces
(or ideals) of a Riesz space~$E$ 
is again a Riesz subspace
(or ideal, respectively)
of~$E$.

For $u\in E^+$, 
the \keyword{principal (Riesz) ideal generated} by~$u$ 
is the intersection of all Riesz ideals
that contain~$u$;
it is the set
\begin{equation*}
\bigcup_{n\in\N} [-nu,nu]
\htam{.}
\end{equation*}

An element $u$ of $E^+$
is called 
a \keyword{(strong) unit} of~$E$
if the principal ideal it generates is~$E$ itself,
i.e., 
if for every~$x$ in~$E$
there exists an~$n$ in~$\N$ with $|x|\leq ne$.

$E$ is said to be \keyword{unitary} if it has a unit.
\end{psec}
%
%                  2.12
%
\begin{psec}{2.12}{Exercise}
\begin{enumerate}
\item \label{2.12-1}
Trivially,
the constant function $\mathbb{1}$
is a unit in~$\BCont{\R}$.
Show that~$\Cont{\R}$ has no unit.
%
\item \label{2.12-2}
Show that $\cseq_0$ and $\cseq$ are not Riesz isomorphic.
\end{enumerate}
\end{psec}
%
%                  2.13
%
\begin{psec}{2.13}{Definition}
If $u$ is a unit in a Riesz space~$E$, 
then
\begin{equation}
\label{eq2.13_1}
\tag{$*$}
x\in E\htam{, }\ x\perp u 
\quad \implies \quad
x=0
\htam{.}
\end{equation}
(Indeed, there is an $n$ in $\N$ with $|x|\leq nu$;
then $|x|=|x|\wedge nu\leq n|x|\wedge nu=n(|x|\wedge u)=0$.)

An element $u$ of $E$ 
is called
a $\keyword{weak unit}$,
or a $\keyword{Freudenthal unit}$,
if~\eqref{eq2.13_1} holds.
The constant function~$\mathbb{1}$
is a weak unit in~$\Cont{\R}$.
\end{psec}
\clearpage
\end{document}
