\documentclass[main.tex]{subfiles}
\begin{document}
% 2
\section{Homomorphisms, ideals}
%
%                  2.1
%
\begin{psec}{2.1}%
(See \ref{1.5}\ref{1.5-8}.) 
Let $(X,\mathcal A, \mu)$ be a measure space.
For an $\mathcal A$-measurable 
function $f\colon X\ra \R$
denote by~$(f)_\mu$
the set of all $\mathcal A$-measurable functions
that are equal to~$f$ $\mu$-a.e.,
and put 
$F_{\mathcal A}(\mu):=\{(f)_\mu\colon f\in\mathcal{F}_{\mathcal A}\}$.
Then~$F_{\mathcal A}(\mu)$ 
is a Riesz space
under the ordering
\begin{align*}
& (f)_\mu \leq (g)_\mu \quad \iff \quad f\leq g\ \mu\htam{-a.e.,} \\
\intertext{%
the lattice operations being given by}%
& (f)_\mu \vee (g)_\mu = (f\vee g)_\mu \htam{, }
  \qquad (f)_\mu \wedge (g)_\mu = (f\wedge g)_\mu
\htam{.}
\end{align*}
For a general approach to similar constructions
it pays to investigate ``Riesz homomorphisms'' 
and ``Riesz ideals''.
\end{psec}
%
%                  2.2
%
\begin{psec}{2.2}{Definitions}
Let $E$ and $F$ be Riesz spaces 
and $T$ a linear map $E\ra F$.
We call~$T$ \keyword{positive} 
if it is order preserving ($=\htam{increasing}$),
which is the case if and only if
\begin{equation*}
T(E^+) \subseteq F^+\htam{, }
\end{equation*}
and $T$ is a \keyword{Riesz homomorphism} if
it is (linear and) a 
lattice homomorphism, i.e.,
\begin{equation*}
T(x\vee y) = Tx \vee Ty\htam{,}
\quad T(x\wedge y)=Tx\wedge Ty
\qquad (x,y\in E)\htam{.}
\end{equation*}
Every Riesz homomorphism is positive. 
($x\in E^+\iff x=x\vee 0$)

We leave it to the reader to define ``Riesz isomorphism''
and ``Riesz isomorphic''.

A linear (bijective) order isomorphism is,
of course,
a Riesz isomorphism,
but some positive linear bijections are not
(e.g., $(x_1,x_2)\mapsto(x_1,x_1+x_2)$ as a map
$\R^2\ra\R^2$).
\end{psec}
%
%                  2.3
%
\begin{psec}{2.3}{Exercise}
Let $E$ and $F$ be Riesz spaces,
$T$ a linear map $E\ra F$.
Prove the equivalence of:
\begin{enumerate}
\item[$(\alpha)$]  \label{2.3-alpha}
$T$ is a Riesz homomorphism.
%
\item[$(\beta)$]  \label{2.3-beta}
$T(|x|)=|Tx| \quad (x\in E)$.
%
\item[$(\gamma)$]  \label{2.3-gamma}
$T(x^+) = \lsub{(Tx)^+}{(Tx)} \quad (x\in E)$.
\end{enumerate}
\end{psec}
%
%                  2.4
%
\begin{psec}{2.4}%
Let $E$ be a Riesz space and $T$ a Riesz homomorphism of $E$
into a Riesz space~$F$.
Consider
 $D:=\{x\in E\colon Tx=0\}$.
This~$D$ is a linear subspace of~$E$ 
with the additional property
\begin{equation}
\tag{$*$}
\label{eq2.4}
a\in D\htam{, }\; x\in E\htam{, }\; |x|\leq |a| 
  \quad \implies \quad x\in D\htam{. }
\end{equation}
($|Tx|=T(|x|)\leq T(|a|)=|Ta|=0$.)  
Conversely:
\end{psec}
%
%                  2.5
%
\begin{psec}{2.5}{Exercise}
Let $E$ be a Riesz space and $D$ a linear subspace of~$E$
with the property~\eqref{eq2.4} of~\ref{2.4}.
Show that there is one (and only one)
ordering on the quotient vector space $E/D$ 
that turns $E/D$ into a Riesz space 
and the quotient map
$E\ra E/D$ into a Riesz homomorphism.
\end{psec}
%
%                  2.6
%
\begin{psec}{2.6}{Definition}
A \keyword{Riesz ideal} of a Riesz space $E$ 
is a linear subspace~$D$ of~$E$ for which
\begin{equation*}
a\in D\htam{, }\; x\in E\htam{, }\; |x|\leq |a| 
  \quad \implies \quad x\in D\htam{. }
\end{equation*}
\end{psec}

Observe that a Riesz ideal is a Riesz subspace. 
(Take $x=|a|$.)
A Riesz subspace (not just a linear subspace) $D$ of $E$
is a Riesz ideal if and only if 
\begin{equation*}
a\in D^+\htam{, }\; x\in E^+\htam{, }\; x\leq a 
  \quad \implies \quad x\in D\htam{. }
\end{equation*}
%
%                  2.7
%
\begin{psec}{2.7}{Theorem}
\statement{
A subset of a Riesz space
is a Riesz ideal 
if and only if
it is the kernel of a Riesz homomorphism.
}

It will be clear
how one defines
``quotient Riesz space''.
\end{psec}
%
%                  2.8
%
\begin{psec}{2.8}{Examples}
\begin{enumerate}
\item \label{2.8-1}%
(Back to~\ref{2.1}.)
Let $(X,\mathcal A, \mu)$ be a measure space.
$D:=\{f\in {\mathcal F}_{\mathcal A}\colon f=0\ \mu\htam{-a.e.} \}$
is a Riesz ideal of~${\mathcal F}_{\mathcal A}$;
the quotient Riesz space is~$\mathrm{F}_{\mathcal A} (\mu)$.

The same~$D$ is a Riesz ideal of~$\mathcal{L}_{\mathcal A}[\mu]$;
the quotient~${\mathcal L}_{\mathcal A}[\mu]/D$ is denoted
\begin{equation*}
\mathrm{L}^1(\mu)\htam{.}
\end{equation*}

Further,
$\mathcal{L}_{\mathcal A}[\mu]$ 
is a Riesz ideal of~${\mathcal F}_{\mathcal A}$
(Maat en Integraal),
and~$\mathrm{L}^1(\mu)$ is a Riesz ideal of~$\mathrm{F}_{\mathcal A}(\mu)$.
%
\item \label{2.8-2}
$\ell^\infty(X)$ is a Riesz ideal of $\R^X$ (see~\ref{1.5}\ref{1.5-1}).
For a topological space~$X$, 
$\BCont{X}$ is a Riesz ideal 
of~$\Cont{X}$ (\ref{1.5}\ref{1.5-3}).
$\cseq_0$ and $\ell^1$ are Riesz ideals 
in~$\ell^\infty$ ($:=\ell^\infty(X)$).
%
\item \label{2.8-3}
Let $\mathcal A$ be a $\sigma$-algebra in a set $X$.
Then $\sigma(\mathcal A)$ is a Riesz ideal of~$\BA(\mathcal A)$.
Indeed, as we already know~$\sigma(\mathcal A)$ 
to be a Riesz subspace of~$\BA(\mathcal A)$,
it will be an ideal if
\begin{equation*}
\mu\in \sigma(\mathcal A)\htam{, }\ 
\tau\in\BA(\mathcal A)\htam{, }\ 
0\leq \tau\leq \mu
\quad \implies \quad
\tau \in \sigma(\mathcal A)
\htam{.}
\end{equation*}
But this follows from the observation
that an additive $\alpha\colon\mathcal A\ra [0,\infty)$
is a finite measure if and only if
\begin{equation*}
A_n \downarrow \eset\ \htam{ in }\mathcal A 
\quad \implies \quad
\alpha(A_n)\ra 0
\htam{.}
\end{equation*}
%
\item \label{2.8-4}
If $X$ is a topological space and $Y\subseteq X$, then
\begin{equation*}
\{f\in\Cont{X}\colon f | Y=0 \}
\end{equation*}
is a Riesz ideal of $\Cont{X}$.

Not all Riesz ideals of $\Cont{X}$ are of this type.
For instance,
in~$\Cont{\R}$ the functions~$f$
with $\lim_{x\ra \infty} x^3 f(x)=0$
form a Riesz ideal.
\end{enumerate}
\end{psec}
%
%                  2.9
%
\begin{psec}{2.9}{Exercise}
Let $X$ be a topological space
and $A$ a closed subset of~$X$ 
such that every continuous function on~$A$
can be extended to a continuous function on~$X$ 
(e.g., $X$ is compact Hausdorff or $X$ is metrizable,
and $A$ is any closed set).
Let~$J_A$ be the Riesz ideal
$\{f\in\Cont{X}\colon f|A=0\}$.
Show that the quotient Riesz space
$\Cont{X}/J_A$ is Riesz isomorphic to~$\Cont{A}$.
\end{psec}
%
%                  2.10
%
\begin{psec}{2.10}{Exercise}
\begin{enumerate}
\item \label{2.10-1}
Show that the sum of two Riesz subspaces 
of a Riesz space~$E$ need not be a Riesz subspace of~$E$.
(The space of all affine functions on~$[0,1]$ (see~\ref{1.7})
is a sum of two $1$-dimensional Riesz subspaces of~$\Cont{[0,1]}$.)
%
\item \label{2.10-2}
Prove that the sum of two Riesz ideals
of a Riesz space~$E$
is a Riesz ideal of~$E$.
(This is not so easy.
It may be useful to prove
the ``Riesz Decomposition Lemma'':
if~$x,a,b\in E^+$ and $x\leq a+b$,
then there exist $y,z\in E^+$ 
with $x=y+z$, $y\leq a$, $z\leq b$;
e.g., $y=x\wedge a$ and $z=x-y=x\vee a-a$.)
\end{enumerate}
\end{psec}
%
%                  2.11
%
\begin{psec}{2.11}{Definitions}
The intersection of any collection of Riesz subspaces
(or ideals) of a Riesz space~$E$ 
is again a Riesz subspace
(or ideal, respectively)
of~$E$.

For $u\in E^+$, 
the \keyword{principal (Riesz) ideal generated} by~$u$ 
is the intersection of all Riesz ideals
that contain~$u$;
it is the set
\begin{equation*}
\bigcup_{n\in\N} [-nu,nu]
\htam{.}
\end{equation*}

An element $u$ of $E^+$
is called 
a \keyword{(strong) unit} of~$E$
if the principal ideal it generates is~$E$ itself,
i.e., 
if for every~$x$ in~$E$
there exists an~$n$ in~$\N$ with $|x|\leq ne$.

$E$ is said to be \keyword{unitary} if it has a unit.
\end{psec}
%
%                  2.12
%
\begin{psec}{2.12}{Exercise}
\begin{enumerate}
\item \label{2.12-1}
Trivially,
the constant function $\mathbb{1}$
is a unit in~$\BCont{\R}$.
Show that~$\Cont{\R}$ has no unit.
%
\item \label{2.12-2}
Show that $\cseq_0$ and $\cseq$ are not Riesz isomorphic.
\end{enumerate}
\end{psec}
%
%                  2.13
%
\begin{psec}{2.13}{Definition}
If $u$ is a unit in a Riesz space~$E$, 
then
\begin{equation}
\label{eq2.13_1}
\tag{$*$}
x\in E\htam{, }\ x\perp u 
\quad \implies \quad
x=0
\htam{.}
\end{equation}
(Indeed, there is an $n$ in $\N$ with $|x|\leq nu$;
then $|x|=|x|\wedge nu\leq n|x|\wedge nu=n(|x|\wedge u)=0$.)

An element $u$ of $E$ 
is called
a $\keyword{weak unit}$,
or a $\keyword{Freudenthal unit}$,
if~\eqref{eq2.13_1} holds.
The constant function~$\mathbb{1}$
is a weak unit in~$\Cont{\R}$.
\end{psec}
%
%                  2.14
%
\begin{psec}{2.14}{Exercise}
Prove the following \textbf{Theorem}: \ %
\statement{%
If $E$ is a Riesz space, $a\in E$, and
\begin{equation*}
a^\perp := \{ x\in E\colon x\perp a \} \htam{, }
\end{equation*}
then $a^\perp$ is a Riesz ideal of~$E$}.
\end{psec}
%
%                  2.15
%
\begin{psec}{2.15}{Definition}
Let $X$ be a set. 
Every point $a$ of $X$ determines
a Riesz homomorphism
$\delta_a\colon \R^X\ra\R$,
the \keyword{evaluation} at~$a$, by
\begin{equation*}
\delta_a(f) = f(a)\qquad (f\in \R^X)
\htam{.}
\end{equation*}
We use the same symbol ``$\delta_a$''
to denote the restriction of~$\delta_a$
to any Riesz subspace of~$\R^X$.
\end{psec}
%
%                  2.16
%
\begin{psec}{2.16}{Theorem}
\statement{%
Let $X$ be a compact Hausdorff space.
Then $a \leftrightarrow \delta_a$
is a bijective correspondence
between the points of~$X$
and the Riesz homomorphisms
$\varphi\colon \Cont{X}\ra \R$ 
with~$\varphi(\mathbb{1})=1$.
}
\end{psec}
\begin{proof}
Let $\varphi$ be a Riesz homomorphism $\Cont{X}\ra \R$
with~$\varphi(\mathbb{1})=1$.
We prove $\varphi=\delta_a$ for some~$a$ in~$X$.
(The rest of the theorem 
we leave to the reader.)

Suppose there is no such~$a$.
Then for every~$a\in X$,
we can choose an~$f_a$ in~$\Cont{X}$
with $\varphi(f_a)\neq f_a(a)$.
Setting $g_a := | f_a - \varphi(f_a) \mathbb{1}|$
we have~$g_a \in \Cont{X}^+$,
$\varphi(g_a)=0$,
$g_a(a)>0$.
By compactness,
there exist $a_1,\dotsc,a_N$ in~$X$ such that
\begin{equation*}
X = \bigcup_n \{ x\colon g_{a_n} (x) > 0 \}\htam{.}
\end{equation*}
Put $g:=g_{a_1}\vee \dotsb \vee g_{a_N}$.
We have $\varphi(g) =0$ 
but $g(x)>0$ for all~$x\in X$.
There is a positive number~$\varepsilon$
with $g\geq \varepsilon \mathbb{1}$.
Then $\varphi(g)\geq \varepsilon \varphi(\mathbb{1})=\varepsilon$.
Contradiction. \xqed
\end{proof}
%
%                  2.17
%
\begin{psec}{2.17}%
Let $X$ be as above 
and let~$\varphi$ be any nonzero Riesz homomorphism.
There is an~$h$ in~$\Cont{X}$
with $\varphi(h)\neq 0$.
There is a positive number~$c$
with $|h|\leq c\mathbb{1}$.
Then $0<|\varphi(h)|=\varphi(|h|)\leq c\varphi(\mathbb{1})$,
so that $\varphi(\mathbb{1})>0$.
By applying the preceding proof to
the homomorphism $f\mapsto \varphi(\mathbb{1})^{-1} \varphi(f)$
we find that there is an~$a$ in~$X$ 
with $\varphi = \varphi(\mathbb{1})\delta_a$.
\end{psec}
%
%                  2.18
%
For an application of Theorem~\ref{2.16}
we need a result from topology:
\begin{psec}{2.18}{Lemma}
\statement{%
Let $X$ be a compact Hausdorff space,
$(x_i)_{i\in I}$ a net in~$X$,
and let~$x\in X$.
Then
\begin{equation*}
x_i \ra x 
\quad \iff \quad
f(x_i)\ra f(x)
\ \htam{ for every }f\htam{ in }\Cont{X}\htam{.}
\end{equation*}
}
\end{psec}
\begin{proof}
The implication ``$\Rightarrow$'' is trivial.
Now assume $f(x_i)\ra f(x)\quad (f\in \Cont{X})$.
Let~$U$ be an open subset of~$X$ containing~$x$.
As~$X$ is compact and Hausdorff,
there is an~$f$ in~$\Cont{X}$ with
\begin{equation*}
f(x)=1\htam{, } 
\qquad f=0\ \htam{ on }X\backslash U
\htam{.}
\end{equation*}
Then $f(x_i)\ra 1$, 
so there is an~$i_0$ in~$I$ with the property
that for all~$i$ in~$I$ with $i\ge i_0$
we have $f(x_i)\neq 0$, whence $x_i \in U$. \xqed
\end{proof}
%
%                  2.19
%
\begin{psec}{2.19}{Theorem (Kaplansky)}
\statement{%
Let $X$ and $Y$ be compact Hausdorff spaces
such that $\Cont{X}$ and $\Cont{Y}$ are Riesz isomorphic.
Then~$X$ and~$Y$ are homeomorphic.
}
\end{psec}
\begin{proof}
Let $T\colon \Cont{X}\ra \Cont{Y}$ be a Riesz isomorphism.
\begin{enumerate}[label=(\Roman*)]
\item \label{2.19-I}
For the moment, assume 
$T\mathbb{1}_X = \mathbb{1}_Y$.
For every~$y\in Y$ 
we can apply~\ref{2.16} to the Riesz homomorphism
$f\mapsto (Tf)(y)\quad (f\in\Cont{X})$.
Thus we obtain a map $\tau\colon Y\ra X$ such that
\begin{alignat*}{2}
(Tf)(y) &= f(\tau(y)) 
\qquad &(f\in\Cont{X}\htam{, }\ y\in Y)
\htam{.}\\
\intertext{%
In the same way (using $\tau^{-1} \mathbb{1}_Y = \mathbb{1}_X$)
we obtain a $\sigma\colon X\ra Y$ with}
(T^{-1}g)(x) &= g(\sigma(x)) 
 &(g\in \Cont{Y}\htam{, }\ x\in X)
\htam{.}
\end{alignat*}

If $x\in X$, then for every~$f$ in $\Cont{X}$ we have
\begin{equation*}
f(x) = (T^{-1} T f)(x) = (Tf)(\sigma(x)) = f(\tau(\sigma (x)))\htam{,}
\end{equation*}
so that $x=\tau(\sigma(x))$.
Similarly, $y=\sigma(\tau(y))$ for all $y\in Y$.
Thus, $\sigma$ and~$\tau$ are bijective and each other's inverses.

If $(x_i)_{i \in I}$ is a net in~$X$,
converging to a point~$x$,
then for all~$g$ in $\Cont{Y}$
we have $g(\sigma(x_i))=(T^{-1} g)(x_i)\ra (T^{-1}g)(x)=g(\sigma(x))$,
so $\sigma(x_i)\ra \sigma(x)$ according to Lemma~\ref{2.18}.
We see that~$\sigma$ is continuous;
and so is~$\tau$.
Then $\sigma$ and $\tau$ are homeomorphisms.
%
\item \label{2.19-II}
For the general case,
observe that~$T^{-1}\mathbb{1}_Y$ is bounded:
there is a positive number~$c$ 
with ~$T^{-1}\mathbb{1}_Y \leq c \mathbb{1}_X$.
This implies $T\mathbb{1}_X \ge c^{-1} \mathbb{1}_Y$.
In particular,
all values of~$T\mathbb{1}_X$ are strictly positive.
Then
\begin{equation*}
f \mapsto \frac{Tf}{T\mathbb{1}_X}
\end{equation*}
is a Riesz isomorphism $\Cont{X}\ra\Cont{Y}$
sending~$\mathbb{1}_X$ to~$\mathbb{1}_Y$.
By~\ref{2.19-I}, 
$X$ and~$Y$ are homeomorphic. \xqed
\end{enumerate}
\end{proof}

In~\ref{2.20} and~\ref{2.21} 
we consider two variations
of Kaplansky's Theorem.
%
%                  2.20
%
\begin{psec}{2.20}{Theorem (Gelfand--Kolmogorov)}
\statement{%
Let $X$ and $Y$ be compact Hausdorff spaces
and suppose there is a linear bijection $T\colon \Cont{X}\ra\Cont{Y}$
that is multiplicative:
\begin{equation*}
T(fg) = Tf \cdot Tg 
\qquad (f,g\in \Cont{X}\,)
\htam{.}
\end{equation*}
Then $X$ and $Y$ are homeomorphic.
}
\end{psec}
\begin{proof}
If $f\in\Cont{X}^+$,
then $f = \sqrt{f}^2$,
so that $Tf = \smash{(T\sqrt{f})^2}\in \Cont{Y}^+$.
It follows that~$T$
is increasing.
So is $T^{-1}$.
Thus, $T$ is an order isomorphism
and thereby a Riesz isomorphism.
Now apply~\ref{2.17}. \xqed
\end{proof}
%
%                  2.21
%
\begin{psec}{2.21}{Theorem (Banach--Stone)}
\statement{%
Let $X$ and $Y$ be compact Hausdorff spaces
and suppose there is a linear
bijection $T\colon \Cont{X} \ra \Cont{Y}$
that is an isometry relative to the sup-norm, i.e.,
\begin{equation*}
\| Tf -Tg \|_\infty = \| f-g \|_\infty 
\qquad (f,g\in\Cont{X}\,)
\htam{.}
\end{equation*}
Then $X$ and $Y$ are homeomorpic.
}
\end{psec}
\begin{proof}
\begin{enumerate}[label=(\Roman*)]
\item \label{2.21-I}
First,
assume $T\mathbb{1}_X = \mathbb{1}_Y$.

Let $a\in\R$, $c\in[0,\infty)$,
For $f\in \Cont{X}$ we have:
\begin{align*}
f(X)\subseteq 
  &\  [a-c,a+c] \iff \| f-a\mathbb{1}_X \|_\infty \leq c \\
  & \iff \|Tf - a\mathbb{1}_Y\|_\infty \leq c \\
  & \iff (Tf)(Y) \subseteq [a-c,a+c]
\htam{.}
\end{align*}
It follows easily that $T$ maps $\Cont{X}^+$ onto $\Cont{Y}^+$,
and we can proceed as in~\ref{2.20}.
\item \label{2.21-II}
For the general case,
set~$u:=T\mathbb{1}_X$,
we prove $|u|=\mathbb{1}_Y$.
(Then $f\mapsto u^{-1} Tf$
is a linear bijection $\Cont{X}\ra\Cont{Y}$ 
that is isometric and sends~$\mathbb{1}_X$ to~$\mathbb{1}_Y$,
so that we can apply~\ref{2.21-I}.)

Observe that for~$f$ in~$\Cont{X}$
\begin{equation}
\label{eq2.21} \tag{$*$}
|f|\leq \mathbb{1}_X
\iff
\|f\|_\infty \leq 1
\iff
\|Tf \|_\infty \leq 1
\iff
|Tf| \leq \mathbb{1}_Y
\htam{.}
\end{equation}

Put $w:=\mathbb{1}_Y - |u|$;
we prove $w=0$.
By~\eqref{eq2.21},
$|u|\leq\mathbb{1}_Y$, 
so $w\ge 0$.
Then $|u+w|\leq |u|+w=\mathbb{1}_Y$,
so that (with~\eqref{eq2.21})
$|\mathbb{1}_X + T^{-1} w|\leq \mathbb{1}_X$,
and $T^{-1} w\leq 0$.
But also $|u-w|\leq |u|+w=\mathbb{1}_Y$,
$|\mathbb{1}_X - T^{-1} w|\leq \mathbb{1}_X$,
and $T^{-1} w\ge 0$.
Hence, $T^{-1} w=0$.
Thus, $w=0$. \xqed
\end{enumerate}
\end{proof}
\clearpage
\end{document}
