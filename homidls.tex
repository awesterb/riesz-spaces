\documentclass[main.tex]{subfiles}
\begin{document}
% 2
\section{Homomorphisms, ideals}
%
%                  2.1
%
\begin{psec}{2.1}%
(See \ref{1.5}\ref{1.5-8}.) 
Let $(X,\mathcal A, \mu)$ be a measure space.
For an $\mathcal A$-measurable 
function $f\colon X\ra \R$
denote by~$(f)_\mu$
the set of all $\mathcal A$-measurable functions
that are equal to~$f$ $\mu$-a.e.,
and put 
$F_{\mathcal A}(\mu):=\{(f)_\mu\colon f\in\mathcal{F}_{\mathcal A}\}$.
Then~$F_{\mathcal A}(\mu)$ 
is a Riesz-space
under the ordering
\begin{align*}
& (f)_\mu \leq (g)_\mu \quad \iff \quad f\leq g\ \mu\htam{-a.e.,} \\
\intertext{%
the lattice operations begin given by}%
& (f)_\mu \vee (g)_\mu = (f\vee g)_\mu \htam{, }
  \qquad (f)_\mu \wedge (g)_\mu = (f\wedge g)_\mu
\htam{.}
\end{align*}
For a general approach to similar constructions
it pays to investigate ``Riesz homomorphisms'' 
and ``Riesz ideals''.
\end{psec}
%
%                  2.2
%
\begin{psec}{2.2}{Definitions}
Let $E$ and $F$ be Riesz spaces 
and $T$ a linear map $E\ra F$.
We call~$T$ \keyword{positive} 
if it is order preserving ($=\htam{increasing}$),
which is the case if and only if
\begin{equation*}
T(E^+) \subseteq F^+\htam{, }
\end{equation*}
whereas $T$ is a \keyword{Riesz homomorphism} 
it is (linear and) a 
lattice homomorphism, i.e.,
\begin{equation*}
T(x\vee y) = Tx \vee Ty\htam{,}
\quad T(x\wedge y)=Tx\wedge Ty
\qquad (x,y\in E)\htam{.}
\end{equation*}
Every Riesz homomorphism is positive. 
($x\in E^+\iff x=x\vee 0$)

We leave it to the reader to define ``Riesz isomorphism''
and ``Riesz isomorphic''.

A linear (bijective) order isomorphism is,
of course,
a Riesz isomorphism,
but some positive linear bijections are not
(e.g., $(x_1,x_2)\mapsto(x_1,x_1+x_2)$ as a map
$\R^2\ra\R^2$).
\end{psec}
%
%                  2.3
%
\begin{psec}{2.3}{Exercise}
Let $E$ and $F$ be Riesz spaces,
$T$ a linear map $E\ra F$.
Prove the equivalence of:
\begin{enumerate}
\item[$(\alpha)$]  \label{2.3-alpha}
$T$ is a Riesz homomorphism.
%
\item[$(\beta)$]  \label{2.3-beta}
$T(|x|)=|Tx| \quad (x\in E)$.
%
\item[$(\gamma)$]  \label{2.3-gamma}
$T(x^+) = \lsub{(Tx)^+}{(Tx)} \quad (x\in E)$.
\end{enumerate}
\end{psec}
%
%                  2.4
%
\begin{psec}{2.4}%
Let $E$ be a Riesz space and $T$ a Riesz homomorphism of $E$
into a Riesz space~$F$.
Consider
 $D:=\{x\in E\colon Tx=0\}$.
This~$D$ is a linear subspace of~$E$ 
with the additional property
\begin{equation}
\tag{$*$}
\label{eq2.4}
a\in D\htam{, }\; x\in E\htam{, }\; |x|\leq |a| 
  \quad \implies \quad x\in D\htam{. }
\end{equation}
($|Tx|=T(|x|)\leq T(|a|)=|Ta|=0$.)  
Conversely:
\end{psec}
%
%                  2.5
%
\begin{psec}{2.5}{Exercise}
Let $E$ be a Riesz space and $D$ a linear subspace of~$E$
with the property~\eqref{eq2.4} of~\ref{2.4}.
Show that there is one (and only one)
ordering on the quotient vector space $E/D$ 
that turns $E/D$ into a Riesz space 
and the quotient map
$E\ra E/D$ into a Riesz homomorphism.
\end{psec}
%
%                  2.6
%
\begin{psec}{2.6}{Definition}
A \keyword{Riesz ideal} of a Riesz space $E$ 
is a linear subspace~$D$ of~$E$ for which
\begin{equation*}
a\in D\htam{, }\; x\in E\htam{, }\; |x|\leq |a| 
  \quad \implies \quad x\in D\htam{. }
\end{equation*}
\end{psec}

Observe that a Riesz ideal is a Riesz subspace. 
(Take $x=|a|$.)
A Riesz subspace (not just a linear subspace) $D$ of $E$
is a Riesz ideal if and only if 
\begin{equation*}
a\in D^+\htam{, }\; x\in E^+\htam{, }\; x\leq a 
  \quad \implies \quad x\in D\htam{. }
\end{equation*}
\clearpage
\end{document}
