\documentclass[main.tex]{subfiles}
\begin{document}
% 5
\section{Hermitian Operators}
We start with some basic facts dealing 
with normed vector spaces and Hilbert spaces.
%
%                  5.1
%
\begin{psec}{5.1}%
Let $E$ and $F$ be vector spaces,
endowed with norms $\|\cdot\|_E$ and $\|\cdot\|_F$, respectively.
\begin{enumerate}
\item \label{5.1-1}
Let $T$ be a linear map $E\ra F$.

If there exists a number $c\in [0,\infty)$ such that
\begin{equation*}
\|Tx\|_F \ \leq \ c\,\|x\|_E \qquad (x\in E)
\end{equation*}
then $\|Tx-Ty\|_F \leq \|x-y\|_E\quad (x,y\in E)$,
so that~$T$ is continuous
(relative to the metrics induced by the norms).

On the other hand,
if no such~$c$ exists,
$T$ is \emph{not} continuous.
Indeed,
for every $n\in\N$
there then exists an~$x_n$ in~$E$
with $\|T x_n\|_F > n \|x_n\|_E$.
Then~$x_n\neq 0$, so $\|x_n\|_E\neq 0$.
Putting $y_n:=(n\|x_n\|_E)^{-1}x_n$
we get $y_n\ra 0$
but not $Ty_n\ra 0=T0$.
%
\item \label{5.1-2}
The continuous linear maps $E\ra F$ form a vector space
\begin{equation*}
\Lin(E,F)\htam{.}
\end{equation*}
With every $T\in\Lin(E,F)$ 
we associate a nonnegative number $\| T \|$ by
\begin{equation*}
\|T\| = \inf \bigl\{ c\colon 
\|Tx\|_F \leq c\,\|x\|_E\htam{ for all }x\in E\bigr\}
\end{equation*}
Then
\begin{equation*}
\|Tx\|_F\ \leq\ \|T\|\,\|x\|_E \qquad (T\in\Lin(E,F),\ x\in E)\htam{.}
\end{equation*}
It is not difficult to verify
that~$\|\cdot\|$ is a norm on $\Lin(E,F)$.
It is called the \keyword{operator norm}.
An important fact:
\statement{$\Lin(E,F)$ is complete relative to the operator norm
in case~$F$ is complete relative to~$\|\cdot\|_F$.}
\end{enumerate}
\end{psec}
%
%                  5.2
%
\begin{psec}{5.2}%
An inner product $\left<\cdot,\cdot\right>$
on a vector space~$H$
induces a norm~$\|\cdot\|$:
\begin{equation*}
\|x\|:=\sqrt{\left<x,x\right>}\qquad(x\in H)
\end{equation*}
satisfying the \keyword{Cauchy--Schwarz Inequality}
\begin{equation*}
\left|\left<x,y\right>\right| \ \leq\ \|x\|\,\|y\|\qquad(x,y\in H)\htam{.}
\end{equation*}

Relative to this norm,
for each $x\in H$
the function $y\mapsto \left<x,y\right>$ is continuous
(since $\left|\left<x,y_1\right>-\left<x,y_2\right>\right|
\leq\|x\|\,\|y_1 - y_2\|$ for all $y_1,y_2$).

The inner product space $(H,\left<\cdot,\cdot\right>)$ 
is called a \keyword{Hilbert space}
if~$H$ is complete relative to the inner product norm.
\end{psec}
\clearpage
\end{document}
