\documentclass[main.tex]{subfiles}
\begin{document}
% 5
\section{Hermitian Operators}
We start with some basic facts dealing 
with normed vector spaces and Hilbert spaces.
%
%                  5.1
%
\begin{psec}{5.1}%
Let $E$ and $F$ be vector spaces,
endowed with norms $\|\cdot\|_E$ and $\|\cdot\|_F$, respectively.
\begin{enumerate}
\item \label{5.1-1}
Let $T$ be a linear map $E\ra F$.

If there exists a number $c\in [0,\infty)$ such that
\begin{equation*}
\|Tx\|_F \ \leq \ c\,\|x\|_E \qquad (x\in E)
\end{equation*}
then $\|Tx-Ty\|_F \leq \|x-y\|_E\quad (x,y\in E)$,
so that~$T$ is continuous
(relative to the metrics induced by the norms).

On the other hand,
if no such~$c$ exists,
$T$ is \emph{not} continuous.
Indeed,
for every $n\in\N$
there then exists an~$x_n$ in~$E$
with $\|T x_n\|_F > n \|x_n\|_E$.
Then~$x_n\neq 0$, so $\|x_n\|_E\neq 0$.
Putting $y_n:=(n\|x_n\|_E)^{-1}x_n$
we get $y_n\ra 0$
but not $Ty_n\ra 0=T0$.
%
\item \label{5.1-2}
The continuous linear maps $E\ra F$ form a vector space
\begin{equation*}
\Lin(E,F)\htam{.}
\end{equation*}
With every $T\in\Lin(E,F)$ 
we associate a nonnegative number $\| T \|$ by
\begin{equation*}
\|T\| = \inf \bigl\{ c\colon 
\|Tx\|_F \leq c\,\|x\|_E\htam{ for all }x\in E\bigr\}\htam{.}
\end{equation*}
Then
\begin{equation*}
\|Tx\|_F\ \leq\ \|T\|\,\|x\|_E \qquad (T\in\Lin(E,F),\ x\in E)\htam{.}
\end{equation*}
It is not difficult to verify
that~$\|\cdot\|$ is a norm on $\Lin(E,F)$.
It is called the \keyword{operator norm}.
An important fact:
\statement{$\Lin(E,F)$ is complete relative to the operator norm
in case~$F$ is complete relative to~$\|\cdot\|_F$.}
\end{enumerate}
\end{psec}
%
%                  5.2
%
\begin{psec}{5.2}%
An inner product $\left<\cdot,\cdot\right>$
on a vector space~$H$
induces a norm~$\|\cdot\|$:
\begin{equation*}
\|x\|:=\sqrt{\left<x,x\right>}\qquad(x\in H)
\end{equation*}
satisfying the \keyword{Cauchy--Schwarz Inequality}
\begin{equation*}
\left|\left<x,y\right>\right| \ \leq\ \|x\|\,\|y\|\qquad(x,y\in H)\htam{.}
\end{equation*}

Relative to this norm,
for each $x\in H$
the function $y\mapsto \left<x,y\right>$ is continuous
(since $\left|\left<x,y_1\right>-\left<x,y_2\right>\right|
\leq\|x\|\,\|y_1 - y_2\|$ for all $y_1,y_2$).

The inner product space $(H,\left<\cdot,\cdot\right>)$ 
is called a \keyword{Hilbert space}
if~$H$ is complete relative to the inner product norm.
\end{psec}
%
%                  5.3
%
\begin{psec}{5.3}%
Let~$E$ be a vector space.
\begin{enumerate}%
\item\label{5.3-1}%
A \keyword{semi-inner product} on~$E$ is a function $[\cdot,\cdot]$
on $E\times E$ satisfying
\begin{enumerate}[label={},itemindent=2em,labelindent=2em]
\item $y\mapsto [x,y]$ is linear for each $x$;
\item $[x,y]=[y,x]\qquad(x,y\in E)$;
\item $[x,x]\geq 0\qquad(x\in E)$.
\end{enumerate}
These properties suffice to 
imply a \keyword{Cauchy--Schwarz Inequality}:
\begin{equation*}
|[x,x]|\ \leq\ \sqrt{[x,x]}\,\sqrt{[y,y]}
\end{equation*}

For example,
let $E$ be the space of all Riemann integrable functions
on the interval~$[0,1]$, with
\begin{equation*}
[f,g]\ :=\ \int_0^1 f(x)g(x)\,\mathrm{d}x\htam{.}
\end{equation*}
%
\item \label{5.3-2}
Let $[\cdot,\cdot]$ be such a semi-inner product on~$E$.
The elements~$x$ of~$E$ for which $[x,x]=0$
form a linear subspace~$D$ of~$E$.
Make the quotient vector space~$E/D$
and the quotient map $Q\colon E\ra E/D$.
Then the formula
\begin{equation*}
\left< Qx,Qy \right> = [x,y]\qquad (x,y\in E)
\end{equation*}
determines a bona fide inner product on~$E/D$.
\end{enumerate}
\end{psec}

\noindent {\sc From here on in this chapter
$H$ is a Hilbert space.}

%
%                  5.4
%
\begin{psec}{5.4}{Definition}
A map $T\colon H\ra H$ is called a \keyword{Hermitian operator} if
\begin{equation*}
\left\{\ 
\begin{alignedat}{2}
&T\htam{ is linear and continuous;} \\
&\htam{if }x,y\in H\htam{, then }
\left<Tx,y\right>=\left<x,Ty\right>\htam{.}
\end{alignedat}
\right.
\end{equation*}
(It can be proved that a linear $T\colon H\ra H$ 
with $\left<Tx,y\right>=\left<x,Ty\right>$
for all~$x$ and~$y$ must be continuous.)

The Hermitian operators form a vector space
\begin{equation*}
\mathscr H\htam{.}
\end{equation*}

$\mathscr H$ is a linear subspace of $\Lin(H,H)$,
closed under the operator norm. (See~\ref{5.1}\ref{5.1-2}.)
Indeed,
if $(T_n)_n$ is a sequence in~$\mathscr H$,
converging to an element~$T$ of $\Lin(H,H)$
in the sense of the operator norm,
then $T_n x\rightarrow Tx$ for every $x\in H$,
so $\left< Tx, y \right> 
= \lim\left< T_n x, y \right>
= \lim\left< x, T_n y \right>
= \left< x, Ty \right>\quad (x,y\in H)$.
\end{psec}

In \ref{5.5}\ref{5.5-1} we will see
that the product of two Hermitian operators
need not be Hermitian.
%
%                  5.5
%
\begin{psec}{5.5}{Examples}
\begin{enumerate}
\item \label{5.5-1}
$\R^2$ is a Hilbert space under the standard inner product.
All linear maps $\R^2\ra\R^2$ are continuous.
If we identify a linear map $\R^2\ra\R^2$ with its matrix,
the Hermitian operators in~$\R^2$ are precisely the matrices
\begin{equation*}
\begin{pmatrix}
a & b \\
b & c 
\end{pmatrix}\htam{.}
\end{equation*}
The identity
\begin{equation*}
\begin{pmatrix} 1 & 0 \\ 0 & 0 \end{pmatrix}
\begin{pmatrix} 1 & 1 \\ 1 & 1 \end{pmatrix}
=
\begin{pmatrix} 1 & 1 \\ 0 & 0 \end{pmatrix}
\end{equation*}
shows that a product of two Hermitian operators
may not be Hermitian.
%
\item \label{5.5-2}
By $\ell^2$ we denote the space of all real number sequences
$x=(x_n)_n$ for which $\sum x_n^2<\infty$.
This~$\ell^2$ is a Riesz ideal in~$\ell^\infty$.

It is not quite trivial to prove that~$\ell^2$
is a Hilbert space under
\begin{equation*}
\left<x,y\right>\ :=\ \sum_{n=1}^\infty x_n y_n \qquad (x,y\in\ell^2)\htam{.}
\end{equation*}

Continuous linear maps $\ell^2\ra\ell^2$
can be presented by infinitely large matrices;
Hermitian operators correspond to symmetric matrices.
(Not every infinitely large matrix denotes a map $\ell^2\ra\ell^2$.)
%
\item \label{5.5-3}
On $\Cont{[0,1]}$ we have the inner product
\begin{equation*}
\left< f,g\right>\ = \ \int_0^1 f(x)g(x)\,\mathrm{d}x
\qquad(f,g\in\Cont{[0,1]}\,)\htam{.}
\end{equation*}
Under this inner product,
$\Cont{[0,1]}$ is not complete.
However,
as a metric space
(with the inner product metric),
it has a completetion.
For esoteric reasons of our own
we call this completion~$\Leb^2[0,1]$.
It turns out to be possible (and not too difficult)
to extend the addition, scalar multiplication and inner product
of $\Cont{[0,1]}$ to $\Leb^2[0,1]$,
rendering $\Leb^2[0,1]$ a Hilbert space.
(In a later chapter we will see how $\Leb^2[0,1]$
can be realized as a space of functions on~$[0,1]$.)

A linear map $\Cont{[0,1]}\ra\Cont{[0,1]}$
that is continuous relative to the inner product norm
has a unique continuous linear extension
$\Leb^2[0,1]\ra\Leb^2[0,1]$.
Examples of continuous linear maps
$\Cont{[0,1]}\ra\Cont{[0,1]}$ that
lead to Hermitian operators in $\Leb^2[0,1]$ are
\begin{align*}
(T_1 f)(x) &= x\,f(x)\htam{,} \\
(T_2 f)(x) &= \textstyle{\int_0^1} x\wedge t\,f(t)\,\mathrm{d}t \htam{,} \\
(T_3 f)(x) &= f(1-x)\htam{.} \\%
\intertext{%
Every $u\in\Cont{[0,1]}$ induces a Hermitian 
``multiplication operator''~$M_u$
in $\Leb^2[0,1]$;
for $f\in \Cont{[0,1]}$,
$M_uf$ is given by
}%
(M_uf)(x) &= u(x)\,f(x)\htam{.}
\end{align*}
\end{enumerate}
\end{psec}
%
%                  5.6
%
\begin{psec}{5.6}%
\begin{enumerate}
\item \label{5.6-1}
For $x,y\in H$ we define
\begin{equation*}
x\perp y \quad \iff \quad \left<x,y\right>=0\htam{.}
\end{equation*}
If $x\perp y$ we have the Pythagoras Formula
\begin{equation*}
\|x+y\|^2\ =\ \|x\|^2 + \|y\|^2\htam{.}
\end{equation*}
%
\item \label{5.6-2}
For a subset $X$ of $H$ the set
\begin{equation*}
X^\perp\ =\ \bigl\{ y\in H\colon \left<x,y\right>=0\quad(x\in H)\bigr\}
\end{equation*}
is a linear subspace of~$H$.
The fact that for every~$x$
the function $y\mapsto\left<x,y\right>$
is continuous (\ref{5.2}) 
implies that~$X^\perp$ is closed.
Thus $X^{\perp\perp}$ ($:=(X^\perp)^\perp$)
is a closed linear subspace of~$H$, containing~$X$.
%
\item \label{5.6-3}
\textbf{Projection Theorem}\  (very non-trivial)
\statement{Let $D$ be a closed linear subspace of~$H$.
Then
\begin{equation*}
\frmd{D\cap D^\perp = \{0\}\htam{, } \quad D+D^\perp = H}
\end{equation*}
and
\begin{equation*}
\frmd{D^{\perp\perp}=D}\htam{.}
\end{equation*}}
%
\item \label{5.6-4}
Continuing~\ref{5.6-3}:
There is a map $P_D\colon H\ra H$ determined by
\begin{equation*}
P_D x\in D\htam{, }\quad x-P_Dx\in D^\perp\qquad (x\in H)\htam{.}
\end{equation*}
This $P_D$,
called the \keyword{projection onto}~$D$,
is linear.
Evidently,
$I-P_D$ is the projection onto~$D^\perp$.

By the Pythagoras Formula,
$\|x\|^2 = \|P_D x\|^2 + \|x-P_Dx\|^2 \geq\|P_D x\|^2$
for all~$x$,
so that~$P_D$ is continuous and
\begin{equation*}
\| P_D \| \leq 1\htam{.}
\end{equation*}

Furthermore,
for all $x,y\in H$
we have $P_D x\perp y-P_D y$,
which implies $\left<P_D x,y\right>=\left<P_Dx,P_Dy\right>$.
Similarly, $\left<x,P_Dy\right>=\left<P_Dx,P_Dy\right>$,
so that $\left<P_Dx,y\right>=\left<x,P_Dy\right>$:
\begin{equation*}
\frmd{P_D \htam{ is Hermitian}}\htam{.}
\end{equation*}
\end{enumerate}
\end{psec}
%
%                  5.7
%
\begin{psec}{5.7}{Exercise}
Let $T$ be a linear map $H\ra H$.
Prove: $T$ is a projection
if and only if~$T\in\mathscr H$ and $T=T^2$.
(Hint: Consider $D:=\{x\colon (I-T)x=0\}$.)
\end{psec}
%
%                  5.8
%
\begin{psec}{5.8}%
With each $T$ in~$\mathscr H$
we associate the function 
$x\mapsto\left<Tx,x\right>$ on~$H$.
This function determines~$T$.
Indeed:

Let $T_1,T_2\in\mathscr H$,
$\left< T_1 x,x\right> = \left<T_2 x,x\right>\quad (x\in H)$;
we prove $T_1 = T_2$.
With $T:=T_1 - T_2$
we have $T\in\mathscr H$,
$\left< Tx,x\right>=0\quad (x\in H)$
and we wish to prove~$T=0$.
Now~$T$ determines a semi-inner product $[\cdot,\cdot]$
by: $[x,y]:=\left<Tx,y\right>$,
and $[x,x]=0$ for all~$x$.
The Cauchy--Schwarz Inequality then entails $[x,y]=0$,
i.e. $\left<Tx,y\right>=0$, for all~$x$ and~$y$.
Choosing $y=Tx$ we find $Tx=0$ for every~$x$.
\end{psec}
\clearpage
\end{document}
